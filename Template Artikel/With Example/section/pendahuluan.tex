\section{Pendahuluan}

%=========================================================%
%              TULIS ISI PENDAHULUAN DI SINI              %
% Jangan Lupa untuk Menghapus Contoh Tulisan di Bawah Ini %
%=========================================================%

Ini adalah \textit{template} artikel untuk tugas tutorial \textit{online}. Template artikel ini hampir disamakan dengan pemformatan artikel biasa---teks teralur secara mulus, tanpa menggunakan bab, dan tanpa loncatan ke halaman baru. Jika tugas menyuruh Anda menulis artikel, tetapi ingin menggunakan bagian bab (misal Bab I, Bab II, dsb.) yang biasa dilompat di halaman baru, jangan gunakan \textit{template} ini. Gunakanlah \textit{template} makalah!

\subsection{Latar Belakang}

Proses penulisan karya ilmiah seperti makalah, skripsi, atau tesis sering kali membutuhkan format penulisan yang baku dan konsisten. Konsistensi ini meliputi gaya penulisan, penomoran halaman, daftar isi, daftar pustaka, hingga format kutipan. Namun, banyak penulis yang masih menghadapi kesulitan dalam mengatur format tersebut secara manual, terutama ketika terjadi perubahan pada isi dokumen. Pengaturan format manual ini tidak hanya memakan waktu, tetapi juga rentan terhadap kesalahan, sehingga dapat mengurangi fokus penulis pada substansi konten.

Penggunaan LaTeX menawarkan solusi yang efektif untuk mengatasi masalah ini. Sebagai \textit{typesetting system}, LaTeX dirancang untuk menghasilkan dokumen berkualitas tinggi dengan tata letak yang profesional dan konsisten. Dengan memanfaatkan class atau template yang sudah ada, penulis dapat memisahkan fokus antara konten dan presentasi. Oleh karenanya, dibuatlah template LaTeX yang dapat mempermudah proses penulisan artikel, sehingga penulis dapat lebih fokus pada isi tulisan dan penelitian yang dilakukan.

\subsection{Rumusan Masalah}

Berdasarkan latar belakang yang telah diuraikan, rumusan masalah dalam makalah ini adalah:

\begin{enumerate}
    \item Bagaimana cara merancang template LaTeX yang dapat memenuhi standar format penulisan yang umum digunakan di Indonesia?
    \item Bagaimana template ini dapat membantu penulis untuk mengatur tata letak, daftar isi otomatis, penomoran halaman, dan daftar pustaka secara efisien?
    \item Fitur-fitur apa saja yang perlu diimplementasikan dalam template LaTeX ini agar dapat mempermudah proses penulisan karya tulis secara keseluruhan?
\end{enumerate}

\subsection{Tujuan}

Tujuan dari penyusunan template LaTeX ini adalah:

\begin{enumerate}
    \item Menciptakan sebuah template LaTeX yang konsisten, profesional, dan mudah digunakan untuk penulisan artikel, sesuai dengan standar yang berlaku.
    \item Menyediakan solusi praktis bagi mahasiswa dan akademisi agar dapat menyusun artikel dengan lebih cepat dan efisien, tanpa perlu khawatir tentang format penulisan.
    \item Memperkenalkan dan mempopulerkan penggunaan LaTeX sebagai alat bantu yang efektif dalam penulisan karya tulis.
\end{enumerate}