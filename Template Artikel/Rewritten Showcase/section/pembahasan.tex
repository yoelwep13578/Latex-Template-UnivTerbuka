\section{Pembahasan}

%=========================================================%
%               TULIS ISI PEMBAHASAN DI SINI              %
% Jangan Lupa untuk Menghapus Contoh Tulisan di Bawah Ini %
%=========================================================%

\subsection{Peran Keluarga dalam Membentuk Demokrasi yang Beradab}

\subsubsection{Keluarga sebagai Sarana Pendidikan Pertama}

Pada dasarnya, anak-anak akan mempelajari segala hal melalui keluarga. Beberapa teori sebelumnya menjelaskan bahwa keluarga merupakan sarana pendidikan pertama bagi anak-anaknya. Keluarga dapat mengajarkan anak-anaknya untuk mengenal sikap jujur, tanggung jawab, serta toleransi dan menghargai perbedaan, sehingga anak-anak dapat tumbuh dengan mendukung demokrasi yang beradab. 

\subsubsection{Keluarga sebagai Tempat Praktik Demokrasi Sehari-hari}

Dalam lingkungan keluarga, anak-anak tidak harus tertuju pada ajaran-ajaran tentang demokrasi yang diberikan dari keluarga, tetapi anak-anak dapat menumbuhkan pemahamannya dengan mempraktikkan dan menerapkan demokrasi yang dimaksud secara langsung. Melalui masa-masa praktik tersebut, anak-anak dapat mengerti dengan konsep demokrasi yang beradab.

\subsubsection{Keluarga sebagai Pembentuk Karakter yang Toleran dan Peduli}

Berkaitan dengan bagian pertama, tidak menutup kemungkinan bahwa keluarga dapat mengajari dan memberikan contoh kepada anak-anaknya tentang sikap toleransi terhadap perbedaan budaya, pandangan hidup, dan agama demi mendukung demokrasi yang beradab.

\subsection{Kendala yang Dihadapi Keluarga dalam Membangun Demokrasi yang Beradab}

\subsubsection{Pengaruh Globalisasi dan Media Digital}

Dalam kasus tertentu, orang tua sengaja memberikan smartphone dan akses internet kepada anak-anak, bahkan kepada anak-anak yang masih kecil dengan alasan agar tidak menangis. Dengan demikian, anak-anak dapat mengakses segala hal secara global, dan tidak menutup kemungkinan bahwa anak-anak menerima ajaran/budaya yang tidak sesuai dengan demokrasi dan keberadaban.

\subsubsection{Kurangnya Pemahaman Demokrasi dalam Keluarga}

Bagian ini sedikit bersinggungan dengan bagian sebelumnya. Tidak semua orang tua mengerti terkait pentingnya menanamkan sikap-sikap yang demokratis. Dari masalah tersebut, orang tua juga tidak dapat mengajarkan nilai demokrasi apa pun kepada anaknya, yang mengakibatkan anak-anak kehilangan arah dalam bersikap demokratis di kesehariannya.

\subsubsection{Perbedaan Sosial dan Budaya dalam Keluarga}

Pada umumnya, keluarga tidak terlepas dari unsur adat dan kebudayaan. Semua keluarga memiliki adat, kebiasaan, tradisi, dan budayanya masing-masing. Namun, perbedaan budaya inilah yang menjadi hambatan karena bisa saja suatu nilai-nilai demokrasi tidak selaras dengan nilai-nilai sosial dan budaya dari keluarga itu sendiri. 

\subsection{Upaya Keluarga dalam Mengatasi Kendala dan Memperkuat Perannya}

\subsubsection{Menanamkan Pendidikan Demokrasi}

Dalam lingkungan keluarga, orang tua dapat mendidik dan mengajarkan anak-anaknya terkait dengan nilai dasar demokrasi, serta mengajari anak-anak untuk menghargai hak dan kewajiban pada mereka sendiri.

\subsubsection{Mengajak Anak untuk Diskusi Bersama Keluarga}

Lanjutan dari bagian sebelumnya, orang tua dapat mengajak anaknya untuk berdiskusi layaknya sebuah rapat. Orang tua dapat melibatkan anak-anaknya untuk melatih proses pengambilan keputusan bersama secara keluarga, serta melatih untuk terbiasa berpikir secara demokratis dan menghargai perbedaan pendapat.

\subsubsection{Menjadi Orang Tua yang Teladan dalam Membentuk Demokrasi}

Orang tua perlu menjadi teladan dan contoh bagi anak-anaknya dalam menerapkan demokrasi, serta sikap-sikap yang mendukung demokrasi tersebut. Anak-anak akan mengikuti tindakan yang biasa dilakukan oleh orang tuanya. Secara tidak langsung, anak-anak mempelajari unsur demokrasi dan sikap yang mengedepankan keberadaban.