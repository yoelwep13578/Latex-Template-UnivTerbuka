\section{Penutup}

%=========================================================%
%                  TULIS PENUTUP DI SINI                  %
% Jangan Lupa untuk Menghapus Contoh Tulisan di Bawah Ini %
%=========================================================%

\subsection{Kesimpulan}

Keluarga memiliki peran yang sangat signifikan dalam membentuk demokrasi yang beradab, terutama melalui pendidikan nilai-nilai dasar seperti kejujuran, tanggung jawab, dan toleransi. Sebagai sarana pendidikan pertama, keluarga mampu menanamkan nilai-nilai tersebut dengan karakter anak-anak yang peduli dan terbuka terhadap perbedaan. Praktik penerapan demokrasi sehari-hari di lingkungan keluarga dapat melatih anak-anak untuk memahami konsep demokrasi secara langsung. 

Namun, terdapat kendala yang menghambat peran keluarga dalam mendukung demokrasi. Pengaruh globalisasi dan media digital sering kali menjadi tantangan karena anak-anak rentan untuk terpapar dengan informasi yang tidak sesuai nilai demokrasi yang sehat. Kurangnya pemahaman orang tua terhadap konsep demokrasi juga dapat membatasi kemampuan mereka untuk menanamkan nilai-nilai tersebut. Perbedaan sosial dan budaya dalam setiap keluarga dapat menjadi hambatan jika tidak dikelola dengan baik. Oleh karena itu, diperlukan upaya melalui pendidikan demokrasi dan menjadi teladan yang baik untuk memperkuat peran keluarga dalam membangun demokrasi yang beradab.

\subsection{Saran}

Berdasarkan pemaparan tersebut, perlu adanya pemahaman mengenai demokrasi dan nilai keberadaban pada setiap keluarga, serta kesadaran untuk menerapkan pendidikan demokrasi sejak dini. Meski demikian, perlu dukungan eksternal dari masyarakat dan seluruh pihak yang dapat mendukung dan membangun demokrasi yang beradab.