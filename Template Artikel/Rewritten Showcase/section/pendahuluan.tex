\section{Pendahuluan}

%=========================================================%
%              TULIS ISI PENDAHULUAN DI SINI              %
% Jangan Lupa untuk Menghapus Contoh Tulisan di Bawah Ini %
%=========================================================%

\subsection{Latar Belakang}

Kehidupan demokrasi yang beradab telah menjadi unsur penting dalam menciptakan masyarakat yang harmonis dan adil. Demokrasi merupakan sekumpulan sistem pemerintahan yang dapat menjadi fondasi bagi kehidupan bersama yang menghargai hak asasi manusia dan supremasi hukum \cite{smkGitaKirtti2019}. Dalam keindonesiaan, penerapan demokrasi yang beradab dapat mendukung peran kontrol sosial dari masyarakat terhadap jalannya pemerintahan, memberikan kebebasan berpendapat, dan memastikan pemerintahan yang transparan dan terbuka \cite{putri2020}.

Namun, tantangan yang dihadapi dalam mewujudkan demokrasi yang beradab masih banyak, seperti rendahnya partisipasi masyarakat dalam proses politik dan adanya ancaman terhadap kebebasan berekspresi. Oleh karena itu, diperlukan upaya untuk meningkatkan kesadaran dan partisipasi masyarakat dalam kehidupan demokrasi, serta memastikan bahwa hak-hak asasi manusia dihormati dan dilindungi. Sarana terdekat dalam mewujudkan demokrasi adalah keluarga.

Keluarga merupakan lingkungan pertama yang menjadi sarana anak-anak untuk belajar tentang nilai-nilai moral, etika, dan prinsip-prinsip demokrasi. Melalui interaksi sehari-hari, orang tua dapat menanamkan sikap toleransi, penghormatan terhadap perbedaan, dan pentingnya partisipasi aktif dalam kehidupan bermasyarakat. Keluarga berperan sebagai pilar utama dalam membentuk karakter anak-anak yang nantinya akan menjadi warga negara yang beradab dan bertanggung jawab dalam masyarakat demokratis \cite{suryawan2018}.

\subsection{Rumusan Masalah}

\begin{enumerate}
    \item Bagaimana peran keluarga dalam membentuk individu yang demokratis?
    \item Apa saja kendala yang dihadapi keluarga dalam membangun demokrasi yang beradab?
    \item Bagaimana upaya untuk mengatasi kendala dan memperkuat perannya?
\end{enumerate}

\subsection{Tujuan}

\begin{enumerate}
    \item Memaparkan peran dan upaya keluarga dalam membentuk individu yang demokratis.
    \item Mendeskripsikan kendala yang dihadapi keluarga dalam membangun demokrasi yang beradab.
    \item Memaparkan upaya untuk mengatasi kendala dan memperkuat perannya.
\end{enumerate}