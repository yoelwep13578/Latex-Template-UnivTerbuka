\section{Kajian Teori} % Dapat diganti menjadi "Kajian Pustaka" bila perlu

%=========================================================%
%                TULIS KAJIAN TEORI DI SINI               %
% Jangan Lupa untuk Menghapus Contoh Tulisan di Bawah Ini %
%=========================================================%

\subsection{Demokrasi dan Konsep Beradab}

\subsubsection{Pengertian Demokrasi}

\paragraph{Menurut Kamus Besar Bahasa Indonesia}

KBBI mengartikan demokrasi sebagai bentuk dan sistem pemerintahan yang mengikutsertakan rakyat untuk memerintah dengan perantaraan wakilnya; pemerintahan rakyat. Dalam bagian kedua, demokrasi diartikan sebagai gagasan atau pandangan hidup yang mengutamakan kesamaan hak, kewajiban, serta kesamaan perlakuan bagi rakyat \cite{kbbi}.

\paragraph{Menurut Abraham Lincoln}

\textit{“Demokrasi adalah pemerintahan dari rakyat, oleh rakyat, dan untuk rakyat.”}

\paragraph{Menurut Sidney Hook}

\textit{“Demokrasi adalah bentuk pemerintahan yang keputusan-keputusan pentingnya, baik secara langsung atau tidak langsung, yang didasarkan pada kesepakatan mayoritas yang diberikan secara bebas oleh rakyat dewasa.”}

\paragraph{Menurut Joseph Schumpeter}

\textit{“Demokrasi adalah prosedur kelembagaan untuk mencapai keputusan politik yang di dalamnya para individu-individu memperoleh kekuasaan untuk membuat keputusan melalui perjuangan kompetitif dalam rangka memperoleh suara rakyat.”}

\paragraph{Menurut C. F. Strong}

\textit{“Demokrasi adalah suatu sistem pemerintahan di mana mayoritas anggota dewasa dari masyarakat politik ikut serta atas dasar sistem perwakilan yang menjamin bahwa pemerintah akhirnya mempertanggungjawabkan tindakan-tindakan kepada mayoritas itu.”}

\subsubsection{Demokrasi yang Beradab}

Demokrasi yang beradab tidak terlepas dari unsur etika, moral, toleransi, dan penghormatan terhadap perbedaan di sekitar. Unsur etika dan moral memiliki peran besar dalam mendukung keberhasilan dan keberlanjutan demokrasi. Masyarakat dan pemerintah perlu bertindak dengan etika dan moral yang tinggi, dengan tujuan menjaga kepercayaan dan dukungan masyarakat, serta memelihara integritas demokrasi yang ada \cite{yusuf2024}.

Kemudian unsur toleransi dan penghormatan terhadap perbedaan berperan sebagai penyatu dalam kebersamaan demokrasi di Indonesia. Unsur tersebut dapat dituangkan melalui sikap saling menghargai dan toleransi terhadap keberagaman agama, yang memuat nilai-nilai dasar demokrasi. Dalam masyarakat demokratis, setiap orang berhak menyuarakan pendapatnya dan harus dihargai meskipun berbeda-beda. Pada akhirnya, toleransi dan penghormatan terhadap keragaman pendapat mampu membentuk harmoni di masyarakat \cite{widiawati2020}.


\subsection{Peran Keluarga dalam Pendidikan}

\subsubsection{Keluarga sebagai Agen Sosial Pertama}

Pendidikan keluarga merupakan upaya yang dilakukan oleh orang tua dalam bentuk pembiasaan dan improvisasi untuk membantu perkembangan pribadi anak. Keluarga adalah pihak pendidik pertama yang menjadi teladan bagi anak ketika dilahirkan ke dunia. Dalam keluarga, anak pertama kali mendapatkan pendidikan dan bimbingan, sehingga keluarga menjadi lingkungan pendidikan yang paling banyak diterima oleh anak.

Dalam jurnal “Pendidikan Keluarga sebagai Pendidikan Pertama bagi Anak”, disebutkan bahwa keluarga sebagai pendidikan pertama memiliki lima fungsi utama, yaitu fungsi afektif, fungsi sosialisasi dan penempatan sosial, fungsi reproduksi, fungsi ekonomi, serta fungsi perawatan dan pemeliharaan kesehatan, yang nantinya dapat mendukung proses pembentukan karakter, pengembangan keterampilan sosial, hingga pemenuhan kebutuhan dasar \cite{besari2022}.

\subsubsection{Pendidikan Karakter dalam Demokrasi}

Demokrasi yang dilaksanakan akan efektif apabila warga negaranya memiliki karakter yang baik melalui unsur etika, moral, toleransi, dan penghormatan terhadap perbedaan. Berdasarkan jurnal "Pendidikan Kewarganegaraan dan Pembentukan Karakter Demokratis Warga Negara" menyebutkan bahwa pendidikan kewarganegaraan dapat menjadi sarana pembentuk karakter demokratis warga negara melalui pengembangan komponen karakter demokratis, yaitu pengetahuan kewargaan dan pemerintahan demokrasi, kecakapan intelektual dari kewargaan demokratis, kecakapan partisipasi dari kewargaan demokratis, dan keutamaan karakter kewargaan demokratis \cite{arif2014}.

Dengan demikian, pendidikan karakter yang diterapkan dalam konteks demokrasi tidak hanya membentuk individu yang beretika dan bermoral, tetapi juga menciptakan warga negara yang mampu berpartisipasi secara aktif dalam kehidupan demokratis, menghargai perbedaan, dan berkontribusi positif terhadap masyarakat.