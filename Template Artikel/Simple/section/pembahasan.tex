\section{Pembahasan}

%=========================================================%
%               TULIS ISI PEMBAHASAN DI SINI              %
% Jangan Lupa untuk Menghapus Contoh Tulisan di Bawah Ini %
%=========================================================%

Ini adalah inti dari artikel Anda, tempat Anda menyajikan dan menganalisis data atau argumen Anda.

\textbf{Penyajian Data:} Sajikan data atau hasil riset Anda secara sistematis. Gunakan tabel, grafik, atau diagram untuk memvisualisasikan data agar lebih mudah dipahami.

\textbf{Analisis Temuan:} Analisis data yang telah disajikan. Hubungkan temuan Anda dengan teori-teori yang telah Anda paparkan di Kajian Pustaka. Jelaskan mengapa data tersebut muncul, apa artinya, dan apa implikasinya terhadap topik yang Anda teliti.

\textbf{Diskusi:} Bandingkan temuan Anda dengan hasil dari penelitian terdahulu. Apakah temuan Anda mendukung atau membantah penelitian sebelumnya? Diskusikan juga keterbatasan dari penelitian Anda dan kemungkinan faktor-faktor lain yang mungkin memengaruhi hasil.