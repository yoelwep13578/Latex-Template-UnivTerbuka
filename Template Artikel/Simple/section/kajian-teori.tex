\section{Kajian Teori} % Dapat diganti menjadi "Kajian Pustaka" bila perlu

%=========================================================%
%                TULIS KAJIAN TEORI DI SINI               %
% Jangan Lupa untuk Menghapus Contoh Tulisan di Bawah Ini %
%=========================================================%

Bagian ini berfungsi sebagai fondasi teoritis artikel Anda. Di sini, Anda menunjukkan bahwa Anda telah melakukan riset mendalam dan memahami konteks topik yang Anda bahas.

\textbf{Landasan Teori:} Jelaskan teori atau konsep-konsep utama yang relevan dengan topik Anda. Misalnya, jika Anda menulis tentang pemasaran digital, jelaskan apa itu SEO, content marketing, dan social media engagement menurut para ahli. Gunakan definisi-definisi dari sumber-sumber terpercaya (buku teks, jurnal ilmiah).

Jika artikel Anda tidak memerlukan bagian Kajian Teori, Anda dapat menghilangkannya dengan menghapus/memberi \textit{comment} bagian \verb|\section{Kajian Teori} % Dapat diganti menjadi "Kajian Pustaka" bila perlu

%=========================================================%
%                TULIS KAJIAN TEORI DI SINI               %
% Jangan Lupa untuk Menghapus Contoh Tulisan di Bawah Ini %
%=========================================================%

Bagian ini berfungsi sebagai fondasi teoritis artikel Anda. Di sini, Anda menunjukkan bahwa Anda telah melakukan riset mendalam dan memahami konteks topik yang Anda bahas.

\textbf{Landasan Teori:} Jelaskan teori atau konsep-konsep utama yang relevan dengan topik Anda. Misalnya, jika Anda menulis tentang pemasaran digital, jelaskan apa itu SEO, content marketing, dan social media engagement menurut para ahli. Gunakan definisi-definisi dari sumber-sumber terpercaya (buku teks, jurnal ilmiah).

Jika artikel Anda tidak memerlukan bagian Kajian Teori, Anda dapat menghilangkannya dengan menghapus/memberi \textit{comment} bagian \verb|\section{Kajian Teori} % Dapat diganti menjadi "Kajian Pustaka" bila perlu

%=========================================================%
%                TULIS KAJIAN TEORI DI SINI               %
% Jangan Lupa untuk Menghapus Contoh Tulisan di Bawah Ini %
%=========================================================%

Bagian ini berfungsi sebagai fondasi teoritis artikel Anda. Di sini, Anda menunjukkan bahwa Anda telah melakukan riset mendalam dan memahami konteks topik yang Anda bahas.

\textbf{Landasan Teori:} Jelaskan teori atau konsep-konsep utama yang relevan dengan topik Anda. Misalnya, jika Anda menulis tentang pemasaran digital, jelaskan apa itu SEO, content marketing, dan social media engagement menurut para ahli. Gunakan definisi-definisi dari sumber-sumber terpercaya (buku teks, jurnal ilmiah).

Jika artikel Anda tidak memerlukan bagian Kajian Teori, Anda dapat menghilangkannya dengan menghapus/memberi \textit{comment} bagian \verb|\section{Kajian Teori} % Dapat diganti menjadi "Kajian Pustaka" bila perlu

%=========================================================%
%                TULIS KAJIAN TEORI DI SINI               %
% Jangan Lupa untuk Menghapus Contoh Tulisan di Bawah Ini %
%=========================================================%

Bagian ini berfungsi sebagai fondasi teoritis artikel Anda. Di sini, Anda menunjukkan bahwa Anda telah melakukan riset mendalam dan memahami konteks topik yang Anda bahas.

\textbf{Landasan Teori:} Jelaskan teori atau konsep-konsep utama yang relevan dengan topik Anda. Misalnya, jika Anda menulis tentang pemasaran digital, jelaskan apa itu SEO, content marketing, dan social media engagement menurut para ahli. Gunakan definisi-definisi dari sumber-sumber terpercaya (buku teks, jurnal ilmiah).

Jika artikel Anda tidak memerlukan bagian Kajian Teori, Anda dapat menghilangkannya dengan menghapus/memberi \textit{comment} bagian \verb|\input{section/kajian-teori.tex}| dalam \textit{file} \texttt{main.tex}.| dalam \textit{file} \texttt{main.tex}.| dalam \textit{file} \texttt{main.tex}.| dalam \textit{file} \texttt{main.tex}.