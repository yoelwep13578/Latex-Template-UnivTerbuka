\section{Pendahuluan}

%=========================================================%
%              TULIS ISI PENDAHULUAN DI SINI              %
% Jangan Lupa untuk Menghapus Contoh Tulisan di Bawah Ini %
%=========================================================%

Sistematika artikel ilmiah ini terdiri dari Judul, Abstrak, Sub-judul (Pendahuluan, Metode, Hasil dan Pembahasan, Kesimpulan, dan Daftar Pustaka). Secara umum, pada bagian Pendahuluan dipaparkan latar belakang, perumusan masalah, dan tujuan penelitian. Jumlah halaman sebanyak 15--20 halaman, termasuk gambar, grafik, atau tabel (jika ada). Artikel memiliki nomor halaman di bagian kanan bawah dengan \textit{font} berukuran 10pt. Dalam setiap halaman artikel tidak ada teks dalam \textit{header} maupun \textit{footer}.