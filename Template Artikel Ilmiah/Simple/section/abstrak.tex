\begin{abstract}
    \itshape\noindent\normalsize
    % Tulis Abstrak Bahasa Indonesia di Sini
    Abstrak merupakan rangkuman dari isi tulisan dalam format yang sangat singkat, terdiri dari satu paragraf dengan jumlah kata 150--200 kata. Abstrak memberikan informasi yang jelas tentang isi artikel dengan menunjukkan hasil utama yang diperoleh beserta kesimpulannya. Abstrak bukan ringkasan dari isi artikel sehingga tidak memuat tabel, gambar, dan daftar pustaka.
    
    \vspace{\baselineskip}\normalfont\noindent\footnotesize
    % Tulis Kata Kuncinya di Sini
    \textbf{Kata Kunci}: Tiga sampai lima kata kunci yang ditulis dengan huruf kecil dan berurut sesuai abjad
\end{abstract}

%=================================%
%      ABSTRAK BAHASA INGGRIS     %
% Dapat digunakan jika diperlukan %
%=================================%

%\begin{otherlanguage}{english}
%\begin{abstract}
%    \itshape\noindent\normalsize
%    % Tulis Abstrak Bahasa Inggris di Sini
%    An abstract is a summary of the content of a paper in a very concise format, consisting of one paragraph with 150-200 words. The abstract provides clear information about the content of the article by presenting the main results obtained and their conclusions. The abstract is not a summary of the content of the article, so it does not include tables, figures, or references.
%    
%    \vspace{\baselineskip}\normalfont\itshape\noindent\footnotesize
%    % Tulis Kata Kuncinya di Sini
%    \textbf{Keywords}: Three to five keywords written in lowercase letters and listed in alphabetical order
%\end{abstract}
%\end{otherlanguage}