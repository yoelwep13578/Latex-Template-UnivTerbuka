% !TEX program = lualatex

% Template Artikel Ilmiah
% Versi 1.3.0

%==========%
% PREAMBLE %
%==========%

\documentclass[a4paper, 12pt]{article}

% Margin & Geometry
\usepackage[
    left=2.5cm,
    right=2.5cm,
    top=2.5cm,
    bottom=2.5cm
]{geometry}

% Linking & Reffering
\usepackage[hidelinks]{hyperref}
\AtBeginDocument{
    \urlstyle{APACrm} % Apacite Roman
}

% Daftar Pustaka & APA 6
\usepackage{apacite}
\bibliographystyle{apacite}

% Bahasa APA 6
%\renewcommand{\onemaskedcitationmsg}[1]{%
    \emph{(#1\ sitasi dihilangkan untuk tinjauan anonim)}}%
\renewcommand{\maskedcitationsmsg}[1]{%
    \emph{(#1\ sitasi dihilangkan untuk tinjauan anonim)}}%
\renewcommand{\refname}{Daftar Pustaka}% Nama daftar pustaka jika berupa bagian.
\renewcommand{\bibname}{Daftar Pustaka}% Nama daftar pustaka jika berupa bab.
\renewcommand{\authorindexname}{Indeks Penulis}% Nama Indeks Penulis
%%
%% A note before the references if a meta-analysis is reported.
\renewcommand{\APACmetaprenote}{%
    Referensi yang ditandai dengan tanda bintang menunjukkan studi
    yang dimasukkan dalam meta-analisis.}%
%%
%% Commands for specific types of @misc entries.
\renewcommand{\bibmessage}{Pesan}%
\renewcommand{\bibcomputerprogram}{Program komputer}%
\renewcommand{\bibcomputerprogrammanual}{Manual program komputer}%
\renewcommand{\bibcomputerprogramandmanual}{Program dan manual komputer}%
\renewcommand{\bibcomputersoftware}{Perangkat lunak}%
\renewcommand{\bibcomputersoftwaremanual}{Manual perangkat lunak}%
\renewcommand{\bibcomputersoftwareandmanual}{Perangkat lunak dan manual}%
\renewcommand{\bibprogramminglanguage}{Bahasa pemrograman}%
%%
%% Other labels
\renewcommand{\bibnodate}{t.t.\hbox{}}%   % ``tanpa tanggal''
\renewcommand{\BIP}{dalam proses terbit}%            % ``in press''
\renewcommand{\BOthers}[1]{dkk.\hbox{}}% ``dan lain-lain''
\renewcommand{\BOthersPeriod}[1]{dkk.\hbox{}}% ``dan lain-lain'' dengan titik
\renewcommand{\BIn}{Dalam}%                  % for ``In '' editor...
\renewcommand{\Bby}{oleh}%                  % for ``by '' editor... (in reprints)
\renewcommand{\BED}{Penyunt.}%          % penyunting
\renewcommand{\BEDS}{Penyunt.}%        % para penyunting
\renewcommand{\BTRANS}{Penerj.}%    % penerjemah
\renewcommand{\BTRANSS}{Penerj.}%   % para penerjemah
\renewcommand{\BTRANSL}{terj.}%   % terjemahan, for the year field
\renewcommand{\BCHAIR}{Ketua}%            % ketua simposium
\renewcommand{\BCHAIRS}{Para Ketua}%          % para ketua
\renewcommand{\BVOL}{Jilid}%        % jilid
\renewcommand{\BVOLS}{Jilid}%        % jilid
\renewcommand{\BNUM}{No.\hbox{}}%         % nomor
\renewcommand{\BNUMS}{No.\hbox{}}%       % nomor
\renewcommand{\BEd}{ed.\hbox{}}%          % edisi
\renewcommand{\BCHAP}{bab}%        % bab
\renewcommand{\BCHAPS}{bab}%       % bab
\renewcommand{\BPG}{h.\hbox{}}%           % halaman
\renewcommand{\BPGS}{hlm.\hbox{}}%         % halaman
%% Default technical report type name.
\renewcommand{\BTR}{Lap. Tek.}
%% Default PhD thesis type name.
\renewcommand{\BPhD}{Disertasi doktor}
%% Default unpublished PhD thesis type name.
\renewcommand{\BUPhD}{Disertasi doktor tidak diterbitkan}
%% Default master's thesis type name.
\renewcommand{\BMTh}{Tesis magister}
%% Default unpublished master's thesis type name.
\renewcommand{\BUMTh}{Tesis magister tidak diterbitkan}
%%
\renewcommand{\BAuthor}{Penulis}% ``Penulis'' jika penerbit = penulis
\renewcommand{\BOWP}{Karya asli diterbitkan}%
\renewcommand{\BREPR}{Dicetak ulang dari}%
\renewcommand{\BAvailFrom}{Tersedia dari\ }%       Situs web; perhatikan spasi.
%% The argument is the date on which it was last checked.
\renewcommand{\BRetrieved}[1]{Diakses {#1}, dari\ }% Situs web; perhatikan spasi.
\renewcommand{\BRetrievedFrom}{Diakses dari\ }% Situs web; perhatikan spasi.
\renewcommand{\BMsgPostedTo}{Pesan dikirim ke\ }%     Pesan; perhatikan spasi.
%%
%% Punctuation
\renewcommand{\BBOP}{(}%   opening parenthesis
\renewcommand{\BBCP}{)}%   closing parenthesis
\renewcommand{\BBOQ}{}%    opening quote for article title
\renewcommand{\BBCQ}{}%    closing quote for article title
\renewcommand{\BBAA}{\&}%  between authors in parenthetical cites and ref. list
\renewcommand{\BBAB}{dan}% between authors in in-text citation
\renewcommand{\BAnd}{\&}%  for ``Ed. & Trans.'' in ref. list
\DeclareRobustCommand{\BPBI}{.~}% Period between initials
\DeclareRobustCommand{\BHBI}{.-}% Hyphen between initials
\renewcommand{\BAP}{ }%   after prefix, before first citation
\renewcommand{\BBAY}{, }% between author(s) and year
\renewcommand{\BBYY}{, }% between years of multiple citations with same author
\renewcommand{\BBC}{; }%  between cites
\renewcommand{\BBN}{, }%  before note
\renewcommand{\BCBT}{,}%  comma between authors in ref. list when no. of
%%  authors = 2
\renewcommand{\BCBL}{,}%  comma before last author when no. of authors > 2
%% Remove excess space with the natbibapa option
\if@APAC@natbib@apa
\renewcommand{\BBAY}{,}%
\renewcommand{\BBYY}{,}%
\renewcommand{\BBC}{;}%
\fi
%%
%% Date formatting
\renewcommand{\APACmonth}[1]{\ifcase #1\or Januari\or Februari\or Maret\or
    April\or Mei\or Juni\or Juli\or Agustus\or September\or Oktober\or
    November\or Desember\or Musim Dingin\or Musim Semi\or Musim Panas\or Musim Gugur\else
    {#1}\fi}%

\renewcommand{\onemaskedcitationmsg}[1]{%
    \emph{(#1\ sitasi dihilangkan untuk tinjauan anonim)}}%
\renewcommand{\maskedcitationsmsg}[1]{%
    \emph{(#1\ sitasi dihilangkan untuk tinjauan anonim)}}%
\renewcommand{\authorindexname}{Indeks Penulis}% Nama Indeks Penulis
%%
%% A note before the references if a meta-analysis is reported.
\renewcommand{\APACmetaprenote}{%
    Referensi yang ditandai dengan tanda bintang menunjukkan studi
    yang dimasukkan dalam meta-analisis.}%
%%
%% Commands for specific types of @misc entries.
\renewcommand{\bibmessage}{Pesan}%
\renewcommand{\bibcomputerprogram}{Program komputer}%
\renewcommand{\bibcomputerprogrammanual}{Manual program komputer}%
\renewcommand{\bibcomputerprogramandmanual}{Program dan manual komputer}%
\renewcommand{\bibcomputersoftware}{Perangkat lunak}%
\renewcommand{\bibcomputersoftwaremanual}{Manual perangkat lunak}%
\renewcommand{\bibcomputersoftwareandmanual}{Perangkat lunak dan manual}%
\renewcommand{\bibprogramminglanguage}{Bahasa pemrograman}%
%%
%% Other labels
\renewcommand{\bibnodate}{t.t.\hbox{}}%   % ``tanpa tanggal''
\renewcommand{\BIP}{dalam proses terbit}%            % ``in press''
\renewcommand{\BOthers}[1]{et al.\hbox{}}% ``dan lain-lain''
\renewcommand{\BOthersPeriod}[1]{et al.\hbox{}}% ``dan lain-lain'' dengan titik
\renewcommand{\BIn}{Dalam}%                  % for ``In '' editor...
\renewcommand{\Bby}{oleh}%                  % for ``by '' editor... (in reprints)
\renewcommand{\BED}{Penyunt.}%          % penyunting
\renewcommand{\BEDS}{Penyunt.}%        % para penyunting
\renewcommand{\BTRANS}{Penerj.}%    % penerjemah
\renewcommand{\BTRANSS}{Penerj.}%   % para penerjemah
\renewcommand{\BTRANSL}{terj.}%   % terjemahan, for the year field
\renewcommand{\BCHAIR}{Ketua}%            % ketua simposium
\renewcommand{\BCHAIRS}{Para Ketua}%          % para ketua
\renewcommand{\BVOL}{Jilid}%        % jilid
\renewcommand{\BVOLS}{Jilid}%        % jilid
\renewcommand{\BNUM}{No.\hbox{}}%         % nomor
\renewcommand{\BNUMS}{No.\hbox{}}%       % nomor
\renewcommand{\BEd}{ed.\hbox{}}%          % edisi
\renewcommand{\BCHAP}{bab}%        % bab
\renewcommand{\BCHAPS}{bab}%       % bab
\renewcommand{\BPG}{h.\hbox{}}%           % halaman
\renewcommand{\BPGS}{hlm.\hbox{}}%         % halaman
%% Default technical report type name.
\renewcommand{\BTR}{Lap. Tek.}
%% Default PhD thesis type name.
\renewcommand{\BPhD}{Disertasi doktor}
%% Default unpublished PhD thesis type name.
\renewcommand{\BUPhD}{Disertasi doktor tidak diterbitkan}
%% Default master's thesis type name.
\renewcommand{\BMTh}{Tesis magister}
%% Default unpublished master's thesis type name.
\renewcommand{\BUMTh}{Tesis magister tidak diterbitkan}
%%
\renewcommand{\BAuthor}{Penulis}% ``Penulis'' jika penerbit = penulis
\renewcommand{\BOWP}{Karya asli diterbitkan}%
\renewcommand{\BREPR}{Dicetak ulang dari}%
\renewcommand{\BAvailFrom}{Tersedia dari\ }%       Situs web; perhatikan spasi.
%% The argument is the date on which it was last checked.
\renewcommand{\BRetrieved}[1]{Diakses {#1}, dari\ }% Situs web; perhatikan spasi.
\renewcommand{\BRetrievedFrom}{Diakses dari\ }% Situs web; perhatikan spasi.
\renewcommand{\BMsgPostedTo}{Pesan dikirim ke\ }%     Pesan; perhatikan spasi.

\renewcommand{\BBAB}{dan}% between authors in in-text citation

% Bahasa & Istilah
\usepackage[indonesian]{babel}

% Grafis
\usepackage{graphicx}
\usepackage{xcolor}

% Matematika / Math Mode
\usepackage{amsmath,amsthm,amssymb}
%\usepackage{MnSymbol}
\usepackage{xfrac}
\usepackage{unicode-math}

\allowdisplaybreaks

\newtheorem{theorem}{Teorema}
\newtheorem{lemma}[theorem]{Lema}
\newtheorem{definition}[theorem]{Definisi}
\newtheorem{example}[theorem]{Contoh}

% Float
\usepackage{float}

% PDF Landscape
\usepackage{pdflscape}

% Tabel
\usepackage{tabularray}

\SetTblrStyle{caption}{font=\vspace{-\baselineskip}\singlespacing\small}
\SetTblrStyle{capcont}{font=\vspace{-\baselineskip}\singlespacing\small}
\SetTblrStyle{note}{font=\small}
\SetTblrStyle{remark}{font=\small}

%========================%
% FONT & FONT SELECTIONS %
%========================%

\usepackage[T1]{fontenc}
\usepackage{fontspec}

% Serif/Main Font --------------------------- %
%\setmainfont{Latin Modern Roman}             % LM (ada di TeXLive)
\setmainfont{Times New Roman}[Ligatures=Rare] % TNR + Ligature
%\setmainfont{TeX Gyre Termes}                % Alternatif Mirip TNR (ada di TeXLive)
%\setmainfont{XITS}                           % Alternatif Mirip TNR (ada di TeXLive)

% Sans-Serif ----------------------------------------- %
\setsansfont{Latin Modern Sans}[Scale=MatchLowercase]  % LM (ada di TeXLive)
%\setsansfont{Alegreya Sans}[Scale=MatchLowercase]     % Alegreya (ada di TeXLive)
%\setsansfont{Noto Sans}[Scale=MatchLowercase]         % Noto
%\setsansfont{Calibri}[Scale=MatchUppercase]           % Calibri

% Monospace ----------------------------------------------- % 
\setmonofont{Latin Modern Mono Light}[Scale=MatchLowercase] % LM (ada di TeXLive)
%\setmonofont{Courier New}[Scale=MatchLowercase] % Courier New (Windows < 10)
%\setmonofont{Cascadia Code Light}[              % Cascadia Code (Windows >= 10)
%    BoldFont={Cascadia Code Bold},
%    ItalicFont={Cascadia Code Light Italic},
%    BoldItalicFont={Cascadia Code Bold Italic},
%    Scale=MatchLowercase
%]
%\setmonofont{Fira Code Light}[                  % Fira Code
%    BoldFont={Fira Code Medium},
%    Contextuals=Alternate,
%    Scale=MatchLowercase
%]
%\setmonofont{JetBrains Mono Thin}[              % JetBrains Mono
%    Contextuals={Alternate},
%    BoldFont={JetBrains Mono Bold},
%    BoldItalicFont={JetBrains Mono Bold Italic},
%    ItalicFont={JetBrains Mono Thin Italic},
%    Scale=MatchLowercase
%]

% Font Matematika -------------- %
%\setmathfont{Latin Modern Math} % Samaan dgn Latin Modern
\setmathfont{XITS Math}          % Samaan dgn Times/TNR

%============%
% CODE BLOCK %
%============%

\usepackage{listings}
%\usepackage{lstfiracode}

\lstdefinelanguage{JavaScript}{ % Setting utk JavaScript karena tdk support
    keywords={typeof, new, true, false, catch, function, return, null, catch, switch, var, if, in, while, do, else, case, break},
    ndkeywords={class, export, boolean, throw, implements, import, this},
    sensitive=false,
    comment=[l]{//},
    morecomment=[s]{/*}{*/},
    morestring=[b]',
    morestring=[b]"
}

\lstset{
    %style=FiraCodeStyle,
    basicstyle=\ttfamily,
    numberstyle=\footnotesize\color{gray},
    numberbychapter=false,
    captionpos=t,
    breaklines=true,
    frame={top|bottom},
    showstringspaces=false,
    commentstyle=\itshape\color{monokai-comment},
    keywordstyle=\bfseries\color{monokai-purple},
    keywordstyle={[2]{\itshape\bfseries\color{monokai-purple}}},
    keywordstyle={[3]{\itshape\bfseries\color{monokai-blue}}},
    ndkeywordstyle=\itshape\bfseries\color{monokai-orange},
    stringstyle=\color{monokai-green},
    identifierstyle=\color{black},
}

%============================%
% FORMATTING TEKS & PARAGRAF %
%============================%

% Parskip Paragraf
\usepackage[indent=1cm]{parskip}
%\setlength{\parskip}{.5\baselineskip}

% Line Spacing
\usepackage{setspace}

% Keterangan
\usepackage[font=small]{caption}

% Color Box
%\usepackage{realboxes}

%============================%
% FORMATTING TEKS & PARAGRAF %
%============================%

\usepackage{titlesec}

% PILIH SATU ------------------------------------
% Numbering Alphanumeric
%\setcounter{secnumdepth}{5}

\renewcommand{\thesection}{\Roman{section}.}
\renewcommand{\thesubsection}{\Alph{subsection}.}
\renewcommand{\thesubsubsection}{\arabic{subsubsection}.}
\renewcommand{\theparagraph}{\alph{paragraph}.}
\renewcommand{\thesubparagraph}{\arabic{subparagraph})}

%\titleformat{\chapter}[block]
%{\normalfont\bfseries\centering\LARGE}
%{}
%{0em}
%{}

%\titleformat{\chapter}[display]
%{\normalfont\bfseries\centering\Large}
%{\MakeUppercase{Bab \thechapter}}
%{-.5em}
%{\MakeUppercase}

\titleformat{\section}[block]
{\normalfont\bfseries\raggedright\Large}
{\hspace{\titleindent}\llap{\parbox[b]{\titleindent}{\normalfont\bfseries\thesection\hfill}}}
{0em}
{\MakeUppercase}

\titleformat{\subsection}[block]
{\normalfont\bfseries\raggedright\Large}
{\hspace{\titleindent}\llap{\parbox[b]{\titleindent}{\normalfont\bfseries\thesubsection\hfill}}}
{0em}
{}

\titleformat{\subsubsection}[block]
{\normalfont\bfseries\itshape\raggedright\large}
{\hspace{\titleindent}\llap{\parbox[b]{\titleindent}{\normalfont\bfseries\thesubsubsection\hfill}}}
{0em}
{}

\titleformat{\paragraph}[runin]
{\normalfont\bfseries\normalsize}
{\hspace{\titleindent}\llap{\parbox[b]{\titleindent}{\normalfont\bfseries\theparagraph\hfill}}}
{0em}
{}[.]

\titleformat{\subparagraph}[runin]
{\normalfont\bfseries\itshape\normalsize}
{\hspace{\titleindent}\llap{\parbox[b]{\titleindent}{\normalfont\bfseries\thesubparagraph\hfill}}}
{0em}
{}[.]
% Numbering Multilevel Number
\setcounter{secnumdepth}{5}

\renewcommand{\thesection}{\arabic{section}.}
\renewcommand{\thesubsection}{\arabic{section}.\arabic{subsection}}

\titleformat{\section}[hang]
{\normalfont\bfseries\raggedright\Large}
{\normalfont\bfseries\thesection}
{1em}
{\MakeUppercase}

\titleformat{\subsection}[hang]
{\normalfont\bfseries\raggedright\Large}
{\normalfont\bfseries\thesubsection}
{1em}
{}

\titleformat{\subsubsection}[hang]
{\normalfont\bfseries\itshape\raggedright\large}
{\normalfont\bfseries\thesubsubsection}
{1em}
{}

\titleformat{\paragraph}[runin]
{\normalfont\bfseries\normalsize}
{\normalfont\bfseries\theparagraph}
{1em}
{}[.]

\titleformat{\subparagraph}[runin]
{\normalfont\bfseries\itshape\normalsize}
{\normalfont\bfseries\thesubparagraph}
{1em}
{}[.]
% Numbering Hanya untuk Level Rendah
%\setcounter{secnumdepth}{5}

\renewcommand{\thesection}{\Roman{section}.}
\renewcommand{\thesubsection}{\Alph{subsection}.}
\renewcommand{\thesubsubsection}{\arabic{subsubsection}.}
\renewcommand{\theparagraph}{\alph{paragraph}.}
\renewcommand{\thesubparagraph}{\arabic{subparagraph})}

%\titleformat{\chapter}[block]
%{\normalfont\bfseries\centering\LARGE}
%{}
%{0em}
%{}

%\titleformat{\chapter}[display]
%{\normalfont\bfseries\centering\Large}
%{\MakeUppercase{Bab \thechapter}}
%{-.5em}
%{\MakeUppercase}

\titleformat{\section}[hang]
{\normalfont\bfseries\raggedright}
{}
{0em}
{\MakeUppercase}

\titleformat{\subsection}[hang]
{\normalfont\bfseries\raggedright}
{}
{0em}
{}

\titleformat{\subsubsection}[hang]
{\normalfont\bfseries\itshape\raggedright}
{}
{0em}
{}

\titleformat{\paragraph}[runin]
{\normalfont\bfseries\normalsize}
{\hspace{\titleindent}\llap{\parbox[b]{\titleindent}{\normalfont\bfseries\theparagraph\hfill}}}
{0em}
{}[.]

\titleformat{\subparagraph}[runin]
{\normalfont\bfseries\itshape\normalsize}
{\hspace{\titleindent}\llap{\parbox[b]{\titleindent}{\normalfont\bfseries\thesubparagraph\hfill}}}
{0em}
{}[.]

% Spacing Heading
\titlespacing*{\section}{0pt}{*3.5}{*1}
\titlespacing*{\subsection}{0pt}{*2}{*1}
\titlespacing*{\subsubsection}{0pt}{*2}{*1}
\titlespacing*{\paragraph}{0pt}{*1.5}{.5em}
\titlespacing*{\subparagraph}{0pt}{*1.5}{.5em}

% List
\usepackage{enumitem}

\setlist[enumerate]{leftmargin=\parindent}
\setlist[itemize]{leftmargin=\parindent}

\newlist{enumerateLeft}{enumerate}{5}
\setlist[enumerateLeft]{
    label=\arabic*.,
    align=left, 
    labelsep=0pt, 
    labelwidth=\parindent, 
    leftmargin=\parindent
}
\newlist{itemizeLeft}{itemize}{5}
\setlist[itemizeLeft]{
    label=\textbullet,
    align=left, 
    labelsep=0pt, 
    labelwidth=\parindent, 
    leftmargin=\parindent
}

\newlist{essaylist}{enumerate}{2}
\setlist[essaylist,1]{
    leftmargin=\parindent, 
    labelwidth=\parindent, 
    labelsep=0pt, 
    align=left, 
    label=\arabic*.
    }
\setlist[essaylist,2]{
    leftmargin=0pt, 
    labelwidth=\parindent, 
    labelsep=0pt, 
    align=left, 
    label=\arabic{essaylisti}. \alph*.
    }

%============%
% PAGE STYLE %
%============%

\usepackage{fancyhdr}
\usepackage{lastpage}

\fancypagestyle{fancy}{
    \fancyhead{}
    \fancyfoot{}
    \fancyfoot[R]{\footnotesize\thepage}
    \renewcommand{\headrulewidth}{0pt}
}

% Load Variabel
%======================%
% NORMAL TEXT VARIABLE %
%======================%

\newcommand{\judul}{Judul Karya Tulis Artikel}

% Informasi Mata Kuliah, Sesi, TUgas, dan Tutor
\newcommand{\skemaTutorial}{Tutorial \textit{Online}}
\newcommand{\sesiKe}{3}
\newcommand{\tugasKe}{1}
\newcommand{\tanggalLengkap}{\today}
\newcommand{\tahun}{\the\year}

\newcommand{\namaMataKuliah}{Pengendalian Berbasis Desktop}
\newcommand{\kodeMataKuliah}{STSI9999}
\newcommand{\KodeMataKuliah}{STSI-9999}

\newcommand{\kodeKelas}{256}

\newcommand{\namaTutorPengampu}{Revi Soekatno, S.Pd., M.Pd., M.Sc.}

% Informasi Mahasiswa
\newcommand{\namaMahasiswa}{Yoeru Sandaru}
\newcommand{\nimMahasiswa}{081298765432}
\newcommand{\programStudi}{Sains Data}
\newcommand{\kodeProgramStudi}{/257}
\newcommand{\fakultas}{Sains dan Teknologi}
\newcommand{\kodeFakultas}{/127}
\newcommand{\utDaerah}{UT Medan}
\newcommand{\kodeUTDaerah}{/28}
\newcommand{\perguruanTinggi}{Universitas Terbuka}
\newcommand{\daearhMahasiswa}{Medan}

%==========================%
% ANOTHER COMMAND VARIABLE %
%==========================%

\newcommand{\linesskip}[1]{\vspace{#1\baselineskip}}
\newcommand{\titleindent}{1.24cm}

% Keterangan dan Sumber
\newcommand{\longcaption}[1]{\caption{\begin{tabular}[t]{@{}l@{}}#1\end{tabular}}}
\newcommand{\tablesource}[1]{\vspace{.3\baselineskip}\caption*{Sumber: #1}\vspace{-\baselineskip}}
\newcommand{\tablesourceleft}[2]{
    
    \raggedright\medskip\hspace{#1}\small Sumber: #2}
\newcommand{\figuresource}[1]{\vspace{-.9em}\caption*{Sumber: #1}\vspace{-.3\baselineskip}}
\newcommand{\lstsource}[1]{\begin{center}\vspace{-1.3\baselineskip}\singlespacing\small Sumber: #1 \vspace{.5\baselineskip}\end{center}}

% Notasi Matematika Tegak
\newcommand{\deriv}{\mathrm{d}}
\newcommand{\adj}[1]{\mathrm{adj}#1}
\newcommand{\euler}{\mathrm{e}}
\newcommand{\imaginary}{\mathrm{i}}

\newcommand{\upa}{\mathrm{a}}
\newcommand{\upb}{\mathrm{b}}
\newcommand{\upc}{\mathrm{c}}
\newcommand{\upd}{\mathrm{d}}
\newcommand{\upe}{\mathrm{e}}
\newcommand{\upf}{\mathrm{f}}
\newcommand{\upg}{\mathrm{g}}
\newcommand{\uph}{\mathrm{h}}
\newcommand{\upi}{\mathrm{i}}
\newcommand{\upj}{\mathrm{j}}
\newcommand{\upk}{\mathrm{k}}
\newcommand{\upl}{\mathrm{l}}
\newcommand{\upm}{\mathrm{m}}
\newcommand{\upn}{\mathrm{n}}
\newcommand{\upo}{\mathrm{o}}
\newcommand{\upp}{\mathrm{p}}
\newcommand{\upq}{\mathrm{q}}
\newcommand{\upr}{\mathrm{r}}
\newcommand{\ups}{\mathrm{s}}
\newcommand{\upt}{\mathrm{t}}
\newcommand{\upu}{\mathrm{u}}
\newcommand{\upv}{\mathrm{v}}
\newcommand{\upw}{\mathrm{w}}
\newcommand{\upx}{\mathrm{x}}
\newcommand{\upy}{\mathrm{y}}
\newcommand{\upz}{\mathrm{z}}

\newcommand{\upA}{\mathrm{A}}
\newcommand{\upB}{\mathrm{B}}
\newcommand{\upC}{\mathrm{C}}
\newcommand{\upD}{\mathrm{D}}
\newcommand{\upE}{\mathrm{E}}
\newcommand{\upF}{\mathrm{F}}
\newcommand{\upG}{\mathrm{G}}
\newcommand{\upH}{\mathrm{H}}
\newcommand{\upI}{\mathrm{I}}
\newcommand{\upJ}{\mathrm{J}}
\newcommand{\upK}{\mathrm{K}}
\newcommand{\upL}{\mathrm{L}}
\newcommand{\upM}{\mathrm{M}}
\newcommand{\upN}{\mathrm{N}}
\newcommand{\upO}{\mathrm{O}}
\newcommand{\upP}{\mathrm{P}}
\newcommand{\upQ}{\mathrm{Q}}
\newcommand{\upR}{\mathrm{R}}
\newcommand{\upS}{\mathrm{S}}
\newcommand{\upT}{\mathrm{T}}
\newcommand{\upU}{\mathrm{U}}
\newcommand{\upV}{\mathrm{V}}
\newcommand{\upW}{\mathrm{W}}
\newcommand{\upX}{\mathrm{X}}
\newcommand{\upY}{\mathrm{Y}}
\newcommand{\upZ}{\mathrm{Z}}

%=====================%
% PENGGANTIAN ISTILAH %
%=====================%

\addto\captionsindonesian{\renewcommand{\bibname}{Referensi}} % Bibliography
\addto\captionsindonesian{\renewcommand{\refname}{Referensi}} % Reference

\renewcommand{\figureautorefname}{Gambar}
\renewcommand{\tableautorefname}{Tabel}
\renewcommand{\equationautorefname}{Pernyataan}

\renewcommand{\lstlistingname}{Kode}

\DeclareTblrTemplate{contfoot-text}{normal}{Lanjutan di halaman berikutnya}
\SetTblrTemplate{contfoot-text}{normal}
\DeclareTblrTemplate{conthead-text}{normal}{(Lanjutan)}
\SetTblrTemplate{conthead-text}{normal}

\DeclareTblrTemplate{remark-tag}{normal}{
    \UseTblrFont {remark-tag} \InsertTblrRemarkTag
}
\SetTblrTemplate{remark-tag}{normal}

%=============%
% COLOR NAMES %
%=============%

\definecolor{monokai-bg}{RGB}{250,250,250}     % Light gray background
\definecolor{monokai-comment}{RGB}{117,113,94} % Grayish comments
\definecolor{monokai-purple}{RGB}{137,89,168}  % Purple for keywords
\definecolor{monokai-green}{RGB}{58,110,59}    % Green for strings
\definecolor{monokai-orange}{RGB}{253,151,31}  % Orange for specials
\definecolor{monokai-blue}{RGB}{81,154,186}    % Blue for types/functions


%============+&
% ISI DOKUMEN &
%=============&

\begin{document}
    
    \pagestyle{fancy}
    
    % Judul & Afiliasi
    \begin{center}
        \textbf{\large\judul}
        
        % Penulis+Mahasiswa dan 2 Informasi -------------------------------
        \textbf{\namaMahasiswa\textsuperscript{1*}, \namaDosen\textsuperscript{2}}
        
        \textsuperscript{1}Mahasiswa Program Studi \programStudi, \fakultas, \perguruanTinggi
        %, \daerahMahasiswa % Daerah
        %, \negaraMahasiswa % Negara
        \\
        \textsuperscript{2}Program Studi \programStudiDosen, \fakultasDosen, \perguruanTinggiDosen
        %, \daerahDosen % Daerah
        %, \negaraDosen % Negara
        
        \textsuperscript{*}\textit{Email: \href{mailto:\emailMahasiswa}{\emailMahasiswa}}
        
        % Dosen+Mahasiswa dan 1 Informasi ---------------------------------
%        \textbf{\namaDosen\textsuperscript{1}, \namaMahasiswa\textsuperscript{2*}}
%       
%        \textsuperscript{1, 2}Program Studi \programStudiDosen, \fakultasDosen, \perguruanTinggiDosen
%        %, \daerahDosen % Daerah
%        %, \negaraDosen % Negara
%        
%        \textsuperscript{*}\textit{Email: \href{mailto:\emailMahasiswa}{\emailMahasiswa}}

        % Custom [Bisa Ditulis Sendiri Sesuai Keperluan] ------------
%        \textbf{Penulis1\textsuperscript{1}, Penulis2\textsuperscript{2}, Penulis3\textsuperscript{3}, ...}
%        
%        \textsuperscript{1}Prodi, Fakultas, Perguruan Tinggi, ... \\
%        \textsuperscript{2}Prodi, Fakultas, Perguruan Tinggi, ...
%        
%        \textsuperscript{*}\textit{Email: \emailref{xyz@ecampus.ut.ac.id}}
    \end{center}
    
    % Load Abstrak
    \begin{abstract}
    \itshape\noindent\normalsize
    % Tulis Abstrak Bahasa Indonesia di Sini
    Jaringan Saraf Tiruan (JST) \textit{Backpropagation} merupakan salah satu JST yang menggunakan algoritma pembelajaran terawasi. Tujuan penelitian ini yaitu untuk mengetahui parameter dan mengukur akurasi ketepatan klasifikasi terhadap status mahasiswa Prodi Statistika Univeritas Terbuka. Berdasarkan hasil, simulasi didapatkan 15 parameter yang dapat memengaruhi status mahasiswa, di antaranya yaitu jenis kelamin, usia, pendidikan (SLTA/SMK, Diploma, S1, dan S2), status pernikahan, status pekerjaan (tidak bekerja, karyawan swasta, wiraswasta, dan PNS), tahun registrasi awal, jumlah registrasi, SKS tempuh, dan IPK. Sedangkan untuk akurasi ketepatan klasifikasi digunakan fungsi aktivasi dan \textit{learning rate} yang dapat menghasilkan nilai kuadrat tengah galat (KTG) yang minimum pada data \textit{training}. Hasil simulasi tersebut diterapkan pula pada data \textit{testing} dengan nilai \textit{cut-off point} sebesar 0,3481, maka didapatkan ketepatan akurasi dengan kurva ROC pada data \textit{training} untuk mahasiswa tidak aktif 99,43\% dan aktif 99,14\%, sedangkan pada data \textit{testing} mahasiswa tidak aktif 94,00\% dan aktif 93,94\%. Jadi dari penelitian ini dapat disimpulkan JST \textit{Backpropagation} merupakan salah satu metode yang sangat baik dalam penerapan metode klasifikasi. 
    
    \vspace{\baselineskip}\normalfont\noindent\footnotesize
    % Tulis Kata Kuncinya di Sini
    \textbf{Kata Kunci}: \textit{Backpropagation}, \textit{cut-off point}, fungsi aktivasi, jaringan saraf tiruan, \textit{learning rate}
\end{abstract}

%=================================%
%      ABSTRAK BAHASA INGGRIS     %
% Dapat digunakan jika diperlukan %
%=================================%

\begin{otherlanguage}{english}
\begin{abstract}
    \itshape\noindent\normalsize
    % Tulis Abstrak Bahasa Inggris di Sini
    Backpropagation Artificial Neural Network (ANN) is an ANN that uses a supervised learning algorithm. The purpose of this study is to determine the parameters and measure the accuracy of the classification accuracy of the student status of the Open University Statistics Study Program. Based on the results, the simulation obtained 15 parameters that can affect student status, including gender, age, education (Senior High School, Diploma, Bachelor, and Magister), marital status, employment status (not working, private employees, entrepreneurs, and civil servants), initial registration year, registration number, semester credit system, and GPA). Meanwhile, for the classification accuracy, the activation function and the learning rate are used minimum mean square of error (MST) on training data. The simulation results are also applied to the testing data with a cut-off point value of 0.3481, so the accuracy of the ROC curve is obtained in the training data for not active students is 99.43\% and 99.14\% active, while the testing data for not active students is 94.00\%. and active 93.94\%. So from this research, it can be concluded that ANN Backpropagation is a very good method in applying the classification method. 
    
    \vspace{\baselineskip}\normalfont\itshape\noindent\footnotesize
    % Tulis Kata Kuncinya di Sini
    \textbf{Keywords}: Backpropagation, cut-off point, activation function, artificial neural network, learning rate
\end{abstract}
\end{otherlanguage}
    
    \onehalfspacing % Line spacing 1,5
    
    % Load Bagian
    \section{Pendahuluan}

%=========================================================%
%              TULIS ISI PENDAHULUAN DI SINI              %
% Jangan Lupa untuk Menghapus Contoh Tulisan di Bawah Ini %
%=========================================================%

Dalam rangka memberikan layanan kepada masyarakat yang mudah dan terintegrasi, Pemerintah meluncurkan suatu kebijakan untuk melakukan reformasi birokrasi dalam pelayanan publik, yaitu Mal Pelayanan Publik (MPP). Kebijakan MPP tertuang dalam Peraturan Menteri Pendayagunaan Aparatur Negara dan Reformasi Birokrasi (Permen PANRB) Nomor 23 Tahun 2017. Kegiatan yang diselenggarakan dalam MPP berupa kegiatan penyelenggaraan pelayanan publik atas barang, jasa, dan/atau pelayanan administrasi. Kegiatan ini merupakan perluasan fungsi pelayanan terpadu baik pusat maupun daerah, serta pelayanan Badan Usaha Milik Negara (BUMN)/Badan Usaha Milik Negara (BUMD)/swasta dalam rangka menyediakan pelayanan yang cepat, mudah, terjangkau, aman, dan nyaman. Untuk mengimplementasikan hal tersebut, pada bulan Agustus 2019 Kota Bogor meluncurkan BPP ``Grha Tiyasa'' dengan melibatkan 14 instansi dan 145 layanan. MPP ``Grha Tiyasa'' mendapatkan sambutan yang baik dari masyarakat. Hal ini dapat dilihat dari statistik pengunjung yang terus meningkat, yakni mencapai 53 159 orang dan melayani 52 683 layanan selama periode 20 Agustus 2019--10 April 2020. Indeks Kepuasan Masyarakat (IKM) terhadap layanan MPP ``Grha Tiyasa'' sangat baik (mencapai nilai A).

Salah satu kriteria penilaian dalam pelayanan publik adalah kapabilitas proses. Kapabilitas proses adalah analisis yang digunakan untuk mengetahui apakah proses kerja yang sedang berjalan memenuhi spesifikasi yang telah ditetapkan \cite{yuri2013tqm}. Analisis Kapabilitas Proses (\textit{Process Capability Analysis}) adalah suatu studi teknik menaksir kapabilitas proses. Taksiran kapabilias proses mungkin dalam bentuk distribusi probabilitas yang mempunyai bentuk, tengah (\textit{mean}), dan penyebaran (\textit{standard deviation}) tertentu \cite{pearn1992distributional, montgomery2009, mccormack2000capability, kotz2002process, wu2009overview}. \citeA{wooluru2014process} mengatakan kapabilitas proses adalah suatu metode yang menggabungkan teknik statistika dari kurva normal dan grafik kontrol dengan kriteria penilaian untuk menafsirkan dan menganalisis data yang mewakili suatu proses. \citeA{arcidiacono2017, harry1991} mengemukakan bahwa untuk seluruh siklus produksi suatu produk atau layanan diperlukan suatu teknik analisis yang dapat mengevaluasi secara benar mengenai suara pelanggan dan kinerja proses. Dalam hal ini terdapat keterkaitan antara kapabilitas proses dengan kemampuan proses, kinerja proses, dan sigma proses.

Kapabilitas proses ini merupakan suatu ukuran kinerja kritis yang menunjukkan proses mampu menghasilkan sesuai dengan spesifikasi produk yang diterapkan oleh manajemen berdasarkan kebutuhan dan ekspektasi pelanggan \cite{gasperz2004production}. Kapabilitas proses menunjukkan rentang suatu variasi suatu proses atau suatu besaran yang menunjukkan kemampuan dari suatu layanan publik untuk menghasilkan pelayanan yang sesuai dengan spesifikasi atau standar pelayanan yang ditentukan. Dengan kata lain, kapabilitas proses menunjukkan sampai seberapa jauh suatu proses mampu memenuhi spesifikasi atau standar yang diinginkan. Hasil analisis kapabilitas proses dapat digunakan antara lain untuk memperkirakan seberapa baik proses akan memenuhi toleransi, membantu pengembangan perancangan produk/layanan dalam memilih atau mengubah proses, dan mengurangi variabilitas dalam proses produksi/layanan \cite{harjosoedarmo1996}.

Suatu proses dikatakan kapabel apabila memenuhi tiga asumsi yaitu: karakteristik kualitas berdistribusi normal, proses terkendali, dan rata-rata proses berada diantara batas spesifikasi atas dan batas spesifikasi bawah. Batas spesifikasi disebut juga sebagai batas toleransi. Penentuan batas spesifikasi ini ditentukan berdasarkan kebutuhan konsumen yakni apa yang diharapkan konsumen terhadap produk/layanan yang diinginkan. Pada umumnya, penentuan batas spesifikasi ini ditentukan melalui riset pasar dan dikombinasikan dengan rancangan produk dan jasa (pelayanan). Terdapat dua indikator untuk mengukur kapabilitas proses yaitu indeks kapabilitas proses (C\textsubscript{p}) dan indeks performansi Kane (C\textsubscript{pk})

Pelayanan yang diberikan di suatu tempat layanan publik seperti MPP bisa bervariasi meskipun sudah mengikuti standar pelayanan yang telah diberikan oleh pemerintah setempat. Namun dalam kenyataan di lapangan mungkin terdapat beberapa pelayanan yang tidak memenuhi harapan konsumen. Variasi atau variabilitas layanan adalah ketidakseragaman layanan yang diberikan oleh suatu produk/jasa layanan yang tidak memenuhi spesifikasi. Pada umumnya, konsumen mengharapkan produk/jasa layanan memiliki variabilitas yang minimum. Oleh karena itu, suatu jasa layanan publik harus melakukan upaya peningkatan kualitas dan memastikan bahwa variasi/varabilitas karakteristik kualitas produk/jasa layanan yang mereka berikan pada konsumen masih dalam batas-batas yang dapat ditoleransikan oleh konsumen. Dengan kata lain, variasi layanan yang diberikan kepada konsumen masih berada dalam batas spesifikasi. Untuk menguji apakah variabiltas masih dalam karakteristik proses dan apakah proses mampu menghasilkan produk/jasa layanan yang diberikan oleh MPP ``Grha Tiyasa'' sesuai dengan spesifikasi, maka perlu dilakukan analisis menggunakan analisis kapabilitas proses. Berdasarkan alasan tersebut maka paper ini bertujuan memaparkan hasil analisis kapabilitas proses pada pelayanan publik di MPP ``Grha Tiyasa'' di Kota Bogor.
    \section{Metode} % Dapat diganti menjadi "Kerangka Pikir" jika perlu

%=========================================================%
%           TULIS METODE/KERANGKA PIKIR DI SINI           %
% Jangan Lupa untuk Menghapus Contoh Tulisan di Bawah Ini %
%=========================================================%

Bagian metode berisi tentang formulasi-formulasi dasar yang menjadi landasan untuk pengembangan hasil yang akan dilakukan. Selain itu, pada bagian ini dapat pula memuat tentang deskripsi lokasi penelitian, sumber data, teknik pengumpulan data dan analisis data.
    \section{Hasil dan Pembahasan}

%=========================================================%
%               TULIS ISI PEMBAHASAN DI SINI              %
% Jangan Lupa untuk Menghapus Contoh Tulisan di Bawah Ini %
%=========================================================%

Ini adalah inti dari artikel Anda, tempat Anda menyajikan dan menganalisis data atau argumen Anda. Sajikan data atau hasil riset Anda secara sistematis --- bisa juga menggunakan tabel, grafik, atau diagram untuk memvisualisasikan data agar lebih mudah dipahami.
    \section{Kesimpulan}

%=========================================================%
%                  TULIS PENUTUP DI SINI                  %
% Jangan Lupa untuk Menghapus Contoh Tulisan di Bawah Ini %
%=========================================================%

Simpulan di sini bukanlah ringkasan uraian, melainkan dari hasil temuan penting dalam bahasan sebelumnya. Simpulan perlu bisa menjawab rumusan masalah yang Anda ajukan di Pendahuluan. Jangan lagi menyajikan data baru, bahasan yang tidak ada, atau mengambil jawaban dari data yang kurang lengkap/tidak jelas.

Saran dapat berisi rekomendasi untuk penelitian di masa depan. Misalnya, sarankan topik yang belum Anda teliti atau variabel lain yang bisa ditambahkan untuk memperkaya pembahasan. Berikan rekomendasi praktis yang dapat diterapkan berdasarkan hasil penelitian Anda. Misalnya, jika bahasan Anda tentang efektivitas metode pembelajaran, berikan saran kepada pendidik atau pembuat kebijakan tentang cara meningkatkan metode tersebut.
    
    % Daftar Pustaka
    \nocite{*}
    \bibliography{reference}
    
\end{document}