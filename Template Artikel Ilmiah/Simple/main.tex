% !TEX program = lualatex

% Template Artikel Ilmiah
% Versi 1.3.0

%==========%
% PREAMBLE %
%==========%

\documentclass[a4paper, 12pt]{article}

% Margin & Geometry
\usepackage[
    left=2.5cm,
    right=2.5cm,
    top=2.5cm,
    bottom=2.5cm
]{geometry}

% Linking & Reffering
\usepackage[hidelinks]{hyperref}
\AtBeginDocument{
    \urlstyle{APACrm} % Apacite Roman
}

% Daftar Pustaka & APA 6
\usepackage{apacite}
\bibliographystyle{apacite}

% Bahasa APA 6
%\input{preset/APA-bahasa-indonesia.tex}
\renewcommand{\onemaskedcitationmsg}[1]{%
    \emph{(#1\ sitasi dihilangkan untuk tinjauan anonim)}}%
\renewcommand{\maskedcitationsmsg}[1]{%
    \emph{(#1\ sitasi dihilangkan untuk tinjauan anonim)}}%
\renewcommand{\authorindexname}{Indeks Penulis}% Nama Indeks Penulis
%%
%% A note before the references if a meta-analysis is reported.
\renewcommand{\APACmetaprenote}{%
    Referensi yang ditandai dengan tanda bintang menunjukkan studi
    yang dimasukkan dalam meta-analisis.}%
%%
%% Commands for specific types of @misc entries.
\renewcommand{\bibmessage}{Pesan}%
\renewcommand{\bibcomputerprogram}{Program komputer}%
\renewcommand{\bibcomputerprogrammanual}{Manual program komputer}%
\renewcommand{\bibcomputerprogramandmanual}{Program dan manual komputer}%
\renewcommand{\bibcomputersoftware}{Perangkat lunak}%
\renewcommand{\bibcomputersoftwaremanual}{Manual perangkat lunak}%
\renewcommand{\bibcomputersoftwareandmanual}{Perangkat lunak dan manual}%
\renewcommand{\bibprogramminglanguage}{Bahasa pemrograman}%
%%
%% Other labels
\renewcommand{\bibnodate}{t.t.\hbox{}}%   % ``tanpa tanggal''
\renewcommand{\BIP}{dalam proses terbit}%            % ``in press''
\renewcommand{\BOthers}[1]{et al.\hbox{}}% ``dan lain-lain''
\renewcommand{\BOthersPeriod}[1]{et al.\hbox{}}% ``dan lain-lain'' dengan titik
\renewcommand{\BIn}{Dalam}%                  % for ``In '' editor...
\renewcommand{\Bby}{oleh}%                  % for ``by '' editor... (in reprints)
\renewcommand{\BED}{Penyunt.}%          % penyunting
\renewcommand{\BEDS}{Penyunt.}%        % para penyunting
\renewcommand{\BTRANS}{Penerj.}%    % penerjemah
\renewcommand{\BTRANSS}{Penerj.}%   % para penerjemah
\renewcommand{\BTRANSL}{terj.}%   % terjemahan, for the year field
\renewcommand{\BCHAIR}{Ketua}%            % ketua simposium
\renewcommand{\BCHAIRS}{Para Ketua}%          % para ketua
\renewcommand{\BVOL}{Jilid}%        % jilid
\renewcommand{\BVOLS}{Jilid}%        % jilid
\renewcommand{\BNUM}{No.\hbox{}}%         % nomor
\renewcommand{\BNUMS}{No.\hbox{}}%       % nomor
\renewcommand{\BEd}{ed.\hbox{}}%          % edisi
\renewcommand{\BCHAP}{bab}%        % bab
\renewcommand{\BCHAPS}{bab}%       % bab
\renewcommand{\BPG}{h.\hbox{}}%           % halaman
\renewcommand{\BPGS}{hlm.\hbox{}}%         % halaman
%% Default technical report type name.
\renewcommand{\BTR}{Lap. Tek.}
%% Default PhD thesis type name.
\renewcommand{\BPhD}{Disertasi doktor}
%% Default unpublished PhD thesis type name.
\renewcommand{\BUPhD}{Disertasi doktor tidak diterbitkan}
%% Default master's thesis type name.
\renewcommand{\BMTh}{Tesis magister}
%% Default unpublished master's thesis type name.
\renewcommand{\BUMTh}{Tesis magister tidak diterbitkan}
%%
\renewcommand{\BAuthor}{Penulis}% ``Penulis'' jika penerbit = penulis
\renewcommand{\BOWP}{Karya asli diterbitkan}%
\renewcommand{\BREPR}{Dicetak ulang dari}%
\renewcommand{\BAvailFrom}{Tersedia dari\ }%       Situs web; perhatikan spasi.
%% The argument is the date on which it was last checked.
\renewcommand{\BRetrieved}[1]{Diakses {#1}, dari\ }% Situs web; perhatikan spasi.
\renewcommand{\BRetrievedFrom}{Diakses dari\ }% Situs web; perhatikan spasi.
\renewcommand{\BMsgPostedTo}{Pesan dikirim ke\ }%     Pesan; perhatikan spasi.

\renewcommand{\BBAB}{dan}% between authors in in-text citation

% Bahasa & Istilah
\usepackage[indonesian]{babel}

% Grafis
\usepackage{graphicx}
\usepackage{xcolor}

% Matematika / Math Mode
\usepackage{amsmath,amsthm,amssymb}
%\usepackage{MnSymbol}
\usepackage{xfrac}
\usepackage{unicode-math}

\allowdisplaybreaks

\newtheorem{theorem}{Teorema}
\newtheorem{lemma}[theorem]{Lema}
\newtheorem{definition}[theorem]{Definisi}
\newtheorem{example}[theorem]{Contoh}

% Float
\usepackage{float}

% PDF Landscape
\usepackage{pdflscape}

% Tabel
\usepackage{tabularray}

\SetTblrStyle{caption}{font=\vspace{-\baselineskip}\singlespacing\small}
\SetTblrStyle{capcont}{font=\vspace{-\baselineskip}\singlespacing\small}
\SetTblrStyle{note}{font=\small}
\SetTblrStyle{remark}{font=\small}

%========================%
% FONT & FONT SELECTIONS %
%========================%

\usepackage[T1]{fontenc}
\usepackage{fontspec}

% Serif/Main Font --------------------------- %
%\setmainfont{Latin Modern Roman}             % LM (ada di TeXLive)
\setmainfont{Times New Roman}[Ligatures=Rare] % TNR + Ligature
%\setmainfont{TeX Gyre Termes}                % Alternatif Mirip TNR (ada di TeXLive)
%\setmainfont{XITS}                           % Alternatif Mirip TNR (ada di TeXLive)

% Sans-Serif ----------------------------------------- %
\setsansfont{Latin Modern Sans}[Scale=MatchLowercase]  % LM (ada di TeXLive)
%\setsansfont{Alegreya Sans}[Scale=MatchLowercase]     % Alegreya (ada di TeXLive)
%\setsansfont{Noto Sans}[Scale=MatchLowercase]         % Noto
%\setsansfont{Calibri}[Scale=MatchUppercase]           % Calibri

% Monospace ----------------------------------------------- % 
\setmonofont{Latin Modern Mono Light}[Scale=MatchLowercase] % LM (ada di TeXLive)
%\setmonofont{Courier New}[Scale=MatchLowercase] % Courier New (Windows < 10)
%\setmonofont{Cascadia Code Light}[              % Cascadia Code (Windows >= 10)
%    BoldFont={Cascadia Code Bold},
%    ItalicFont={Cascadia Code Light Italic},
%    BoldItalicFont={Cascadia Code Bold Italic},
%    Scale=MatchLowercase
%]
%\setmonofont{Fira Code Light}[                  % Fira Code
%    BoldFont={Fira Code Medium},
%    Contextuals=Alternate,
%    Scale=MatchLowercase
%]
%\setmonofont{JetBrains Mono Thin}[              % JetBrains Mono
%    Contextuals={Alternate},
%    BoldFont={JetBrains Mono Bold},
%    BoldItalicFont={JetBrains Mono Bold Italic},
%    ItalicFont={JetBrains Mono Thin Italic},
%    Scale=MatchLowercase
%]

% Font Matematika -------------- %
%\setmathfont{Latin Modern Math} % Samaan dgn Latin Modern
\setmathfont{XITS Math}          % Samaan dgn Times/TNR

%============%
% CODE BLOCK %
%============%

\usepackage{listings}
%\usepackage{lstfiracode}

\lstdefinelanguage{JavaScript}{ % Setting utk JavaScript karena tdk support
    keywords={typeof, new, true, false, catch, function, return, null, catch, switch, var, if, in, while, do, else, case, break},
    ndkeywords={class, export, boolean, throw, implements, import, this},
    sensitive=false,
    comment=[l]{//},
    morecomment=[s]{/*}{*/},
    morestring=[b]',
    morestring=[b]"
}

\lstset{
    %style=FiraCodeStyle,
    basicstyle=\ttfamily,
    numberstyle=\footnotesize\color{gray},
    numberbychapter=false,
    captionpos=t,
    breaklines=true,
    frame={top|bottom},
    showstringspaces=false,
    commentstyle=\itshape\color{monokai-comment},
    keywordstyle=\bfseries\color{monokai-purple},
    keywordstyle={[2]{\itshape\bfseries\color{monokai-purple}}},
    keywordstyle={[3]{\itshape\bfseries\color{monokai-blue}}},
    ndkeywordstyle=\itshape\bfseries\color{monokai-orange},
    stringstyle=\color{monokai-green},
    identifierstyle=\color{black},
}

%============================%
% FORMATTING TEKS & PARAGRAF %
%============================%

% Parskip Paragraf
\usepackage[indent=1cm]{parskip}
%\setlength{\parskip}{.5\baselineskip}

% Line Spacing
\usepackage{setspace}

% Keterangan
\usepackage[font=small]{caption}

% Color Box
%\usepackage{realboxes}

%============================%
% FORMATTING TEKS & PARAGRAF %
%============================%

\usepackage{titlesec}

% PILIH SATU ------------------------------------
% Numbering Alphanumeric
%\setcounter{secnumdepth}{5}

\renewcommand{\thesection}{\Roman{section}.}
\renewcommand{\thesubsection}{\Alph{subsection}.}
\renewcommand{\thesubsubsection}{\arabic{subsubsection}.}
\renewcommand{\theparagraph}{\alph{paragraph}.}
\renewcommand{\thesubparagraph}{\arabic{subparagraph})}

%\titleformat{\chapter}[block]
%{\normalfont\bfseries\centering\LARGE}
%{}
%{0em}
%{}

%\titleformat{\chapter}[display]
%{\normalfont\bfseries\centering\Large}
%{\MakeUppercase{Bab \thechapter}}
%{-.5em}
%{\MakeUppercase}

\titleformat{\section}[hang]
{\normalfont\bfseries\raggedright}
{\hspace{\titleindent}\llap{\parbox[b]{\titleindent}{\normalfont\bfseries\thesection\hfill}}}
{0em}
{\MakeUppercase}

\titleformat{\subsection}[hang]
{\normalfont\bfseries\raggedright}
{\hspace{\titleindent}\llap{\parbox[b]{\titleindent}{\normalfont\bfseries\thesubsection\hfill}}}
{0em}
{}

\titleformat{\subsubsection}[hang]
{\normalfont\bfseries\itshape\raggedright}
{\hspace{\titleindent}\llap{\parbox[b]{\titleindent}{\normalfont\bfseries\thesubsubsection\hfill}}}
{0em}
{}

\titleformat{\paragraph}[runin]
{\normalfont\bfseries\normalsize}
{\hspace{\titleindent}\llap{\parbox[b]{\titleindent}{\normalfont\bfseries\theparagraph\hfill}}}
{0em}
{}[.]

\titleformat{\subparagraph}[runin]
{\normalfont\bfseries\itshape\normalsize}
{\hspace{\titleindent}\llap{\parbox[b]{\titleindent}{\normalfont\bfseries\thesubparagraph\hfill}}}
{0em}
{}[.]
% Numbering Multilevel Number
\setcounter{secnumdepth}{5}

\renewcommand{\thesection}{\arabic{section}.}
\renewcommand{\thesubsection}{\arabic{section}.\arabic{subsection}}

\titleformat{\section}[block]
{\normalfont\bfseries\raggedright\Large}
{\normalfont\bfseries\thesection}
{1em}
{\MakeUppercase}

\titleformat{\subsection}[block]
{\normalfont\bfseries\raggedright\Large}
{\normalfont\bfseries\thesubsection}
{1em}
{}

\titleformat{\subsubsection}[block]
{\normalfont\bfseries\itshape\raggedright\large}
{\normalfont\bfseries\thesubsubsection}
{1em}
{}

\titleformat{\paragraph}[runin]
{\normalfont\bfseries\normalsize}
{\normalfont\bfseries\theparagraph}
{1em}
{}[.]

\titleformat{\subparagraph}[runin]
{\normalfont\bfseries\itshape\normalsize}
{\normalfont\bfseries\thesubparagraph}
{1em}
{}[.]
% Numbering Hanya untuk Level Rendah
%\input{preset/lowlevel-numbering-heading.tex}

% Spacing Heading
\titlespacing*{\section}{0pt}{*3.5}{*1}
\titlespacing*{\subsection}{0pt}{*2}{*1}
\titlespacing*{\subsubsection}{0pt}{*2}{*1}
\titlespacing*{\paragraph}{0pt}{*1.5}{.5em}
\titlespacing*{\subparagraph}{0pt}{*1.5}{.5em}

% List
\usepackage{enumitem}

\setlist[enumerate]{leftmargin=\parindent}
\setlist[itemize]{leftmargin=\parindent}

\newlist{enumerateLeft}{enumerate}{5}
\setlist[enumerateLeft]{
    label=\arabic*.,
    align=left, 
    labelsep=0pt, 
    labelwidth=\parindent, 
    leftmargin=\parindent
}
\newlist{itemizeLeft}{itemize}{5}
\setlist[itemizeLeft]{
    label=\textbullet,
    align=left, 
    labelsep=0pt, 
    labelwidth=\parindent, 
    leftmargin=\parindent
}

\newlist{essaylist}{enumerate}{2}
\setlist[essaylist,1]{
    leftmargin=\parindent, 
    labelwidth=\parindent, 
    labelsep=0pt, 
    align=left, 
    label=\arabic*.
    }
\setlist[essaylist,2]{
    leftmargin=0pt, 
    labelwidth=\parindent, 
    labelsep=0pt, 
    align=left, 
    label=\arabic{essaylisti}. \alph*.
    }

%============%
% PAGE STYLE %
%============%

\usepackage{fancyhdr}
\usepackage{lastpage}

\fancypagestyle{fancy}{
    \fancyhead{}
    \fancyfoot{}
    \fancyfoot[R]{\footnotesize\thepage}
    \renewcommand{\headrulewidth}{0pt}
}

% Load Variabel
%======================%
% NORMAL TEXT VARIABLE %
%======================%

\newcommand{\judul}{Metode Klasifikasi Jaringan Saraf Tiruan \textit{Backpropagation} Pada Mahasiswa Statistika Universitas Terbuka}

% Informasi Waktu
\newcommand{\tanggalLengkap}{\today}
\newcommand{\tahun}{\the\year}

% Informasi Mahasiswa
\newcommand{\namaMahasiswa}{Yoeru Sandaru}
\newcommand{\niMahasiswa}{081298765432}
\newcommand{\programStudi}{Matematika}
\newcommand{\fakultas}{Fakultas Sains dan Teknologi}
\newcommand{\utDaerah}{UT Medan}
\newcommand{\perguruanTinggi}{Universitas Terbuka}
\newcommand{\daerahMahasiswa}{Medan}
\newcommand{\negaraMahasiswa}{Indonesia}

\newcommand{\emailMahasiswa}{\niMahasiswa @ecampus.ut.ac.id} % Email E-Campus sesuai NIM

% Informasi Tutor/Dosen Pembimbing
\newcommand{\namaDosen}{Revo Wibowo}
\newcommand{\niDosen}{081298765432}
\newcommand{\programStudiDosen}{Matematika}
\newcommand{\fakultasDosen}{Fakultas Sains dan Teknologi}
\newcommand{\utDaerahDosen}{UT Medan}
\newcommand{\perguruanTinggiDosen}{Universitas Terbuka}
\newcommand{\daerahDosen}{Medan}
\newcommand{\negaraDosen}{Indonesia}

\newcommand{\emailDosen}{bowo@ecampus.ut.ac.id}

%==========================%
% ANOTHER COMMAND VARIABLE %
%==========================%

\newcommand{\linesskip}[1]{\vspace{#1\baselineskip}}
\newcommand{\titleindent}{1cm}

% Keterangan dan Sumber
\newcommand{\longcaption}[1]{\caption{\begin{tabular}[t]{@{}l@{}}#1\end{tabular}}}
\newcommand{\tablesource}[1]{\vspace{.3\baselineskip}\caption*{Sumber: #1}\vspace{-\baselineskip}}
\newcommand{\tablesourceleft}[2]{
    
    \raggedright\medskip\hspace{#1}\small Sumber: #2}
\newcommand{\figuresource}[1]{\vspace{-.9em}\caption*{Sumber: #1}\vspace{-.3\baselineskip}}
\newcommand{\lstsource}[1]{\begin{center}\vspace{-1.3\baselineskip}\singlespacing\small Sumber: #1 \vspace{.5\baselineskip}\end{center}}

% Notasi Matematika Tegak
\newcommand{\deriv}{\mathrm{d}}
\newcommand{\adj}[1]{\mathrm{adj}#1}
\newcommand{\euler}{\mathrm{e}}
\newcommand{\imaginary}{\mathrm{i}}

\newcommand{\upa}{\mathrm{a}}
\newcommand{\upb}{\mathrm{b}}
\newcommand{\upc}{\mathrm{c}}
\newcommand{\upd}{\mathrm{d}}
\newcommand{\upe}{\mathrm{e}}
\newcommand{\upf}{\mathrm{f}}
\newcommand{\upg}{\mathrm{g}}
\newcommand{\uph}{\mathrm{h}}
\newcommand{\upi}{\mathrm{i}}
\newcommand{\upj}{\mathrm{j}}
\newcommand{\upk}{\mathrm{k}}
\newcommand{\upl}{\mathrm{l}}
\newcommand{\upm}{\mathrm{m}}
\newcommand{\upn}{\mathrm{n}}
\newcommand{\upo}{\mathrm{o}}
\newcommand{\upp}{\mathrm{p}}
\newcommand{\upq}{\mathrm{q}}
\newcommand{\upr}{\mathrm{r}}
\newcommand{\ups}{\mathrm{s}}
\newcommand{\upt}{\mathrm{t}}
\newcommand{\upu}{\mathrm{u}}
\newcommand{\upv}{\mathrm{v}}
\newcommand{\upw}{\mathrm{w}}
\newcommand{\upx}{\mathrm{x}}
\newcommand{\upy}{\mathrm{y}}
\newcommand{\upz}{\mathrm{z}}

\newcommand{\upA}{\mathrm{A}}
\newcommand{\upB}{\mathrm{B}}
\newcommand{\upC}{\mathrm{C}}
\newcommand{\upD}{\mathrm{D}}
\newcommand{\upE}{\mathrm{E}}
\newcommand{\upF}{\mathrm{F}}
\newcommand{\upG}{\mathrm{G}}
\newcommand{\upH}{\mathrm{H}}
\newcommand{\upI}{\mathrm{I}}
\newcommand{\upJ}{\mathrm{J}}
\newcommand{\upK}{\mathrm{K}}
\newcommand{\upL}{\mathrm{L}}
\newcommand{\upM}{\mathrm{M}}
\newcommand{\upN}{\mathrm{N}}
\newcommand{\upO}{\mathrm{O}}
\newcommand{\upP}{\mathrm{P}}
\newcommand{\upQ}{\mathrm{Q}}
\newcommand{\upR}{\mathrm{R}}
\newcommand{\upS}{\mathrm{S}}
\newcommand{\upT}{\mathrm{T}}
\newcommand{\upU}{\mathrm{U}}
\newcommand{\upV}{\mathrm{V}}
\newcommand{\upW}{\mathrm{W}}
\newcommand{\upX}{\mathrm{X}}
\newcommand{\upY}{\mathrm{Y}}
\newcommand{\upZ}{\mathrm{Z}}

% Code Snippet Berwarna
\newcommand{\inlinesnippet}[1]{\colorbox{gray!10}{\lstinline‖#1‖}}
\newcommand{\verbsnippet}[1]{\colorbox{gray!10}{\verb‖#1‖}}

% Email AutoLink
\newcommand{\emailref}[1]{\href{mailto:#1}{#1}}

%=====================%
% PENGGANTIAN ISTILAH %
%=====================%

\addto\captionsindonesian{\renewcommand{\abstractname}{Abstrak}} % Abstract
\addto\captionsindonesian{\renewcommand{\bibname}{Daftar Pustaka}} % Bibliography
\addto\captionsindonesian{\renewcommand{\refname}{Daftar Pustaka}} % Reference

\renewcommand{\chapterautorefname}{Bab}
\renewcommand{\sectionautorefname}{Bagian}
\renewcommand{\subsectionautorefname}{Sub-bagian}
\renewcommand{\subsubsectionautorefname}{Sub-sub-bagian}
\renewcommand{\paragraphautorefname}{Paragraf}
\renewcommand{\subparagraphautorefname}{Sub-paragraf}

\renewcommand{\figureautorefname}{Gambar}
\renewcommand{\tableautorefname}{Tabel}
\renewcommand{\equationautorefname}{Persamaan}
\renewcommand{\lstlistingname}{Kode}

\DeclareTblrTemplate{contfoot-text}{normal}{Lanjutan di halaman berikutnya}
\SetTblrTemplate{contfoot-text}{normal}
\DeclareTblrTemplate{conthead-text}{normal}{(Lanjutan)}
\SetTblrTemplate{conthead-text}{normal}

\DeclareTblrTemplate{remark-tag}{normal}{
    \UseTblrFont {remark-tag} \InsertTblrRemarkTag
}
\SetTblrTemplate{remark-tag}{normal}

%====================%
% MODIFIKASI ABSTRAK %
%====================%

\makeatletter
\renewenvironment{abstract}{%
    \if@twocolumn
    \section*{\abstractname}%
    \else %% <- \small dihapus
    \begin{center}%
        {\bfseries \normalsize\abstractname\vspace{\z@}}%  %% <- \normalsize ditambah
    \end{center}%
    \quotation
    \fi}
{\if@twocolumn\else\endquotation\fi}
\makeatother


%=============%
% COLOR NAMES %
%=============%

\definecolor{monokai-bg}{RGB}{250,250,250}     % Light gray background
\definecolor{monokai-comment}{RGB}{117,113,94} % Grayish comments
\definecolor{monokai-purple}{RGB}{137,89,168}  % Purple for keywords
\definecolor{monokai-green}{RGB}{58,110,59}    % Green for strings
\definecolor{monokai-orange}{RGB}{253,151,31}  % Orange for specials
\definecolor{monokai-blue}{RGB}{81,154,186}    % Blue for types/functions


%============+&
% ISI DOKUMEN &
%=============&

\begin{document}
    
    \pagestyle{fancy}
    
    % Judul & Afiliasi
    \begin{center}
        \textbf{\large\judul}
        
        % Penulis+Mahasiswa dan 2 Informasi -------------------------------
        \textbf{\namaMahasiswa\textsuperscript{1*}, \namaDosen\textsuperscript{2}}
        
        \textsuperscript{1}Mahasiswa Program Studi \programStudi, \fakultas, \perguruanTinggi
        %, \daerahMahasiswa % Daerah
        %, \negaraMahasiswa % Negara
        \\
        \textsuperscript{2}Program Studi \programStudiDosen, \fakultasDosen, \perguruanTinggiDosen
        %, \daerahDosen % Daerah
        %, \negaraDosen % Negara
        
        \textsuperscript{*}\textit{Email: \href{mailto:\emailMahasiswa}{\emailMahasiswa}}
        
        % Dosen+Mahasiswa dan 1 Informasi ---------------------------------
%        \textbf{\namaDosen\textsuperscript{1}, \namaMahasiswa\textsuperscript{2*}}
%       
%        \textsuperscript{1, 2}Program Studi \programStudiDosen, \fakultasDosen, \perguruanTinggiDosen
%        %, \daerahDosen % Daerah
%        %, \negaraDosen % Negara
%        
%        \textsuperscript{*}\textit{Email: \href{mailto:\emailMahasiswa}{\emailMahasiswa}}

        % Custom [Bisa Ditulis Sendiri Sesuai Keperluan] ------------
%        \textbf{Penulis1\textsuperscript{1}, Penulis2\textsuperscript{2}, Penulis3\textsuperscript{3}, ...}
%        
%        \textsuperscript{1}Prodi, Fakultas, Perguruan Tinggi, ... \\
%        \textsuperscript{2}Prodi, Fakultas, Perguruan Tinggi, ...
%        
%        \textsuperscript{*}\textit{Email: \emailref{xyz@ecampus.ut.ac.id}}
    \end{center}
    
    % Load Abstrak
    \begin{abstract}
    \itshape\noindent\normalsize
    % Tulis Abstrak Bahasa Indonesia di Sini
    Jaringan Saraf Tiruan (JST) \textit{Backpropagation} merupakan salah satu JST yang menggunakan algoritma pembelajaran terawasi. Tujuan penelitian ini yaitu untuk mengetahui parameter dan mengukur akurasi ketepatan klasifikasi terhadap status mahasiswa Prodi Statistika Univeritas Terbuka. Berdasarkan hasil, simulasi didapatkan 15 parameter yang dapat memengaruhi status mahasiswa, di antaranya yaitu jenis kelamin, usia, pendidikan (SLTA/SMK, Diploma, S1, dan S2), status pernikahan, status pekerjaan (tidak bekerja, karyawan swasta, wiraswasta, dan PNS), tahun registrasi awal, jumlah registrasi, SKS tempuh, dan IPK. Sedangkan untuk akurasi ketepatan klasifikasi digunakan fungsi aktivasi dan \textit{learning rate} yang dapat menghasilkan nilai kuadrat tengah galat (KTG) yang minimum pada data \textit{training}. Hasil simulasi tersebut diterapkan pula pada data \textit{testing} dengan nilai \textit{cut-off point} sebesar 0,3481, maka didapatkan ketepatan akurasi dengan kurva ROC pada data \textit{training} untuk mahasiswa tidak aktif 99,43\% dan aktif 99,14\%, sedangkan pada data \textit{testing} mahasiswa tidak aktif 94,00\% dan aktif 93,94\%. Jadi dari penelitian ini dapat disimpulkan JST \textit{Backpropagation} merupakan salah satu metode yang sangat baik dalam penerapan metode klasifikasi. 
    
    \vspace{\baselineskip}\normalfont\noindent\footnotesize
    % Tulis Kata Kuncinya di Sini
    \textbf{Kata Kunci}: \textit{Backpropagation}, \textit{cut-off point}, fungsi aktivasi, jaringan saraf tiruan, \textit{learning rate}
\end{abstract}

%=================================%
%      ABSTRAK BAHASA INGGRIS     %
% Dapat digunakan jika diperlukan %
%=================================%

\begin{otherlanguage}{english}
\begin{abstract}
    \itshape\noindent\normalsize
    % Tulis Abstrak Bahasa Inggris di Sini
    Backpropagation Artificial Neural Network (ANN) is an ANN that uses a supervised learning algorithm. The purpose of this study is to determine the parameters and measure the accuracy of the classification accuracy of the student status of the Open University Statistics Study Program. Based on the results, the simulation obtained 15 parameters that can affect student status, including gender, age, education (Senior High School, Diploma, Bachelor, and Magister), marital status, employment status (not working, private employees, entrepreneurs, and civil servants), initial registration year, registration number, semester credit system, and GPA). Meanwhile, for the classification accuracy, the activation function and the learning rate are used minimum mean square of error (MST) on training data. The simulation results are also applied to the testing data with a cut-off point value of 0.3481, so the accuracy of the ROC curve is obtained in the training data for not active students is 99.43\% and 99.14\% active, while the testing data for not active students is 94.00\%. and active 93.94\%. So from this research, it can be concluded that ANN Backpropagation is a very good method in applying the classification method. 
    
    \vspace{\baselineskip}\normalfont\itshape\noindent\footnotesize
    % Tulis Kata Kuncinya di Sini
    \textbf{Keywords}: Backpropagation, cut-off point, activation function, artificial neural network, learning rate
\end{abstract}
\end{otherlanguage}
    
    \onehalfspacing % Line spacing 1,5
    
    % Load Bagian
    \chapter{Pendahuluan}

%=========================================================%
%              TULIS ISI PENDAHULUAN DI SINI              %
% Jangan Lupa untuk Menghapus Contoh Tulisan di Bawah Ini %
%=========================================================%

\section{Latar Belakang}

Proses penulisan karya ilmiah seperti makalah, skripsi, atau tesis sering kali membutuhkan format penulisan yang baku dan konsisten. Konsistensi ini meliputi gaya penulisan, penomoran halaman, daftar isi, daftar pustaka, hingga format kutipan. Namun, banyak penulis yang masih menghadapi kesulitan dalam mengatur format tersebut secara manual, terutama ketika terjadi perubahan pada isi dokumen. Pengaturan format manual ini tidak hanya memakan waktu, tetapi juga rentan terhadap kesalahan, sehingga dapat mengurangi fokus penulis pada substansi konten.

Penggunaan LaTeX menawarkan solusi yang efektif untuk mengatasi masalah ini. Sebagai \textit{typesetting system}, LaTeX dirancang untuk menghasilkan dokumen berkualitas tinggi dengan tata letak yang profesional dan konsisten. Dengan memanfaatkan class atau template yang sudah ada, penulis dapat memisahkan fokus antara konten dan presentasi. Oleh karenanya, dibuatlah template LaTeX yang dapat mempermudah proses penulisan makalah, sehingga penulis dapat lebih fokus pada isi tulisan dan penelitian yang dilakukan.

\section{Rumusan Masalah}

Berdasarkan latar belakang yang telah diuraikan, rumusan masalah dalam makalah ini adalah:

\begin{enumerate}
    \item Bagaimana cara merancang template LaTeX yang dapat memenuhi standar format penulisan makalah yang umum digunakan di Indonesia?
    \item Bagaimana template ini dapat membantu penulis untuk mengatur tata letak, daftar isi otomatis, penomoran halaman, dan daftar pustaka secara efisien?
    \item Fitur-fitur apa saja yang perlu diimplementasikan dalam template LaTeX ini agar dapat mempermudah proses penulisan karya ilmiah secara keseluruhan?
\end{enumerate}

\section{Tujuan}

Tujuan dari penyusunan template LaTeX ini adalah:

\begin{enumerate}
    \item Menciptakan sebuah template LaTeX yang konsisten, profesional, dan mudah digunakan untuk penulisan makalah, sesuai dengan standar yang berlaku.
    \item Menyediakan solusi praktis bagi mahasiswa dan akademisi agar dapat menyusun makalah dengan lebih cepat dan efisien, tanpa perlu khawatir tentang format penulisan.
    \item Memperkenalkan dan mempopulerkan penggunaan LaTeX sebagai alat bantu yang efektif dalam penulisan karya ilmiah.
\end{enumerate}
    \section{Metode} % Dapat diganti menjadi "Kerangka Pikir" jika perlu

%=========================================================%
%           TULIS METODE/KERANGKA PIKIR DI SINI           %
% Jangan Lupa untuk Menghapus Contoh Tulisan di Bawah Ini %
%=========================================================%

Bagian metode berisi tentang formulasi-formulasi dasar yang menjadi landasan untuk pengembangan hasil yang akan dilakukan. Selain itu, pada bagian ini dapat pula memuat tentang deskripsi lokasi penelitian, sumber data, teknik pengumpulan data dan analisis data.
    \section{Hasil dan Pembahasan}

%=========================================================%
%               TULIS ISI PEMBAHASAN DI SINI              %
% Jangan Lupa untuk Menghapus Contoh Tulisan di Bawah Ini %
%=========================================================%

Ini adalah inti dari artikel Anda, tempat Anda menyajikan dan menganalisis data atau argumen Anda. Sajikan data atau hasil riset Anda secara sistematis --- bisa juga menggunakan tabel, grafik, atau diagram untuk memvisualisasikan data agar lebih mudah dipahami.
    \section{Simpulan}

%=========================================================%
%                  TULIS PENUTUP DI SINI                  %
% Jangan Lupa untuk Menghapus Contoh Tulisan di Bawah Ini %
%=========================================================%

Berdasarkan hasil analisis kapabilitas proses terhadap layanan waktu terbit perizinan dan persepsi konsumen terhadap layanan di MPP ``Grha Tiyasa'' Kota Bogor dapat disimpulkan hal-hal berikut.

\begin{enumerate}[align=left, label=\alph*., labelsep=0pt, labelwidth=\parindent]
    \item Kapabilitas proses layanan waktu terbit perizinan dikatakan sudah baik. Hal ini ditunjukkan oleh nilai $P_\upp = 1$ dan $P_{\mathrm{pk}} = 0{,}49$. Nilai $P_{\mathrm{pk}}$ yang berada di antara 0 dan 1 menunjukkan bahwa rata-rata dari proses dalam batas spesifikasi, tetapi sebagian dari variasi proses berada di luar batas-batas spesifikasinya perlu ada peningkatan pada kualitas. Terdapat dua variasi proses yang berada di luar batas spesifikasi dan perlu ditingkatkan kualitasnya. Kondisi ini tercermin pada \textit{I-MR chart} yang menunjukkan terdapat 2 (dua) titik yang berada di luar batas spesifikasi.
    
    \item Kapabilitas proses persepsi atau penilaian konsumen terhadap layanan di MPP ``Grha Tiyasa'' dilihat dinilai cukup baik, namun masih perlu ada perbaikan. Hal ini ditunjukkan oleh nilai $P_\upp = 5{,}52$ dan $P_{\mathrm{pk}} = 1{,}70$. Nilai $P_\upp$ yang diperoleh lebih besar dari minimal nilai yang ditetapkan $P_\upp$, yakni 1,33. Hal ini berarti sebaran pengamatan atau lebar proses lebih kecil daripada lebar spesifikasi. MPP ``Grha Tiyasa'' perlu memperbaiki kualitas layanan yang diberikan kepada konsumen. Namun, dari \textit{I-MR chart} semua variasi proses berada pada batas spesifikasi.
\end{enumerate}
    
    % Daftar Pustaka
    \nocite{*}
    \bibliography{reference}
    
\end{document}