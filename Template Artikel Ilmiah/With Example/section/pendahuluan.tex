\section{Pendahuluan}

%=========================================================%
%              TULIS ISI PENDAHULUAN DI SINI              %
% Jangan Lupa untuk Menghapus Contoh Tulisan di Bawah Ini %
%=========================================================%

Proses penulisan karya ilmiah seperti makalah, skripsi, atau tesis sering kali membutuhkan format penulisan yang baku dan konsisten. Konsistensi ini meliputi gaya penulisan, penomoran halaman, daftar isi, daftar pustaka, hingga format kutipan. Namun, banyak penulis yang masih menghadapi kesulitan dalam mengatur format tersebut secara manual, terutama ketika terjadi perubahan pada isi dokumen. Pengaturan format manual ini tidak hanya memakan waktu, tetapi juga rentan terhadap kesalahan, sehingga dapat mengurangi fokus penulis pada substansi konten.

Penggunaan LaTeX menawarkan solusi yang efektif untuk mengatasi masalah ini. Sebagai \textit{typesetting system}, LaTeX dirancang untuk menghasilkan dokumen berkualitas tinggi dengan tata letak yang profesional dan konsisten. Dengan memanfaatkan \textit{class} atau \textit{template} yang sudah ada, penulis dapat memisahkan fokus antara konten dan presentasi. Oleh karenanya, dibuatlah template LaTeX yang dapat mempermudah proses penulisan artikel, sehingga penulis dapat lebih fokus pada isi tulisan dan penelitian yang dilakukan.

Sistematika artikel ilmiah ini terdiri dari Judul, Abstrak, Sub-judul (Pendahuluan, Metode, Hasil dan Pembahasan, Kesimpulan, dan Daftar Pustaka). Secara umum, pada bagian Pendahuluan dipaparkan latar belakang, perumusan masalah, dan tujuan penelitian. Jumlah halaman sebanyak 15--20 halaman, termasuk gambar, grafik, atau tabel (jika ada). Artikel memiliki nomor halaman di bagian kanan bawah dengan \textit{font} berukuran 10pt. Dalam setiap halaman artikel tidak ada teks dalam \textit{header} maupun \textit{footer}.