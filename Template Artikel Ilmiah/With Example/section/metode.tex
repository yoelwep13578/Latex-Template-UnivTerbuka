\section{Metode} % Dapat diganti menjadi "Kerangka Pikir" jika perlu

%=========================================================%
%           TULIS METODE/KERANGKA PIKIR DI SINI           %
% Jangan Lupa untuk Menghapus Contoh Tulisan di Bawah Ini %
%=========================================================%

Bagian metode berisi tentang formulasi-formulasi dasar yang menjadi landasan untuk pengembangan hasil yang akan dilakukan. Selain itu, pada bagian ini dapat pula memuat tentang deskripsi lokasi penelitian, sumber data, teknik pengumpulan data dan analisis data.

\subsection{Daftar Pustaka}

Daftar pustaka dan perujukan pengarang dalam batang tubuh artikel mengikuti \textit{style American Psychological Association 6th edition}. Daftar Pustaka hanya memuat referensi-referensi yang dirujuk di dalam artikel. Jenis-jenis dan cara perujukan pengarang dalam LaTeX dapat Anda lihat di \autoref{subsec:kutipan}.

Pustaka yang dirujuk diusahakan berasal dari referensi primer dan sedapat mungkin merupakan pustaka-pustaka terbitan 10 tahun terakhir. Isi Daftar Pustaka ditulis menggunakan format BibTeX. Contoh perintah penulisan BibTeX yang lengkap dapat dilihat di \autoref{subsec:mengelola-dapus}. Hasilnya akan tampil di daftar pustaka pada halaman akhir artikel ini.

\subsubsection{Referensi dari Buku}

Untuk menulis referensi dari buku, gunakan \textit{entry} \verb|@book|. \textit{Field} yang wajib ada di antaranya \verb|author|/\verb|editor|, \verb|year|, \verb|title|, dan \verb|publishser|. \textit{Field} yang opsional di antaranya \verb|volume|/\verb|number|, \verb|series|, \verb|address|, \verb|edition|, \verb|month|, dll.

Ini adalah contoh referensi buku BMP UT ``Aljabar Linear Elementer I'' (Edisi 3) yang ditulis oleh Rasjidin Jainudin Pamuntjak dan Warsito.

\begin{lstlisting}
@book{warsito-2022:ALE,
    author = {Rasjidin Jainudin Pamuntjak and Warsito},
    year = {2022},
    title = {{Aljabar Linear Elementer I}},
    edition = {3},
    address = {Tangerang Selatan},
    publisher = {Universitas Terbuka}
}
\end{lstlisting}

Ini adalah contoh referensi buku ``Panduan Mata Kuliah Karya Ilmiah Program Sarjana dan Diploma IV Universitas Terbuka'' yang ditulis oleh Mohamad Yunus dan kawan-kawan (\textit{dkk} atau \textit{et al.}).

\begin{lstlisting}
@book{yunus-2022:panduan-matkul-karil,
    author={Mohamad Yunus and 
            Kartono and 
            Tuti Purwaningsih and 
            Siti Aisyah and 
            Dewi Juliah Ratnaningsih and 
            Pepi Rospina and 
            Inggit Winarni and 
            Nang Budianto},
    year={2022},
    title={{Panduan Mata Kuliah Karya Ilmiah Program Sarjana dan Diploma IV Universitas Terbuka}},
    address={Tangerang Selatan},
    publisher={Universitas Terbuka}
}
\end{lstlisting}

\subsubsection{Referensi dari Jurnal}

Untuk menulis referensi dari jurnal, gunakan \textit{entry} \texttt{@article}. \textit{Field} yang wajib ada di antaranya \texttt{author}, \texttt{year}, \texttt{title}, dan \texttt{journal}. Field yang opsional di antaranya \texttt{volume}, \texttt{number}/\texttt{issue}, \texttt{pages}, \texttt{month}, \texttt{day}, dll.

Ini adalah contoh referensi jurnal ``Pelatihan LaTeX Menggunakan Overleaf untuk Meningkatkan Kemampuan Penulisan Karya Ilmiah bagi Dosen di Pringsewu'' yang ditulis oleh Fitriani dan kawan-kawan (\textit{dkk} atau \textit{et al.}).

\begin{lstlisting}
@article{fitriani-2024:pelatihan-latex,
    author = {Fitriani and Ahmad Faisol and Aang Nuryaman and Dian Kurniasari and Bernadhita Herindri Samodera Utami},
    year = {2024},
    title = {{Pelatihan LaTeX Menggunakan Overleaf untuk Meningkatkan Kemampuan Penulisan Karya Ilmiah bagi Dosen di Pringsewu}},
    journal = {Jurnal Pengabdian Kepada Masyarakat TABIKPUN},
    volume = {5},
    number = {3},
    doi = {10.23960/jpkmt.v5i3.184}
}
\end{lstlisting}

\subsubsection{Referensi dari Majalah}

Untuk menulis referensi dari majalah, gunakan \textit{entry} \texttt{@article} atau \texttt{@magazine}. \textit{Field} yang wajib ada di antaranya \texttt{author}, \texttt{year}, \texttt{title}, dan \texttt{journal}. \textit{Field} yang opsional di antaranya \texttt{month}, \texttt{day}, \texttt{pages}, dll.

Ini adalah contoh referensi majalah Risalah ``Pajak Negara Bikin Rakyat Sengsara'' karya Jeje Zaenudin.

\begin{lstlisting}
@magazine{risalah-2025:pajak-sengsara,
    author = {Jeje Zaenudin},
    year = {2025},
    month = {September},
    day = {1},
    title = {{Pajak Negara Bikin Rakyat Sengsara}},
    journal = {Risalah},
    volume = {63},
    pages = {45--61}
}
\end{lstlisting}

\subsubsection{Referensi dari Berita atau Koran}

Untuk menulis referensi dari berita/koran, gunakan \textit{entry} \texttt{@article}. \textit{Field} yang wajib ada di antaranya \texttt{author}, \texttt{year}, \texttt{title}, dan \texttt{journal}. \textit{Field} yang opsional di antaranya \texttt{month}, \texttt{day}, \texttt{pages}, dll.

Ini adalah contoh berita Kompas ``Aturan Baru Mapel Pendukung SNBP 2026 buat 59 Jurusan Kuliah'' yang ditulis oleh Sandra Desi Caesaria dan Mahar Prastiwi.

\begin{lstlisting}
@article{kompas-2025:mapel-snbp-26,
    author = {Sandra Desi Caesaria and Mahar Prastiwi},
    year = {2025},
    month = {Agustus},
    day = {26},
    title = {{Aturan Baru Mapel Pendukung SNBP 2026 buat 59 Jurusan Kuliah}},
    journal = {Kompas}
}
\end{lstlisting}

\subsubsection{Referensi dari Tesis/Disertasi yang \textit{Unpublished}}

\paragraph{Tesis Magister}

Untuk menulis referensi dari tesis magister, Anda bisa gunakan \textit{entry} \texttt{@mastersthesis}. \textit{Field} yang wajib ada di antaranya \texttt{author}, \texttt{year}, \texttt{title}, dan \texttt{school}. \textit{Field} yang opsional di antaranya \texttt{type}, \texttt{address}, \texttt{month}, dll.

Ini adalah contoh tesis magister (Strata 2) berjudul ``Analisis Faktor yang Memengaruhi Harga Saham Industri Perbankan di Bursa Efek Jakarta'' yang ditulis oleh Mula Pandapotan Sitinjak.

\begin{lstlisting}
@mastersthesis{sitinjak-2006:analisis-saham-bursa-efek-jakarta,
    author = {Mula Pandapotan Sitinjak},
    year = {2006},
    title = {{Analisis Faktor yang Memengaruhi Harga Saham Industri Perbankan di Bursa Efek Jakarta}},
    school = {Universitas Terbuka}
}
\end{lstlisting}

\paragraph{Disertasi Doktor}

Untuk menulis referensi dari disertasi doktor, Anda bisa gunakan \textit{entry} \texttt{@phdthesis}. \textit{Field} yang wajib ada di antaranya \texttt{author}, \texttt{year}, \texttt{title}, dan \texttt{school}. \textit{Field} yang opsional di antaranya \texttt{type}, \texttt{address}, \texttt{month}, dll.

Ini adalah contoh disertasi doktor (Strata 3) berjudul ``Proses Dialog dan Aksi Kolektif dalam Kegiatan Restorasi Lahan Gambut di Sumatera Selatan'' yang ditulis oleh Elly Rosana dan kawan-kawan (\textit{dkk} atau \textit{et al.}).

\begin{lstlisting}
@phdthesis{rosana-2025:restorasi-lahan-gambut,
    author = {Elly Rosana and Pudji Muljono and Djuara P. Lubis and Anna Fatchiya},
    year = {2025},
    title = {{Proses Dialog dan Aksi Kolektif dalam Kegiatan Restorasi Lahan Gambut di Sumatera Selatan}},
    school = {Institut Pertanian Bogor}
}
\end{lstlisting}

\subsubsection{Referensi dari Presentasi Makalah}

\paragraph{Presentasi Lisan}

Untuk menulis referensi dari presentasi lisan, gunakan \textit{entry} \texttt{@misc}. \textit{Field} yang wajib ada di antaranya \texttt{author}/\texttt{editor}, \texttt{year}, \texttt{title}, dan \texttt{howpublished}. 

Bagian \texttt{howpublished} bisa diisi dengan kalimat seperti ``Presentasi makalah dalam \linebreak $\langle$\textit{Seminar/Pertemuan}$\rangle$, $\langle$\textit{Kota}$\rangle$'' atau dalam bahasa Inggris seperti ``Paper presentation in \linebreak $\langle$\textit{Seminar/Pertemuan}$\rangle$,  $\langle$\textit{Kota}$\rangle$''

Ini adalah contoh presentasi makalah ``Unsur Linguistik dan Wacana: Memengaruhi Kompleksitas Soal Cerita Matematika'' yang dipresentasikan dalam Pertemuan Ilmiah Bahasa dan Sastra Indonesia di Jakarta (APA Bahasa Indonesia).

\begin{lstlisting}
@misc{sumawarti-2010:kompleks-soal-mtk,
    author = {Sumawarti},
    year = {2010},
    month = {November},
    title = {{Unsur Linguistik dan Wacana: Memengaruhi Kompleksitas Soal Cerita Matematika}},
    howpublished = {Presentasi makalah dalam Pertemuan Ilmiah Bahasa dan Sastra Indonesia, Jakarta}
}
\end{lstlisting}

Ini adalah contoh presentasi makalah ``Unsur Linguistik dan Wacana: Memengaruhi Kompleksitas Soal Cerita Matematika'' yang dipresentasikan dalam Pertemuan Ilmiah Bahasa dan Sastra Indonesia di Jakarta (APA Bahasa Inggris).

\begin{lstlisting}
@misc{sumawarti-2010:kompleks-soal-mtk,
    author = {Sumawarti},
    year = {2010},
    month = {November},
    title = {{Unsur Linguistik dan Wacana: Memengaruhi Kompleksitas Soal Cerita Matematika}},
    howpublished = {Paper presentation in Pertemuan Ilmiah Bahasa dan Sastra Indonesia, Jakarta}
}
\end{lstlisting}

\paragraph{Makalah dalam Prosiding (\textit{In Proceeding})}

Untuk menulis referensi dari salah satu makalah yang dipresentasikan dalam acara ilmiah, gunakan \textit{entry} \texttt{@inproceedings}. \textit{Field} yang wajib ada di antaranya \texttt{author}/\texttt{editor}, \texttt{year}, \texttt{title}, dan \texttt{booktitle}. \textit{Field} yang opsional di antaranya \texttt{volume}, \texttt{number}, \texttt{publisher}, \texttt{address}, \texttt{pages}, \texttt{series}, \texttt{month}, \texttt{day}, dll.

Ini adalah contoh makalah ``Pengembangan Pangan Darurat untuk Memenuhi Kebutuhan Gizi Masyarakat di Daerah Terdampak Bencana'' dalam Seminar Nasional Matematika, Sains, dan Teknologi di Bandung. 

\begin{lstlisting}
@inproceedings{mariam-2019:saintek-kebencanaan:pangan-darurat,
    author = {Siti Mariam},
    year = {2019},
    month = {Oktober},
    day = {3},
    title = {{Pengembangan Pangan Darurat untuk Memenuhi Kebutuhan Gizi Masyarakat di Daerah Terdampak Bencana}},
    booktitle = {{Seminar Nasional Matematika, Sains, dan Teknologi}},
    address = {Bandung}
}
\end{lstlisting}

\paragraph{Kumpulan Makalah dalam Prosiding}

Untuk menulis referensi dari seluruh kumpulan makalah yang dipresentasikan dalam acara ilmiah, gunakan \textit{entry} \texttt{@proceedings}. \textit{Field} yang wajib ada di antaranya \texttt{author}/\texttt{editor}, \texttt{year}, dan \texttt{title}. \textit{Field} yang opsional di antaranya \texttt{volume}, \texttt{number}, \texttt{publisher}, \texttt{address}, \texttt{pages}, \texttt{series}, \texttt{month}, \texttt{day}, dll.

Ini adalah contoh kumpulan makalah dalam prosiding ``Seminar Nasional Matematika, Sains, dan Teknologi''

\begin{lstlisting}
@proceedings{santoso-2019:saintek-kebencanaan,
    author = {Agus Santoso},
    year = {2019},
    month = {Oktober},
    day = {3},
    title = {{Prosiding Seminar Nasional Matematika, Sains, dan Teknologi}},
    address = {Tangerang Selatan},
    issn = {2088-0014}
}
\end{lstlisting}

\subsubsection{Referensi dari Situs/Artikel Internet}

Untuk menulis referensi dari situs/artikel di internet, gunakan \textit{entry} \texttt{@online}. \textit{Field} yang wajib ada di antaranya \texttt{author}, \texttt{year}, \texttt{title}, dan \texttt{url}. \textit{Field} yang opsional di antaranya \texttt{urldate}, \texttt{publisher}, \texttt{month}, \texttt{day}, dll.

Ini adalah contoh jawaban Revi Soekatno untuk pertanyaan ``Bagaimana penggunaan kata penghubung `di mana'? Apakah ini baku?'' di situs Quora.

\begin{lstlisting}
@online{soekatno-2021:konjungsi-dimana,
    author = {Revi Soekatno},
    year = {2021},
    month = {Mei},
    day = {11},
    title = {{Bagaimana penggunaan kata penghubung ``di mana''? Apakah ini baku?}},
    urldate = {September 14, 2025},
    url = {https://qr.ae/p2iXdi}
}
\end{lstlisting}

\subsubsection{Referensi dari Terbitan Pemerintah}

Untuk menulis referensi dari terbitan pemerintah, bisa gunakan \textit{entry} \texttt{@nisc} dengan \textit{field} wajib tersedia \texttt{author}, \texttt{year}, dan \texttt{title}. \textit{Field} opsional di antaranya \texttt{urldate} \texttt{month}, dan \texttt{day}.

Ini adalah contoh Undang-undang PPLH nomor 32 tahun 2009 yang diterbitkan oleh Badan Pemeriksa Keuangan Republik Indonesia.

\begin{lstlisting}
@misc{BPK-RI-2009:PPLH,
    author = {{Badan Pemeriksa Keuangan Republik Indonesia}},
    year = {2009},
    title = {{Undang-undang (UU) Nomor 32 Tahun 2009 tentang Perlindungan dan Pengelolaan Lingkungan Hidup}},
    url = {https://peraturan.bpk.go.id/Download/28100/UU%20Nomor%2032%20Tahun%202009.pdf}
}
\end{lstlisting}

\subsection{Formula/Rumus Matematika}

Untuk membuat sebuah formula/rumus matematika di LaTeX, cukup tuliskan formula tersebut menggunakan perintah matematika LaTeX. Jika dalam artikel terdapat beberapa rumus/formula yang ingin dijadikan acuan, maka rumus yang diacu dalam pembahasan harus diberi nomor dengan memakai \textit{perintah environment} \verb|equation| dan diberi \verb|\label{KATA_TUNJUK}| di akhir, seperti contoh ini.

\begin{equation}
    \frac{1}{\upM_a} \left( \int_{0}^{\infty} \upd \omega \frac{|S_0|^2}{N} \right)^{-1}
    \label{eq:contoh-rumus}
\end{equation}

Lalu untuk merujuk rumus tersebut, Anda bisa menggunakan \verb|\eqref{KATA_TUNJUK}| untuk disebut di tengah-tengah kalimat atau \verb|\autoref{KATA_TUNJUK}| untuk disebut di awal kalimat. Penunjukkan ini umumnya disebut sebagai referensi silang sehingga pembaca dapat mengeklik bagian tersebut untuk diarahkan ke sumber acuannya. Pemakaiannya haruslah seperti ini:

Rumus yang ada pada \eqref{eq:contoh-rumus} diperoleh dari contoh penulisan karya ilmiah FST UT. \autoref{eq:contoh-rumus} sengaja ditulis dengan dalam kurung pangkat minus satu agar lebih rapi.

Jangan seperti ini:

Rumus yang ada pada \autoref{eq:contoh-rumus} diperoleh dari contoh penulisan karya ilmiah FST UT. \eqref{eq:contoh-rumus} sengaja ditulis dengan dalam kurung pangkat minus satu agar lebih rapi.

Panduan lengkap, saran penulisan matematika dengan \LaTeX, serta penulisan teorema, lema, dan bukti dapat Anda lihat pada \autoref{subsec:teks-mathmode}.