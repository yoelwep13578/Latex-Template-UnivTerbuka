\section{Simpulan}

%=========================================================%
%                  TULIS PENUTUP DI SINI                  %
% Jangan Lupa untuk Menghapus Contoh Tulisan di Bawah Ini %
%=========================================================%

Berdasarkan hasil analisis kapabilitas proses terhadap layanan waktu terbit perizinan dan persepsi konsumen terhadap layanan di MPP ``Grha Tiyasa'' Kota Bogor dapat disimpulkan hal-hal berikut.

\begin{enumerate}[align=left, label=\alph*., labelsep=0pt, labelwidth=\parindent]
    \item Kapabilitas proses layanan waktu terbit perizinan dikatakan sudah baik. Hal ini ditunjukkan oleh nilai $P_\upp = 1$ dan $P_{\mathrm{pk}} = 0{,}49$. Nilai $P_{\mathrm{pk}}$ yang berada di antara 0 dan 1 menunjukkan bahwa rata-rata dari proses dalam batas spesifikasi, tetapi sebagian dari variasi proses berada di luar batas-batas spesifikasinya perlu ada peningkatan pada kualitas. Terdapat dua variasi proses yang berada di luar batas spesifikasi dan perlu ditingkatkan kualitasnya. Kondisi ini tercermin pada \textit{I-MR chart} yang menunjukkan terdapat 2 (dua) titik yang berada di luar batas spesifikasi.
    
    \item Kapabilitas proses persepsi atau penilaian konsumen terhadap layanan di MPP ``Grha Tiyasa'' dilihat dinilai cukup baik, namun masih perlu ada perbaikan. Hal ini ditunjukkan oleh nilai $P_\upp = 5{,}52$ dan $P_{\mathrm{pk}} = 1{,}70$. Nilai $P_\upp$ yang diperoleh lebih besar dari minimal nilai yang ditetapkan $P_\upp$, yakni 1,33. Hal ini berarti sebaran pengamatan atau lebar proses lebih kecil daripada lebar spesifikasi. MPP ``Grha Tiyasa'' perlu memperbaiki kualitas layanan yang diberikan kepada konsumen. Namun, dari \textit{I-MR chart} semua variasi proses berada pada batas spesifikasi.
\end{enumerate}