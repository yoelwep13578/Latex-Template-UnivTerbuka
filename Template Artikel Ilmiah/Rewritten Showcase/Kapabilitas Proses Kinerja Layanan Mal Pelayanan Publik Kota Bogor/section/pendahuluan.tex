\section{Pendahuluan}

%=========================================================%
%              TULIS ISI PENDAHULUAN DI SINI              %
% Jangan Lupa untuk Menghapus Contoh Tulisan di Bawah Ini %
%=========================================================%

Dalam rangka memberikan layanan kepada masyarakat yang mudah dan terintegrasi, Pemerintah meluncurkan suatu kebijakan untuk melakukan reformasi birokrasi dalam pelayanan publik, yaitu Mal Pelayanan Publik (MPP). Kebijakan MPP tertuang dalam Peraturan Menteri Pendayagunaan Aparatur Negara dan Reformasi Birokrasi (Permen PANRB) Nomor 23 Tahun 2017. Kegiatan yang diselenggarakan dalam MPP berupa kegiatan penyelenggaraan pelayanan publik atas barang, jasa, dan/atau pelayanan administrasi. Kegiatan ini merupakan perluasan fungsi pelayanan terpadu baik pusat maupun daerah, serta pelayanan Badan Usaha Milik Negara (BUMN)/Badan Usaha Milik Negara (BUMD)/swasta dalam rangka menyediakan pelayanan yang cepat, mudah, terjangkau, aman, dan nyaman. Untuk mengimplementasikan hal tersebut, pada bulan Agustus 2019 Kota Bogor meluncurkan BPP ``Grha Tiyasa'' dengan melibatkan 14 instansi dan 145 layanan. MPP ``Grha Tiyasa'' mendapatkan sambutan yang baik dari masyarakat. Hal ini dapat dilihat dari statistik pengunjung yang terus meningkat, yakni mencapai 53 159 orang dan melayani 52 683 layanan selama periode 20 Agustus 2019--10 April 2020. Indeks Kepuasan Masyarakat (IKM) terhadap layanan MPP ``Grha Tiyasa'' sangat baik (mencapai nilai A).

Salah satu kriteria penilaian dalam pelayanan publik adalah kapabilitas proses. Kapabilitas proses adalah analisis yang digunakan untuk mengetahui apakah proses kerja yang sedang berjalan memenuhi spesifikasi yang telah ditetapkan \cite{yuri2013tqm}. Analisis Kapabilitas Proses (\textit{Process Capability Analysis}) adalah suatu studi teknik menaksir kapabilitas proses. Taksiran kapabilias proses mungkin dalam bentuk distribusi probabilitas yang mempunyai bentuk, tengah (\textit{mean}), dan penyebaran (\textit{standard deviation}) tertentu \cite{pearn1992distributional, montgomery2009, mccormack2000capability, kotz2002process, wu2009overview}. \citeA{wooluru2014process} mengatakan kapabilitas proses adalah suatu metode yang menggabungkan teknik statistika dari kurva normal dan grafik kontrol dengan kriteria penilaian untuk menafsirkan dan menganalisis data yang mewakili suatu proses. \citeA{arcidiacono2017, harry1991} mengemukakan bahwa untuk seluruh siklus produksi suatu produk atau layanan diperlukan suatu teknik analisis yang dapat mengevaluasi secara benar mengenai suara pelanggan dan kinerja proses. Dalam hal ini terdapat keterkaitan antara kapabilitas proses dengan kemampuan proses, kinerja proses, dan sigma proses.

Kapabilitas proses ini merupakan suatu ukuran kinerja kritis yang menunjukkan proses mampu menghasilkan sesuai dengan spesifikasi produk yang diterapkan oleh manajemen berdasarkan kebutuhan dan ekspektasi pelanggan \cite{gasperz2004production}. Kapabilitas proses menunjukkan rentang suatu variasi suatu proses atau suatu besaran yang menunjukkan kemampuan dari suatu layanan publik untuk menghasilkan pelayanan yang sesuai dengan spesifikasi atau standar pelayanan yang ditentukan. Dengan kata lain, kapabilitas proses menunjukkan sampai seberapa jauh suatu proses mampu memenuhi spesifikasi atau standar yang diinginkan. Hasil analisis kapabilitas proses dapat digunakan antara lain untuk memperkirakan seberapa baik proses akan memenuhi toleransi, membantu pengembangan perancangan produk/layanan dalam memilih atau mengubah proses, dan mengurangi variabilitas dalam proses produksi/layanan \cite{harjosoedarmo1996}.

Suatu proses dikatakan kapabel apabila memenuhi tiga asumsi yaitu: karakteristik kualitas berdistribusi normal, proses terkendali, dan rata-rata proses berada diantara batas spesifikasi atas dan batas spesifikasi bawah. Batas spesifikasi disebut juga sebagai batas toleransi. Penentuan batas spesifikasi ini ditentukan berdasarkan kebutuhan konsumen yakni apa yang diharapkan konsumen terhadap produk/layanan yang diinginkan. Pada umumnya, penentuan batas spesifikasi ini ditentukan melalui riset pasar dan dikombinasikan dengan rancangan produk dan jasa (pelayanan). Terdapat dua indikator untuk mengukur kapabilitas proses yaitu indeks kapabilitas proses (C\textsubscript{p}) dan indeks performansi Kane (C\textsubscript{pk})

Pelayanan yang diberikan di suatu tempat layanan publik seperti MPP bisa bervariasi meskipun sudah mengikuti standar pelayanan yang telah diberikan oleh pemerintah setempat. Namun dalam kenyataan di lapangan mungkin terdapat beberapa pelayanan yang tidak memenuhi harapan konsumen. Variasi atau variabilitas layanan adalah ketidakseragaman layanan yang diberikan oleh suatu produk/jasa layanan yang tidak memenuhi spesifikasi. Pada umumnya, konsumen mengharapkan produk/jasa layanan memiliki variabilitas yang minimum. Oleh karena itu, suatu jasa layanan publik harus melakukan upaya peningkatan kualitas dan memastikan bahwa variasi/varabilitas karakteristik kualitas produk/jasa layanan yang mereka berikan pada konsumen masih dalam batas-batas yang dapat ditoleransikan oleh konsumen. Dengan kata lain, variasi layanan yang diberikan kepada konsumen masih berada dalam batas spesifikasi. Untuk menguji apakah variabiltas masih dalam karakteristik proses dan apakah proses mampu menghasilkan produk/jasa layanan yang diberikan oleh MPP ``Grha Tiyasa'' sesuai dengan spesifikasi, maka perlu dilakukan analisis menggunakan analisis kapabilitas proses. Berdasarkan alasan tersebut maka paper ini bertujuan memaparkan hasil analisis kapabilitas proses pada pelayanan publik di MPP ``Grha Tiyasa'' di Kota Bogor.