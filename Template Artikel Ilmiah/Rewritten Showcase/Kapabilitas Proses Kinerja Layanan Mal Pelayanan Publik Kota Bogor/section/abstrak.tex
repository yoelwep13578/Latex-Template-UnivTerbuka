\begin{abstract}
    \itshape\noindent\normalsize
    % Tulis Abstrak Bahasa Indonesia di Sini
    Mall Pelayanan Publik (MPP) Grha Tiyasadi Kota Bogor merupakan salah satu tempat yang memberikan pelayanan kepada masyarakat. Sebagai pelayan publik, Grha Tiyasa harus mengukur apakah layanan yang ditawarkan memenuhi harapan konsumen. Tujuan penulisan makalah ini adalah untuk menganalisis kapabilitas proses pemberian layanan perizinan dan persepsi konsumen terhadap layanan yang diberikan oleh MPP Grha Tiyasa. Kriteria analisis kapabilitas proses yang digunakan adalah Pp dan Ppk karena data tidak berdistribusi normal. Analisis menggunakan Minitab 16.0. Hasil penelitian menunjukkan bahwa kapabilitas proses pelayanan pada saat izin diterbitkan memiliki nilai P\textsubscript{p} = 1 dan P\textsubscript{pk} = 0,49. Nilai ini menunjukkan bahwa kapabilitas proses sangat baik. Kemampuan proses persepsi atau penilaian konsumen terhadap jasa cukup baik. Fakta ini ditunjukkan dengan nilai P\textsubscript{p} = 5,52 dan P\textsubscript{pk} = 1,70.
    
    \vspace{\baselineskip}\normalfont\noindent\footnotesize
    % Tulis Kata Kuncinya di Sini
    \textbf{Kata Kunci}: kapabilitas proses, layanan public, distribusi tidak normal
\end{abstract}

%=================================%
%      ABSTRAK BAHASA INGGRIS     %
% Dapat digunakan jika diperlukan %
%=================================%

\begin{otherlanguage}{english}
\begin{abstract}
    \itshape\noindent\normalsize
    % Tulis Abstrak Bahasa Inggris di Sini
    Public Service Mall (PSM) Grha Tiyasain Bogor City is one place that provides services to the public in terms of services. As a public service, Grha Tiyasa should measure whether the services offered to meet consumer expectations. The purpose of writing this paper is to analyze the capability of the process of issuing licensing services and consumer perceptions of services provided by PSMGrha Tiyasa. The process capability analysis criteria used are Pp and Ppk because the data is non-normal distribution. Analysis using Minitab 16.0. The results of the study show that the capability of the service process at the time of issuance of permits has a value of P\textsubscript{p} = 1 and P\textsubscript{pk} = 0.49. This value indicates that the process capability is excellent. The capability of the process of perception or assessment of consumers of services is quite good. This fact is indicated by the value of P\textsubscript{p} = 5.52 and P\textsubscript{pk} = 1.70.
    
    \vspace{\baselineskip}\normalfont\itshape\noindent\footnotesize
    % Tulis Kata Kuncinya di Sini
    \textbf{Keywords}: process capability, public service, non-normal distribution
\end{abstract}
\end{otherlanguage}