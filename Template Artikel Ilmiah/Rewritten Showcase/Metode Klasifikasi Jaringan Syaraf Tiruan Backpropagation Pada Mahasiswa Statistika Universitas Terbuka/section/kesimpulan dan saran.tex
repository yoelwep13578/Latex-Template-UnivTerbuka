\section{Kesimpulan}

%=========================================================%
%                  TULIS PENUTUP DI SINI                  %
% Jangan Lupa untuk Menghapus Contoh Tulisan di Bawah Ini %
%=========================================================%

Parameter dalam pengklasifikasian status mahasiswa prodi Statistika dengan menggunakan metode Jaringan Saraf Tiruan \textit{Backpropagation} berjumlah 15 parameter yang terdiri dari jenis kelamin, usia, pendidikan (SLTA/SMK, Diploma, S1, S2), status pernikahan, status pekerjaan (tidak bekerja, karyawan swasta, wiraswasta, PNS), tahun registrasi awal, jumlah registrasi, SKS tempuh, dan IPK. Akurasi ketepatan klasifikasi dengan Jaringan Saraf Tiruan \textit{Backpropagation} adalah sangat baik dengan hasil ketepatan klasifikasi pada data \textit{training} untuk mahasiswa tidak aktif sebesar 99,43\% dan mahasiswa aktif sebesar 99,14\%. Sedangkan pada data \textit{testing} untuk mahasiswa tidak aktif sebesar 94\% dan yang aktif tepat sebesar 93,94\%. Jadi, dari penelitian ini dapat disimpulkan JST \textit{Backpropagation} merupakan salah satu metode yang sangat baik dalam penerapan metode klasifikasi pada mahasiswa Statistika Universitas Terbuka.