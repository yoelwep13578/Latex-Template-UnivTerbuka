\section{Pendahuluan}

%=========================================================%
%              TULIS ISI PENDAHULUAN DI SINI              %
% Jangan Lupa untuk Menghapus Contoh Tulisan di Bawah Ini %
%=========================================================%

Universitas Terbuka (UT) memiliki beberapa fakultas, salah satunya Fakultas Matematika dan Ilmu Pengetahuan Alam (FMIPA). FMIPA didirikan pada tahun 1984 dan kini telah berganti nama menjadi Fakulta Sains dan Teknologi (FST). UT menawarkan program studi S1 Statistika dan telah berhasil meluluskan mahasiswa program studi Statistika dari berbagai macam karakteristik yang dimiliki mahasiswa UT \cite{prodi-statistika}.

Universitas Terbuka memiliki 39 kantor layanan yang tersebar di seluruh Indonesia dan salah satu universitas yang pertama kali menerapkan sistem pembelajaran jarak jauh dan terbuka. \textit{Jarak jauh} yaitu pembelajaran tidak dilakukan secara tatap muka, melainkan melalui media cetak seperti modul dan non-cetak seperti audio/video, komputer, internet \cite{univterbuka-2020:katalog}. Sedangkan \textit{terbuka} yaitu UT tidak ada pembatasan jangka waktu penyelesaian studi, tidak ada penerapan sistem \textit{drop out} (\textit{DO}), tidak ada pembatasan tahun kelulusan ijazah, umur, dan waktu pendaftaran \cite{tentang-ut}. Maka hal ini yang membuat UT memiliki kelebihan dari universitas lainnya di Indonesia.

Berdasarkan kelebihan berkuliah di UT ada permasalahan yang harus dihadapi universitas, salah satunya adalah status mahasiswa. Hal ini karena mahasiswa diberikan kemudahan untuk meregistrasikan dirinya kapan saja sehingga banyak mahasiswa yang bersifat tidak aktif pada suatu semester dan tidak diketahui pasti kapan mahasiswa tersebut menjadi mahasiswa aktif kembali. Penelitian ini dilakukan untuk mengklasifikasikan status mahasiswa aktif dan tidak aktif pada prodi Statistika dengan menggunakan metode \textit{Artificial Ingelligent} (\textit{AI}) yang salah satunya yaitu metode Jaringan Saraf Tiruan (JST) dengan menggunakan algoritma \textit{Backpropagation}.\textit{Backpropagation} merupakan algoritma \textit{supervised learning} dan \textit{multi layer} dengan mengubah bobot-bobot pada masing-masing lapisan dengan langkah pertama aalah mengatur jumlah lapisan, \textit{input}, tersembunyi, dan \textit{output} \cite{izhari-2020:analysis-of-backpropagation}. Pada tahun 1990--2000 \textit{AI} kurang berkembang karena keterbatasan dalam hal komputasi dan pendanaan, hingga pada tahun 2010 \textit{AI} membawa peningkatan besar, baik dari sektor negeri maupun swasta --- sehingga model JST dapat berkembang menjadi kompleks \cite{roman-2020:application-metamodel-ANN}.

Jaringan Saraf Tiruan (JST) merupakan suatu metode komputasi yang meniru cara kerja otak manusia yang terdiri dari neuron-neuron, dan antar-neuron tersebut saling berhubungan \cite{haykin-2007:neutral-networks}. Keuntungan menggunakan JST daripada metode klasifikasi lain yaitu sifatnya yang non-parametrik \cite{paola-1995:review-analysis-backpropagation}, sangat baik bila parameter yang digunakan cukup banyak, dan dapat bekerja dengan data yang cukup besar \cite{chiroma-2018:}.

Tujuan penelitian ini yaitu untuk mengetahui parameter-parameter yang digunakan untuk pengklasifikasian status mahasiswa yang aktif dan tidak aktif, yang nantinya akan dibuatkan suatu kebijakan atau solusi untuk mengatasi mahasiswa yang kemungkinan akan berstatus tidak aktif dan untuk mengetahui ketepatan klasifikasi status mahasiswa di prodi Statistika UT dengan menggunakan metode Jaringan Saraf Tiruan \textit{Backpropagation} \cite{Hasanah-2014:isoss}.

Data terbagi menjadi dua bagian, yaitu data \textit{training} dan data \textit{testing}. Pada data \textit{training} dilakukan simulasi dengan memaksimalkan fungsi aktivasi dan \textit{learning rate} ($\alpha$) \cite{Boithias-2012}. Hasil simulasi yang digunakan adalah model JST \textit{Backpropagation} dengan nilai bobot, fungsi aktivasi, dan \textit{learning rate} ($\alpha$) yang menghasilkan nilai KTG yang minimum. Hasil simulasi dari data \textit{training} diterapkan pada data \textit{testing} dan diukur akurasinya pada kedua data tersebut dengan menggunakan kurva ROC \cite{kalyan-2014}.