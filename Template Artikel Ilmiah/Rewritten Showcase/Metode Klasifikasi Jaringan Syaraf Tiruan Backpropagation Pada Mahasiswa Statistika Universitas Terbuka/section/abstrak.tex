\begin{abstract}
    \itshape\noindent\normalsize
    % Tulis Abstrak Bahasa Indonesia di Sini
    Jaringan Saraf Tiruan (JST) \textit{Backpropagation} merupakan salah satu JST yang menggunakan algoritma pembelajaran terawasi. Tujuan penelitian ini yaitu untuk mengetahui parameter dan mengukur akurasi ketepatan klasifikasi terhadap status mahasiswa Prodi Statistika Univeritas Terbuka. Berdasarkan hasil, simulasi didapatkan 15 parameter yang dapat memengaruhi status mahasiswa, di antaranya yaitu jenis kelamin, usia, pendidikan (SLTA/SMK, Diploma, S1, dan S2), status pernikahan, status pekerjaan (tidak bekerja, karyawan swasta, wiraswasta, dan PNS), tahun registrasi awal, jumlah registrasi, SKS tempuh, dan IPK. Sedangkan untuk akurasi ketepatan klasifikasi digunakan fungsi aktivasi dan \textit{learning rate} yang dapat menghasilkan nilai kuadrat tengah galat (KTG) yang minimum pada data \textit{training}. Hasil simulasi tersebut diterapkan pula pada data \textit{testing} dengan nilai \textit{cut-off point} sebesar 0,3481, maka didapatkan ketepatan akurasi dengan kurva ROC pada data \textit{training} untuk mahasiswa tidak aktif 99,43\% dan aktif 99,14\%, sedangkan pada data \textit{testing} mahasiswa tidak aktif 94,00\% dan aktif 93,94\%. Jadi dari penelitian ini dapat disimpulkan JST \textit{Backpropagation} merupakan salah satu metode yang sangat baik dalam penerapan metode klasifikasi. 
    
    \vspace{\baselineskip}\normalfont\noindent\footnotesize
    % Tulis Kata Kuncinya di Sini
    \textbf{Kata Kunci}: \textit{Backpropagation}, \textit{cut-off point}, fungsi aktivasi, jaringan saraf tiruan, \textit{learning rate}
\end{abstract}

%=================================%
%      ABSTRAK BAHASA INGGRIS     %
% Dapat digunakan jika diperlukan %
%=================================%

\begin{otherlanguage}{english}
\begin{abstract}
    \itshape\noindent\normalsize
    % Tulis Abstrak Bahasa Inggris di Sini
    Backpropagation Artificial Neural Network (ANN) is an ANN that uses a supervised learning algorithm. The purpose of this study is to determine the parameters and measure the accuracy of the classification accuracy of the student status of the Open University Statistics Study Program. Based on the results, the simulation obtained 15 parameters that can affect student status, including gender, age, education (Senior High School, Diploma, Bachelor, and Magister), marital status, employment status (not working, private employees, entrepreneurs, and civil servants), initial registration year, registration number, semester credit system, and GPA). Meanwhile, for the classification accuracy, the activation function and the learning rate are used minimum mean square of error (MST) on training data. The simulation results are also applied to the testing data with a cut-off point value of 0.3481, so the accuracy of the ROC curve is obtained in the training data for not active students is 99.43\% and 99.14\% active, while the testing data for not active students is 94.00\%. and active 93.94\%. So from this research, it can be concluded that ANN Backpropagation is a very good method in applying the classification method. 
    
    \vspace{\baselineskip}\normalfont\itshape\noindent\footnotesize
    % Tulis Kata Kuncinya di Sini
    \textbf{Keywords}: Backpropagation, cut-off point, activation function, artificial neural network, learning rate
\end{abstract}
\end{otherlanguage}