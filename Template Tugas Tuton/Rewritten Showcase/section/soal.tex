\chapter{Soal}

%======================================================%
%                  TULIS SOAL DI SINI                  %
% Jangan Lupa untuk Menghapus Contoh Soal di Bawah Ini %
%======================================================%

\begin{center}
    \bfseries
    TUGAS TUTORIAL SESI 3 \\
    EKMA4116 / MANAJEMEN / 4 SKS \\
    PROGRAM STUDI MANAJEMEN \\
    PERIODE 2025.1
\end{center}

\linesskip{1}

Warung Lezat adalah sebuah UMKM di bidang kuliner yang telah beroperasi selama 5 tahun di sebuah kota kecil. Warung ini dikenal dengan menu khas nusantara yang autentik dan harga terjangkau. Karena adanya pertumbuhan penduduk dan peningkatan minat terhadap kuliner lokal di kota besar, pemilik Warung Lezat mempertimbangkan untuk membuka \textit{outlet} baru di kawasan strategis kota besar tersebut. Sebelum memutuskan ekspansi, pemilik ingin melakukan studi kelayakan bisnis untuk mengevaluasi potensi pasar, operasional, keuangan, dan SDM yang mungkin timbul.

Diketahui untuk membuka \textit{outlet} baru di kawasan strategis kota besar dibutuhkan biaya sewa tempat dan renovasi mencapai Rp50.000.000, peralatan dapur dan perlengkapan sekitar Rp30.000.000, serta modal kerja awal sebesar Rp20.000.000. Diperkirakan pendapatan bersih bulanan mencapai Rp10.000.000

\linesskip{1}

\noindent Pertanyaan:

\begin{enumerate}
    \item Jelaskan analisis studi kelayakan aspek pemasaran yang tepat untuk mendukung ekspansi \textit{outlet} baru Warung Lezat ditinjau dari tingkat persaingan di area \textit{outlet} baru dan strategi promosi yang perlu diterapkan?
    
    \item Jelaskan analisis studi kelayakan aspek keuangan yang dapat mendukung ekspansi outlet baru Warung Lezat ditinjau dari estimasi investasi awal yang diperlukan dan titik impas (\textit{break-even point}) untuk outlet baru?
    
    \item Jelaskan tantangan yang mungkin dihadapi dalam pengelolaan sumber daya manusia (SDM) saat membuka \textit{outlet} baru Warung Lezat ditinjau dari masalah rekrutmen untuk karyawan yang akan ditempatkan pada outlet baru tersebut?
    
    \item Jelaskan analisis studi kelayakan aspek operasional yang dapat mendukung kelancaran operasional \textit{outlet} baru Warung Lezat ditinjau dari aspek peran pengadaan bahan baku?
\end{enumerate}