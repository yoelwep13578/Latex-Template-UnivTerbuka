% !TEX program = lualatex

% Template Makalah Tugas Tutorial [Rewritten Showcase]
% Versi 1.3.1

%   /‾‾‾‾‾‾‾‾‾‾‾‾‾‾‾‾‾‾‾‾‾‾‾‾‾‾‾/
%  /   INFORMASI KARYA TULIS   /
% /___________________________/
%    
%    Penulis         : [1] Yoeru Sandaru
%    Status Penulis  : [1] Mahasiswa
%    Alamat Afiliasi : [1] Program Studi Sistem Informasi,
%                          Fakultas Sains dan Teknologi,
%                          Universitas Terbuka
%    Korespondensi   : - (tidak ada)
%    Judul           : Peran Mahasiswa dalam Mewujudkan Indonesia Maju
%    
%    /‾‾‾‾‾‾‾‾‾‾‾‾‾‾‾‾‾‾‾‾‾‾‾‾‾‾‾‾‾‾‾‾‾‾‾‾‾‾‾‾‾‾‾‾‾/
%   /                DIFF: HEADING                /
%  /   [+ ditambah]  [- dihapus]  [--> diubah]   /
% /_____________________________________________/
%    
%    I    PENDAHULUAN
%          A. Latar Belakang
%          B. Rumusan Masalah
%          C. Tujuan
%   -II   KAJIAN TEORI
%    II   PEMBAHASAN
%         +A. Peran Mahasiswa dalam Mewujudkan Indonesia Maju
%             +1. Pendidikan
%             +2. Masyarakat Sosial
%             +3. Ekonomi
%             +4. Politik
%             +5. Kebudayaan
%         +B. Kendala dalam Mewujudkan Indonesia Maju
%             +1. Faktor Internal Mahasiswa
%             +2. Faktor Pendidikan
%             +3. Faktor Seni Budaya
%    III  PENUTUP
%          A. Kesimpulan
%          B. Saran
%    REFERENSI --> DAFTAR RUJUKAN
%    
%   /‾‾‾‾‾‾‾‾‾‾‾‾‾‾‾‾‾‾‾‾‾‾/
%  /   SETELAN TEMPLATE   /
% /______________________/
%    
%    Penomoran Heaidng : Alphanumeric Numbering
%    Citation Style    : APA 6
%    Bahasa Citation   : Inggris-Indonesia 
%                        (singkatan Bahasa Inggris dipertahankan)

%==========%
% PREAMBLE %
%==========%

\documentclass[a4paper, 12pt]{report}

% Margin & Geometry
\usepackage[
    left=3cm,
    right=3cm,
    top=3cm,
    bottom=3cm
]{geometry}

% Linking & Reffering
\usepackage[hidelinks]{hyperref}
\AtBeginDocument{
    \urlstyle{APACrm} % Apacite Roman
}

% Daftar Pustaka & APA 6
\usepackage{apacite}
\bibliographystyle{apacite}

% Bahasa APA 6
%\input{preset/APA-bahasa-indonesia.tex}
\renewcommand{\onemaskedcitationmsg}[1]{%
    \emph{(#1\ sitasi dihilangkan untuk tinjauan anonim)}}%
\renewcommand{\maskedcitationsmsg}[1]{%
    \emph{(#1\ sitasi dihilangkan untuk tinjauan anonim)}}%
\renewcommand{\authorindexname}{Indeks Penulis}% Nama Indeks Penulis
%%
%% A note before the references if a meta-analysis is reported.
\renewcommand{\APACmetaprenote}{%
    Referensi yang ditandai dengan tanda bintang menunjukkan studi
    yang dimasukkan dalam meta-analisis.}%
%%
%% Commands for specific types of @misc entries.
\renewcommand{\bibmessage}{Pesan}%
\renewcommand{\bibcomputerprogram}{Program komputer}%
\renewcommand{\bibcomputerprogrammanual}{Manual program komputer}%
\renewcommand{\bibcomputerprogramandmanual}{Program dan manual komputer}%
\renewcommand{\bibcomputersoftware}{Perangkat lunak}%
\renewcommand{\bibcomputersoftwaremanual}{Manual perangkat lunak}%
\renewcommand{\bibcomputersoftwareandmanual}{Perangkat lunak dan manual}%
\renewcommand{\bibprogramminglanguage}{Bahasa pemrograman}%
%%
%% Other labels
\renewcommand{\bibnodate}{t.t.\hbox{}}%   % ``tanpa tanggal''
\renewcommand{\BIP}{dalam proses terbit}%            % ``in press''
\renewcommand{\BOthers}[1]{et al.\hbox{}}% ``dan lain-lain''
\renewcommand{\BOthersPeriod}[1]{et al.\hbox{}}% ``dan lain-lain'' dengan titik
\renewcommand{\BIn}{Dalam}%                  % for ``In '' editor...
\renewcommand{\Bby}{oleh}%                  % for ``by '' editor... (in reprints)
\renewcommand{\BED}{Penyunt.}%          % penyunting
\renewcommand{\BEDS}{Penyunt.}%        % para penyunting
\renewcommand{\BTRANS}{Penerj.}%    % penerjemah
\renewcommand{\BTRANSS}{Penerj.}%   % para penerjemah
\renewcommand{\BTRANSL}{terj.}%   % terjemahan, for the year field
\renewcommand{\BCHAIR}{Ketua}%            % ketua simposium
\renewcommand{\BCHAIRS}{Para Ketua}%          % para ketua
\renewcommand{\BVOL}{Jilid}%        % jilid
\renewcommand{\BVOLS}{Jilid}%        % jilid
\renewcommand{\BNUM}{No.\hbox{}}%         % nomor
\renewcommand{\BNUMS}{No.\hbox{}}%       % nomor
\renewcommand{\BEd}{ed.\hbox{}}%          % edisi
\renewcommand{\BCHAP}{bab}%        % bab
\renewcommand{\BCHAPS}{bab}%       % bab
\renewcommand{\BPG}{h.\hbox{}}%           % halaman
\renewcommand{\BPGS}{hlm.\hbox{}}%         % halaman
%% Default technical report type name.
\renewcommand{\BTR}{Lap. Tek.}
%% Default PhD thesis type name.
\renewcommand{\BPhD}{Disertasi doktor}
%% Default unpublished PhD thesis type name.
\renewcommand{\BUPhD}{Disertasi doktor tidak diterbitkan}
%% Default master's thesis type name.
\renewcommand{\BMTh}{Tesis magister}
%% Default unpublished master's thesis type name.
\renewcommand{\BUMTh}{Tesis magister tidak diterbitkan}
%%
\renewcommand{\BAuthor}{Penulis}% ``Penulis'' jika penerbit = penulis
\renewcommand{\BOWP}{Karya asli diterbitkan}%
\renewcommand{\BREPR}{Dicetak ulang dari}%
\renewcommand{\BAvailFrom}{Tersedia dari\ }%       Situs web; perhatikan spasi.
%% The argument is the date on which it was last checked.
\renewcommand{\BRetrieved}[1]{Diakses {#1}, dari\ }% Situs web; perhatikan spasi.
\renewcommand{\BRetrievedFrom}{Diakses dari\ }% Situs web; perhatikan spasi.
\renewcommand{\BMsgPostedTo}{Pesan dikirim ke\ }%     Pesan; perhatikan spasi.

\renewcommand{\BBAB}{dan}% between authors in in-text citation

% Bahasa & Istilah
\usepackage[indonesian]{babel}

% Grafis
\usepackage{graphicx}
\usepackage{xcolor}

% Matematika / Math Mode
\usepackage{amsmath,amsthm,amssymb}
%\usepackage{MnSymbol}
\usepackage{xfrac}
\usepackage{unicode-math}

\allowdisplaybreaks

\newtheorem{theorem}{Teorema}
\newtheorem{lemma}[theorem]{Lema}
\newtheorem{definition}[theorem]{Definisi}
\newtheorem{example}[theorem]{Contoh}

% Float
\usepackage{float}

% PDF Landscape
\usepackage{pdflscape}

% Tabel
\usepackage{tabularray}

\SetTblrStyle{caption}{font=\vspace{-\baselineskip}\singlespacing\small}
\SetTblrStyle{capcont}{font=\vspace{-\baselineskip}\singlespacing\small}
\SetTblrStyle{note}{font=\small}
\SetTblrStyle{remark}{font=\small}

%========================%
% FONT & FONT SELECTIONS %
%========================%

\usepackage[T1]{fontenc}
\usepackage{fontspec}

% Serif/Main Font --------------------------- %
%\setmainfont{Latin Modern Roman}             % LM (ada di TeXLive)
\setmainfont{Times New Roman}[Ligatures=Rare] % TNR + Ligature
%\setmainfont{TeX Gyre Termes}                % Alternatif Mirip TNR (ada di TeXLive)
%\setmainfont{XITS}                           % Alternatif Mirip TNR (ada di TeXLive)

% Sans-Serif ----------------------------------------- %
\setsansfont{Latin Modern Sans}[Scale=MatchLowercase]  % LM (ada di TeXLive)
%\setsansfont{Alegreya Sans}[Scale=MatchLowercase]     % Alegreya (ada di TeXLive)
%\setsansfont{Noto Sans}[Scale=MatchLowercase]         % Noto
%\setsansfont{Calibri}[Scale=MatchUppercase]           % Calibri

% Monospace ----------------------------------------------- % 
\setmonofont{Latin Modern Mono Light}[Scale=MatchLowercase] % LM (ada di TeXLive)
%\setmonofont{Courier New}[Scale=MatchLowercase] % Courier New (Windows < 10)
%\setmonofont{Cascadia Code Light}[              % Cascadia Code (Windows >= 10)
%    BoldFont={Cascadia Code Bold},
%    ItalicFont={Cascadia Code Light Italic},
%    BoldItalicFont={Cascadia Code Bold Italic},
%    Scale=MatchLowercase
%]
%\setmonofont{Fira Code Light}[                  % Fira Code
%    BoldFont={Fira Code Medium},
%    Contextuals=Alternate,
%    Scale=MatchLowercase
%]
%\setmonofont{JetBrains Mono Thin}[              % JetBrains Mono
%    Contextuals={Alternate},
%    BoldFont={JetBrains Mono Bold},
%    BoldItalicFont={JetBrains Mono Bold Italic},
%    ItalicFont={JetBrains Mono Thin Italic},
%    Scale=MatchLowercase
%]

% Font Matematika -------------- %
%\setmathfont{Latin Modern Math} % Samaan dgn Latin Modern
\setmathfont{XITS Math}          % Samaan dgn Times/TNR

%============%
% CODE BLOCK %
%============%

\usepackage{listings}
%\usepackage{lstfiracode}

\lstdefinelanguage{JavaScript}{ % Setting utk JavaScript karena tdk support
    keywords={typeof, new, true, false, catch, function, return, null, catch, switch, var, if, in, while, do, else, case, break},
    ndkeywords={class, export, boolean, throw, implements, import, this},
    sensitive=false,
    comment=[l]{//},
    morecomment=[s]{/*}{*/},
    morestring=[b]',
    morestring=[b]"
}

\lstset{
    %style=FiraCodeStyle,
    basicstyle=\ttfamily,
    numberstyle=\footnotesize\color{gray},
    numberbychapter=false,
    captionpos=t,
    breaklines=true,
    frame={top|bottom},
    showstringspaces=false,
    commentstyle=\itshape\color{monokai-comment},
    keywordstyle=\bfseries\color{monokai-purple},
    keywordstyle={[2]{\itshape\bfseries\color{monokai-purple}}},
    keywordstyle={[3]{\itshape\bfseries\color{monokai-blue}}},
    ndkeywordstyle=\itshape\bfseries\color{monokai-orange},
    stringstyle=\color{monokai-green},
    identifierstyle=\color{black},
}

%==========%
% TEXT BOX %
%==========%

\usepackage[most]{tcolorbox}

% Block Quote
\newtcolorbox{blockquote}{
    parbox=false,
    boxrule=0pt,
    frame hidden,
    colback=gray!3,
    enhanced,
    sharp corners,
    borderline west={2pt}{0pt}{black}
}

% Block Quote Indent
\newtcolorbox{indentbquote}{
    size=minimal,
    parbox=false,
    boxrule=0pt,
    frame hidden,
    colback=gray!3,
    enhanced,
    sharp corners,
    left=1.24cm,
    right=1.24cm,
    top=10pt,
    bottom=10pt,
    borderline west={2pt}{0pt}{black}
}

% Text Box
\newtcolorbox{textbox}{
    parbox=false,
    sharp corners,
    colback=white,
    boxrule=1pt
}

%============================%
% FORMATTING TEKS & PARAGRAF %
%============================%

% Parskip Paragraf
\usepackage[indent=1.24cm]{parskip}
%\setlength{\parskip}{.5\baselineskip}

% Line Spacing
\usepackage{setspace}

% Keterangan
\usepackage[font=small]{caption}

%============================%
% FORMATTING TEKS & PARAGRAF %
%============================%

\usepackage{titlesec}

% PILIH SATU -------------------------------------
% Alphanumeric Numbering
\setcounter{secnumdepth}{5}

\renewcommand{\thesection}{\Roman{section}.}
\renewcommand{\thesubsection}{\Alph{subsection}.}
\renewcommand{\thesubsubsection}{\arabic{subsubsection}.}
\renewcommand{\theparagraph}{\alph{paragraph}.}
\renewcommand{\thesubparagraph}{\arabic{subparagraph})}

%\titleformat{\chapter}[block]
%{\normalfont\bfseries\centering\LARGE}
%{}
%{0em}
%{}

%\titleformat{\chapter}[display]
%{\normalfont\bfseries\centering\Large}
%{\MakeUppercase{Bab \thechapter}}
%{-.5em}
%{\MakeUppercase}

\titleformat{\section}[hang]
{\normalfont\bfseries\raggedright}
{\hspace{\titleindent}\llap{\parbox[b]{\titleindent}{\normalfont\bfseries\thesection\hfill}}}
{0em}
{\MakeUppercase}

\titleformat{\subsection}[hang]
{\normalfont\bfseries\raggedright}
{\hspace{\titleindent}\llap{\parbox[b]{\titleindent}{\normalfont\bfseries\thesubsection\hfill}}}
{0em}
{}

\titleformat{\subsubsection}[hang]
{\normalfont\bfseries\itshape\raggedright}
{\hspace{\titleindent}\llap{\parbox[b]{\titleindent}{\normalfont\bfseries\thesubsubsection\hfill}}}
{0em}
{}

\titleformat{\paragraph}[runin]
{\normalfont\bfseries\normalsize}
{\hspace{\titleindent}\llap{\parbox[b]{\titleindent}{\normalfont\bfseries\theparagraph\hfill}}}
{0em}
{}[.]

\titleformat{\subparagraph}[runin]
{\normalfont\bfseries\itshape\normalsize}
{\hspace{\titleindent}\llap{\parbox[b]{\titleindent}{\normalfont\bfseries\thesubparagraph\hfill}}}
{0em}
{}[.]
% Multilevel Numbering
%\setcounter{secnumdepth}{5}

\renewcommand{\thesection}{\arabic{section}.}
\renewcommand{\thesubsection}{\arabic{section}.\arabic{subsection}}

\titleformat{\section}[block]
{\normalfont\bfseries\raggedright\Large}
{\normalfont\bfseries\thesection}
{1em}
{\MakeUppercase}

\titleformat{\subsection}[block]
{\normalfont\bfseries\raggedright\Large}
{\normalfont\bfseries\thesubsection}
{1em}
{}

\titleformat{\subsubsection}[block]
{\normalfont\bfseries\itshape\raggedright\large}
{\normalfont\bfseries\thesubsubsection}
{1em}
{}

\titleformat{\paragraph}[runin]
{\normalfont\bfseries\normalsize}
{\normalfont\bfseries\theparagraph}
{1em}
{}[.]

\titleformat{\subparagraph}[runin]
{\normalfont\bfseries\itshape\normalsize}
{\normalfont\bfseries\thesubparagraph}
{1em}
{}[.]

% Spacing Heading
\titlespacing*{\chapter}{0pt}{*-6}{\baselineskip}
\titlespacing*{\section}{0pt}{*2}{*1.5}
\titlespacing*{\subsection}{0pt}{*2}{*1.5}
\titlespacing*{\subsubsection}{0pt}{*2}{*1.5}
\titlespacing*{\paragraph}{0pt}{*1.5}{.5em}
\titlespacing*{\subparagraph}{0pt}{*1.5}{.5em}

% Label Counter
\counterwithout{table}{chapter}
\counterwithout{figure}{chapter}
\counterwithout{equation}{chapter}

% List
\usepackage{enumitem}

\setlist[enumerate]{leftmargin=\parindent}
\setlist[itemize]{leftmargin=\parindent}

\newlist{enumerateLeft}{enumerate}{5}
\setlist[enumerateLeft]{
    label=\arabic*.,
    align=left, 
    labelsep=0pt, 
    labelwidth=\parindent, 
    leftmargin=\parindent
}
\newlist{itemizeLeft}{itemize}{5}
\setlist[itemizeLeft]{
    label=\textbullet,
    align=left, 
    labelsep=0pt, 
    labelwidth=\parindent, 
    leftmargin=\parindent
}

\newlist{essaylist}{enumerate}{2}
\setlist[essaylist,1]{
    leftmargin=\parindent, 
    labelwidth=\parindent, 
    labelsep=0pt, 
    align=left, 
    label=\arabic*.
    }
\setlist[essaylist,2]{
    leftmargin=0pt, 
    labelwidth=\parindent, 
    labelsep=0pt, 
    align=left, 
    label=\arabic{essaylisti}. \alph*.
    }

%============%
% PAGE STYLE %
%============%

\usepackage{fancyhdr}
\usepackage{lastpage}

\fancypagestyle{fancy}{
    \fancyhead{}
    \renewcommand{\headrulewidth}{0pt}
}
\fancypagestyle{plain}[fancy]{}

%============%
% DAFTAR ISI %
%============%

\usepackage[titles]{tocloft}
\setcounter{tocdepth}{5}
%\tocloftpagestyle{fancy}

% Load Variabel
%======================%
% NORMAL TEXT VARIABLE %
%======================%

\newcommand{\judul}{Metode Klasifikasi Jaringan Saraf Tiruan \textit{Backpropagation} Pada Mahasiswa Statistika Universitas Terbuka}

% Informasi Waktu
\newcommand{\tanggalLengkap}{\today}
\newcommand{\tahun}{\the\year}

% Informasi Mahasiswa
\newcommand{\namaMahasiswa}{Yoeru Sandaru}
\newcommand{\niMahasiswa}{081298765432}
\newcommand{\programStudi}{Matematika}
\newcommand{\fakultas}{Fakultas Sains dan Teknologi}
\newcommand{\utDaerah}{UT Medan}
\newcommand{\perguruanTinggi}{Universitas Terbuka}
\newcommand{\daerahMahasiswa}{Medan}
\newcommand{\negaraMahasiswa}{Indonesia}

\newcommand{\emailMahasiswa}{\niMahasiswa @ecampus.ut.ac.id} % Email E-Campus sesuai NIM

% Informasi Tutor/Dosen Pembimbing
\newcommand{\namaDosen}{Revo Wibowo}
\newcommand{\niDosen}{081298765432}
\newcommand{\programStudiDosen}{Matematika}
\newcommand{\fakultasDosen}{Fakultas Sains dan Teknologi}
\newcommand{\utDaerahDosen}{UT Medan}
\newcommand{\perguruanTinggiDosen}{Universitas Terbuka}
\newcommand{\daerahDosen}{Medan}
\newcommand{\negaraDosen}{Indonesia}

\newcommand{\emailDosen}{bowo@ecampus.ut.ac.id}

%==========================%
% ANOTHER COMMAND VARIABLE %
%==========================%

\newcommand{\linesskip}[1]{\vspace{#1\baselineskip}}
\newcommand{\titleindent}{1cm}

% Keterangan dan Sumber
\newcommand{\longcaption}[1]{\caption{\begin{tabular}[t]{@{}l@{}}#1\end{tabular}}}
\newcommand{\tablesource}[1]{\vspace{.3\baselineskip}\caption*{Sumber: #1}\vspace{-\baselineskip}}
\newcommand{\tablesourceleft}[2]{
    
    \raggedright\medskip\hspace{#1}\small Sumber: #2}
\newcommand{\figuresource}[1]{\vspace{-.9em}\caption*{Sumber: #1}\vspace{-.3\baselineskip}}
\newcommand{\lstsource}[1]{\begin{center}\vspace{-1.3\baselineskip}\singlespacing\small Sumber: #1 \vspace{.5\baselineskip}\end{center}}

% Notasi Matematika Tegak
\newcommand{\deriv}{\mathrm{d}}
\newcommand{\adj}[1]{\mathrm{adj}#1}
\newcommand{\euler}{\mathrm{e}}
\newcommand{\imaginary}{\mathrm{i}}

\newcommand{\upa}{\mathrm{a}}
\newcommand{\upb}{\mathrm{b}}
\newcommand{\upc}{\mathrm{c}}
\newcommand{\upd}{\mathrm{d}}
\newcommand{\upe}{\mathrm{e}}
\newcommand{\upf}{\mathrm{f}}
\newcommand{\upg}{\mathrm{g}}
\newcommand{\uph}{\mathrm{h}}
\newcommand{\upi}{\mathrm{i}}
\newcommand{\upj}{\mathrm{j}}
\newcommand{\upk}{\mathrm{k}}
\newcommand{\upl}{\mathrm{l}}
\newcommand{\upm}{\mathrm{m}}
\newcommand{\upn}{\mathrm{n}}
\newcommand{\upo}{\mathrm{o}}
\newcommand{\upp}{\mathrm{p}}
\newcommand{\upq}{\mathrm{q}}
\newcommand{\upr}{\mathrm{r}}
\newcommand{\ups}{\mathrm{s}}
\newcommand{\upt}{\mathrm{t}}
\newcommand{\upu}{\mathrm{u}}
\newcommand{\upv}{\mathrm{v}}
\newcommand{\upw}{\mathrm{w}}
\newcommand{\upx}{\mathrm{x}}
\newcommand{\upy}{\mathrm{y}}
\newcommand{\upz}{\mathrm{z}}

\newcommand{\upA}{\mathrm{A}}
\newcommand{\upB}{\mathrm{B}}
\newcommand{\upC}{\mathrm{C}}
\newcommand{\upD}{\mathrm{D}}
\newcommand{\upE}{\mathrm{E}}
\newcommand{\upF}{\mathrm{F}}
\newcommand{\upG}{\mathrm{G}}
\newcommand{\upH}{\mathrm{H}}
\newcommand{\upI}{\mathrm{I}}
\newcommand{\upJ}{\mathrm{J}}
\newcommand{\upK}{\mathrm{K}}
\newcommand{\upL}{\mathrm{L}}
\newcommand{\upM}{\mathrm{M}}
\newcommand{\upN}{\mathrm{N}}
\newcommand{\upO}{\mathrm{O}}
\newcommand{\upP}{\mathrm{P}}
\newcommand{\upQ}{\mathrm{Q}}
\newcommand{\upR}{\mathrm{R}}
\newcommand{\upS}{\mathrm{S}}
\newcommand{\upT}{\mathrm{T}}
\newcommand{\upU}{\mathrm{U}}
\newcommand{\upV}{\mathrm{V}}
\newcommand{\upW}{\mathrm{W}}
\newcommand{\upX}{\mathrm{X}}
\newcommand{\upY}{\mathrm{Y}}
\newcommand{\upZ}{\mathrm{Z}}

% Code Snippet Berwarna
\newcommand{\inlinesnippet}[1]{\colorbox{gray!10}{\lstinline‖#1‖}}
\newcommand{\verbsnippet}[1]{\colorbox{gray!10}{\verb‖#1‖}}

% Email AutoLink
\newcommand{\emailref}[1]{\href{mailto:#1}{#1}}

%=====================%
% PENGGANTIAN ISTILAH %
%=====================%

\addto\captionsindonesian{\renewcommand{\abstractname}{Abstrak}} % Abstract
\addto\captionsindonesian{\renewcommand{\bibname}{Daftar Pustaka}} % Bibliography
\addto\captionsindonesian{\renewcommand{\refname}{Daftar Pustaka}} % Reference

\renewcommand{\chapterautorefname}{Bab}
\renewcommand{\sectionautorefname}{Bagian}
\renewcommand{\subsectionautorefname}{Sub-bagian}
\renewcommand{\subsubsectionautorefname}{Sub-sub-bagian}
\renewcommand{\paragraphautorefname}{Paragraf}
\renewcommand{\subparagraphautorefname}{Sub-paragraf}

\renewcommand{\figureautorefname}{Gambar}
\renewcommand{\tableautorefname}{Tabel}
\renewcommand{\equationautorefname}{Persamaan}
\renewcommand{\lstlistingname}{Kode}

\DeclareTblrTemplate{contfoot-text}{normal}{Lanjutan di halaman berikutnya}
\SetTblrTemplate{contfoot-text}{normal}
\DeclareTblrTemplate{conthead-text}{normal}{(Lanjutan)}
\SetTblrTemplate{conthead-text}{normal}

\DeclareTblrTemplate{remark-tag}{normal}{
    \UseTblrFont {remark-tag} \InsertTblrRemarkTag
}
\SetTblrTemplate{remark-tag}{normal}

%====================%
% MODIFIKASI ABSTRAK %
%====================%

\makeatletter
\renewenvironment{abstract}{%
    \if@twocolumn
    \section*{\abstractname}%
    \else %% <- \small dihapus
    \begin{center}%
        {\bfseries \normalsize\abstractname\vspace{\z@}}%  %% <- \normalsize ditambah
    \end{center}%
    \quotation
    \fi}
{\if@twocolumn\else\endquotation\fi}
\makeatother


%=============%
% COLOR NAMES %
%=============%

\definecolor{monokai-bg}{RGB}{250,250,250}     % Light gray background
\definecolor{monokai-comment}{RGB}{117,113,94} % Grayish comments
\definecolor{monokai-purple}{RGB}{137,89,168}  % Purple for keywords
\definecolor{monokai-green}{RGB}{58,110,59}    % Green for strings
\definecolor{monokai-orange}{RGB}{253,151,31}  % Orange for specials
\definecolor{monokai-blue}{RGB}{81,154,186}    % Blue for types/functions


%============+&
% ISI DOKUMEN &
%=============&

\begin{document}
    
    % Line Spacing 1,5
    \onehalfspacing
    
    % Load Bagian
    \begin{titlepage}
    
    \centering
    
    % Judul
    {\Large\textbf{\MakeUppercase{\judul}}}
    
    \large
    \textbf{\MakeUppercase{Mata Kuliah \namaMataKuliah}} \\
    \textbf{\kodeMataKuliah}
    
    \normalsize 
    
    % Logo Perguruan Tinggi
    \vfill
    \includegraphics[width=.4\linewidth]{image/Logo_Universitas_Terbuka.png}
    \vfill
    
    {\large \textbf{TUTOR PENGAMPU}} \\
    
    \namaTutorPengampu
    
    \linesskip{1}
    
    {\large \textbf{DISUSUN OLEH}}
    
    \begin{tblr}{
            colspec={l c l}, 
            colsep=0pt, 
            rowsep=0pt, 
            column{2}={colsep=4pt}
            }
        Nama &:& \namaMahasiswa \\
        NIM &:& \nimMahasiswa \\
        Kode Kelas &:& \kodeKelas
    \end{tblr}
    
    \linesskip{2}
    
    \large
    \textbf{\MakeUppercase{Program Studi \programStudi}} \\
    \textbf{\MakeUppercase{Fakultas \fakultas}} \\
    \textbf{\MakeUppercase{UPBJJ \utDaerah}} \\
    \textbf{\MakeUppercase{\perguruanTinggi}} \\
    \textbf{\tahun}
    
    \normalsize
    
\end{titlepage}
    
    \pagestyle{fancy}
    \pagenumbering{roman}
    \chapter*{Kata Pengantar}

%=========================================================%
%               TULIS KATA PENGANTAR DI SINI              %
% Jangan Lupa untuk Menghapus Contoh Tulisan di Bawah Ini %
%=========================================================%

Puji dan syukur penulis panjatkan kepada Tuhan yang Maha Esa atas berkat dan rahmat-Nya, sehingga penulis dapat menyelesaikan makalah ini dengan judul “Peran Mahasiswa dalam Mewujudkan Indonesia Maju”. Dalam penulisan makalah ini, penulis juga berterima kasih kepada \namaTutorPengampu\ selaku tutor mata kuliah Bahasa Indonesia yang telah membimbing penulis melalui materi inisiasi, serta kepada teman-teman dalam grup daring mata kuliah Bahasa Indonesia yang telah membagikan inspirasinya di dalam kelompok belajar.

Makalah ini ditulis untuk memenuhi tugas kedua dalam tutorial \textit{online} mata kuliah Bahasa Indonesia di Universitas Terbuka. Makalah ini membahas tentang peran dan kontribusi mahasiswa dalam mewujudkan Indonesia Maju, sehingga pembaca dapat mengerti peran-peran yang dilakukan oleh mahasiswa dalam proses perwujudan Indonesia Maju.

Penulis menyadari bahwa makalah ini masih memiliki kekurangan. Oleh karena itu, penulis berharap adanya saran demi perbaikan karya tulis yang akan datang. Penulis memohon maaf bila ada kesalahan yang kurang berkenan.

\vfill
\hfill \daearhMahasiswa, \tanggalLengkap

\linesskip{1}

\hfill \namaMahasiswa
\vfill
    \tableofcontents
    
    \clearpage
    
    \pagenumbering{arabic}
    \chapter{Pendahuluan}

%=========================================================%
%              TULIS ISI PENDAHULUAN DI SINI              %
% Jangan Lupa untuk Menghapus Contoh Tulisan di Bawah Ini %
%=========================================================%

\section{Latar Belakang}

Proses penulisan karya ilmiah seperti makalah, skripsi, atau tesis sering kali membutuhkan format penulisan yang baku dan konsisten. Konsistensi ini meliputi gaya penulisan, penomoran halaman, daftar isi, daftar pustaka, hingga format kutipan. Namun, banyak penulis yang masih menghadapi kesulitan dalam mengatur format tersebut secara manual, terutama ketika terjadi perubahan pada isi dokumen. Pengaturan format manual ini tidak hanya memakan waktu, tetapi juga rentan terhadap kesalahan, sehingga dapat mengurangi fokus penulis pada substansi konten.

Penggunaan LaTeX menawarkan solusi yang efektif untuk mengatasi masalah ini. Sebagai \textit{typesetting system}, LaTeX dirancang untuk menghasilkan dokumen berkualitas tinggi dengan tata letak yang profesional dan konsisten. Dengan memanfaatkan class atau template yang sudah ada, penulis dapat memisahkan fokus antara konten dan presentasi. Oleh karenanya, dibuatlah template LaTeX yang dapat mempermudah proses penulisan makalah, sehingga penulis dapat lebih fokus pada isi tulisan dan penelitian yang dilakukan.

\section{Rumusan Masalah}

Berdasarkan latar belakang yang telah diuraikan, rumusan masalah dalam makalah ini adalah:

\begin{enumerate}
    \item Bagaimana cara merancang template LaTeX yang dapat memenuhi standar format penulisan makalah yang umum digunakan di Indonesia?
    \item Bagaimana template ini dapat membantu penulis untuk mengatur tata letak, daftar isi otomatis, penomoran halaman, dan daftar pustaka secara efisien?
    \item Fitur-fitur apa saja yang perlu diimplementasikan dalam template LaTeX ini agar dapat mempermudah proses penulisan karya ilmiah secara keseluruhan?
\end{enumerate}

\section{Tujuan}

Tujuan dari penyusunan template LaTeX ini adalah:

\begin{enumerate}
    \item Menciptakan sebuah template LaTeX yang konsisten, profesional, dan mudah digunakan untuk penulisan makalah, sesuai dengan standar yang berlaku.
    \item Menyediakan solusi praktis bagi mahasiswa dan akademisi agar dapat menyusun makalah dengan lebih cepat dan efisien, tanpa perlu khawatir tentang format penulisan.
    \item Memperkenalkan dan mempopulerkan penggunaan LaTeX sebagai alat bantu yang efektif dalam penulisan karya ilmiah.
\end{enumerate}
    %\chapter{Kajian Teori} % Dapat diganti menjadi "Kajian Pustaka" bila perlu

%=========================================================%
%                TULIS KAJIAN TEORI DI SINI               %
% Jangan Lupa untuk Menghapus Contoh Tulisan di Bawah Ini %
%=========================================================%

Bagian ini berfungsi sebagai fondasi teoritis makalah Anda. Di sini, Anda menunjukkan bahwa Anda telah melakukan riset mendalam dan memahami konteks topik yang Anda bahas.

\textbf{Landasan Teori:} Jelaskan teori atau konsep-konsep utama yang relevan dengan topik Anda. Misalnya, jika Anda menulis tentang pemasaran digital, jelaskan apa itu SEO, content marketing, dan social media engagement menurut para ahli. Gunakan definisi-definisi dari sumber-sumber terpercaya (buku teks, jurnal ilmiah).

Jika makalah Anda tidak memerlukan bagian Kajian Teori, Anda dapat menghilangkannya dengan menghapus/memberi \textit{comment} bagian \verb|\chapter{Kajian Teori} % Dapat diganti menjadi "Kajian Pustaka" bila perlu

%=========================================================%
%                TULIS KAJIAN TEORI DI SINI               %
% Jangan Lupa untuk Menghapus Contoh Tulisan di Bawah Ini %
%=========================================================%

Bagian ini berfungsi sebagai fondasi teoritis makalah Anda. Di sini, Anda menunjukkan bahwa Anda telah melakukan riset mendalam dan memahami konteks topik yang Anda bahas.

\textbf{Landasan Teori:} Jelaskan teori atau konsep-konsep utama yang relevan dengan topik Anda. Misalnya, jika Anda menulis tentang pemasaran digital, jelaskan apa itu SEO, content marketing, dan social media engagement menurut para ahli. Gunakan definisi-definisi dari sumber-sumber terpercaya (buku teks, jurnal ilmiah).

Jika makalah Anda tidak memerlukan bagian Kajian Teori, Anda dapat menghilangkannya dengan menghapus/memberi \textit{comment} bagian \verb|\chapter{Kajian Teori} % Dapat diganti menjadi "Kajian Pustaka" bila perlu

%=========================================================%
%                TULIS KAJIAN TEORI DI SINI               %
% Jangan Lupa untuk Menghapus Contoh Tulisan di Bawah Ini %
%=========================================================%

Bagian ini berfungsi sebagai fondasi teoritis makalah Anda. Di sini, Anda menunjukkan bahwa Anda telah melakukan riset mendalam dan memahami konteks topik yang Anda bahas.

\textbf{Landasan Teori:} Jelaskan teori atau konsep-konsep utama yang relevan dengan topik Anda. Misalnya, jika Anda menulis tentang pemasaran digital, jelaskan apa itu SEO, content marketing, dan social media engagement menurut para ahli. Gunakan definisi-definisi dari sumber-sumber terpercaya (buku teks, jurnal ilmiah).

Jika makalah Anda tidak memerlukan bagian Kajian Teori, Anda dapat menghilangkannya dengan menghapus/memberi \textit{comment} bagian \verb|\input{section/kajian-teori.tex}| dalam \textit{file} \texttt{main.tex}.| dalam \textit{file} \texttt{main.tex}.| dalam \textit{file} \texttt{main.tex}.
    \section{Pembahasan}

%=========================================================%
%               TULIS ISI PEMBAHASAN DI SINI              %
% Jangan Lupa untuk Menghapus Contoh Tulisan di Bawah Ini %
%=========================================================%

Ini adalah inti dari artikel Anda, tempat Anda menyajikan dan menganalisis data atau argumen Anda.

\textbf{Penyajian Data:} Sajikan data atau hasil riset Anda secara sistematis. Gunakan tabel, grafik, atau diagram untuk memvisualisasikan data agar lebih mudah dipahami.

\textbf{Analisis Temuan:} Analisis data yang telah disajikan. Hubungkan temuan Anda dengan teori-teori yang telah Anda paparkan di Kajian Pustaka. Jelaskan mengapa data tersebut muncul, apa artinya, dan apa implikasinya terhadap topik yang Anda teliti.

\textbf{Diskusi:} Bandingkan temuan Anda dengan hasil dari penelitian terdahulu. Apakah temuan Anda mendukung atau membantah penelitian sebelumnya? Diskusikan juga keterbatasan dari penelitian Anda dan kemungkinan faktor-faktor lain yang mungkin memengaruhi hasil.
    \section{Penutup}

%=========================================================%
%                  TULIS PENUTUP DI SINI                  %
% Jangan Lupa untuk Menghapus Contoh Tulisan di Bawah Ini %
%=========================================================%

\subsection{Kesimpulan}

Keluarga memiliki peran yang sangat signifikan dalam membentuk demokrasi yang beradab, terutama melalui pendidikan nilai-nilai dasar seperti kejujuran, tanggung jawab, dan toleransi. Sebagai sarana pendidikan pertama, keluarga mampu menanamkan nilai-nilai tersebut dengan karakter anak-anak yang peduli dan terbuka terhadap perbedaan. Praktik penerapan demokrasi sehari-hari di lingkungan keluarga dapat melatih anak-anak untuk memahami konsep demokrasi secara langsung. 

Namun, terdapat kendala yang menghambat peran keluarga dalam mendukung demokrasi. Pengaruh globalisasi dan media digital sering kali menjadi tantangan karena anak-anak rentan untuk terpapar dengan informasi yang tidak sesuai nilai demokrasi yang sehat. Kurangnya pemahaman orang tua terhadap konsep demokrasi juga dapat membatasi kemampuan mereka untuk menanamkan nilai-nilai tersebut. Perbedaan sosial dan budaya dalam setiap keluarga dapat menjadi hambatan jika tidak dikelola dengan baik. Oleh karena itu, diperlukan upaya melalui pendidikan demokrasi dan menjadi teladan yang baik untuk memperkuat peran keluarga dalam membangun demokrasi yang beradab.

\subsection{Saran}

Berdasarkan pemaparan tersebut, perlu adanya pemahaman mengenai demokrasi dan nilai keberadaban pada setiap keluarga, serta kesadaran untuk menerapkan pendidikan demokrasi sejak dini. Meski demikian, perlu dukungan eksternal dari masyarakat dan seluruh pihak yang dapat mendukung dan membangun demokrasi yang beradab.
    
    % Daftar Pustaka
    \nocite{*}
    \bibliography{reference}
    
\end{document}