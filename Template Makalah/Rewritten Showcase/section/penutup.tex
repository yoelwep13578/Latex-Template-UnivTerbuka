\chapter{Penutup}

%=========================================================%
%                  TULIS PENUTUP DI SINI                  %
% Jangan Lupa untuk Menghapus Contoh Tulisan di Bawah Ini %
%=========================================================%

\section{Kesimpulan}

Mahasiswa memiliki peran utama sebagai agen perubahan dan kontrol sosial terhadap aspek/sektor pendidikan, masyarakat sosial, ekonomi, politik, dan kebudayaan sesuai pada bidang yang diminatinya. Proses perwujudan tersebut tidak selalu berjalan mulus karena kendala dari faktor mahasiswa itu sendiri, faktor pendidikan, dan faktor sosial budaya. Oleh karena kendala tersebut, perlu adanya solusi dari mahasiswa dan pihak terkait untuk menangani kendala tersebut, dengan cara membangun karakter mahasiswa itu sendiri, membenahi kesenjangan dalam pendidikan, serta menguatkan nilai sosial dan budaya. Hal tersebut patut dilaksanakan secara komplit demi meraih perwujudan Indonesia Maju.

\section{Saran}

Untuk mahasiswa, alangkah baiknya untuk terus menguatkan karakter dan jati diri agar mahasiswa menjadi lebih tangguh dalam melaksanakan peran-perannya, serta dapat memperjuangkan dan mewujudkan Indonesia yang lebih maju. Untuk masyarakat, alangkah baiknya untuk tetap mempertahankan sikap terbuka dan ramah kepada semua mahasiswa yang sedang bertugas dalam pengabdian masyarakat untuk mendukung kontribusi mahasiswa sebagai agen perubahan dan kontrol sosialnya demi memajukan Indonesia di berbagai aspek sesuai bidangnya.