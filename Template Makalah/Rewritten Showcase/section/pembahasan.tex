\chapter{Pembahasan}

%=========================================================%
%               TULIS ISI PEMBAHASAN DI SINI              %
% Jangan Lupa untuk Menghapus Contoh Tulisan di Bawah Ini %
%=========================================================%

% Bagian 1
\section{Peran Mahasiswa dalam Mewujudkan Indonesia Maju}

Dalam kutipan dari beberapa jurnal, mahasiswa diharapkan mampu berperan sebagai agen perubahan dan kontrol sosial dalam masyarakat. Peran sebagai agen perubahan dan kontrol sosial mengharuskan mahasiswa untuk memiliki kepekaan yang baik dalam melihat penyimpangan yang terjadi di sekitarnya, serta melakukan inisiasi untuk menanggulangi penyimpangan tersebut sesuai kemampuannya. Peran tersebut umumnya didasari oleh bidang yang berkaitan dengan aspek/sektor berikut: 

\subsection{Pendidikan}

Dalam tahap belajarnya, mahasiswa akan dijumpai dengan pembuatan karya ilmiah dan penelitian. Karya tersebut umumnya dipublikasikan secara terbuka untuk semua orang. Dalam hal tersebut, mahasiswa berkontribusi dengan memberikan wawasan dan membagikan pengetahuan ke khalayak umum melalui karya tulisnya. 

Dalam kesempatan lain, mahasiswa dapat memperoleh tugas praktik untuk menjadi pengajar ataupun tutor sebaya di sekolah/tempat tertentu, serta ditugaskan untuk menganalisis hasil kinerja pengajaran dan kurikulumnya. Oleh karena itulah, mahasiswa dapat menjadi sarana dengan melakukan inisiatif menjadi tenaga pengajar, serta mewujudkan perubahan pembelajaran demi pendidikan yang lebih maju.

\subsection{Masyarakat Sosial}

Seperti yang telah disinggung dalam bagian pendahuluan, mahasiswa dapat diberi kesempatan untuk mengikuti kegiatan yang mengabdi kepada masyarakat, salah satunya kegiatan seperti Kuliah Kerja Nyata (KKN) atau sejenisnya. Kegiatan tersebut disediakan oleh pihak pendidikan dengan tujuan untuk membaurkan mahasiswa dengan masyarakat, sehingga dapat saling berinteraksi, belajar, mengajar, menghibur, dan mendapatkan pengalaman di desa. Oleh hal tersebut, mahasiswa dapat menjadi sarana dalam mengetahui kebutuhan dan keadaan sosial di desa setempat, melakukan inisiatif dalam bentuk rangkaian sosialisasi yang membahas toleransi, serta mengusulkan perubahan demi masyarakat sosial yang lebih maju dan merata ke seluruh daerah. 

\subsection{Ekonomi}

Dalam lingkungan kampusnya, mahasiswa kerap kali mengunjungi pedagang/penjual yang ada di dekat kampusnya untuk membeli sesuatu (umumnya membeli minuman atau makanan). Pedagang/penjual tersebut sebenarnya dapat dianggap sebagai Usaha Mikro Kecil Menengah (UMKM). Saat sepulang kuliah, kadang kala mahasiswa menyempatkan diri untuk mengajak mahasiswa lain ke tempat UMKM untuk membeli sesuatu yang diinginkan. Oleh karena itulah, mahasiswa secara tidak langsung dapat membantu perekonomian UMKM.

Dalam kesempatan tertentu, terdapat kejuruan/program studi yang mampu melibatkan mahasiswa untuk membuka wirausaha ataupun membangun UMKM-nya sendiri, bahkan mahasiswa dapat menerapkan inovasi dan penerapan teknologi terhadap usaha tersebut. Oleh karena hal tersebut, mahasiswa dapat membantu memajukan sektor perekonomian Indonesia menjadi lebih baik.

\subsection{Politik}

Dalam bagian sebelumnya telah disinggung bahwa mahasiswa diberikan kesempatan untuk memilih organisasi yang disediakan, salah satunya adalah Badan Eksekutif Mahasiswa (BEM) atau sejenisnya. BEM sendiri umumnya sekelompok mahasiswa yang berperan penting dan terhubung langsung dengan departemen dan kementerian dalam pemerintahan. Selain BEM, umumnya mahasiswa memiliki kepekaan terhadap suatu penyimpangan atau hal-hal yang terjadi dalam politik.

Dalam hal tersebut, mahasiswa berperan sebagai jembatan dalam menghubungkan rakyat dengan pemerintahan agar masyarakat atau mahasiswa itu sendiri dapat menyampaikan aspirasinya kepada pemerintahan, sehingga dapat memajukan politik Indonesia menjadi lebih baik.

\subsection{Kebudayaan}

Dalam kejuruan/program studi yang melibatkan budaya, mahasiswa diajarkan untuk melakukan penelitian dan dokumentasi terhadap budaya yang dirujuk. Dalam hal ini, mahasiswa dapat menelaah kebudayaan lokal yang ada di Indonesia dan mendokumentasikannya untuk pelestarian budaya. Perkembangan teknologi dapat membantu mahasiswa dalam dokumentasi dan memperkenalkan ragam kebudayaan lokal ke semua kalangan. Oleh karena itu, tidak menutup kemungkinan bahwa mahasiswa dapat memajukan budaya Indonesia dengan cara melestarikan dan mewariskan kepada generasi berikutnya.


% Bagian 2
\section{Kendala dalam Mewujudkan Indonesia Maju}

Proses yang dialami mahasiswa dalam mewujudkan Indonesia tidak selalu lancar seperti biasanya. Terdapat beberapa kendala yang menghambat proses dalam perwujudan Indonesia Maju. Kendala tersebut dapat dikelompokkan sebagai berikut.

\subsection{Faktor Internal Mahasiswa}

Pada dasarnya, setiap orang memiliki ciri kepribadian yang berbeda-beda. Bagian ini membahas tentang sifat kepribadian seseorang yang justru menghambat perwujudan Indonesia Maju.

Dalam sifat kepribadian, pola pikir mahasiswa sudah sepatutnya berkembang menjadi lebih dewasa dan lebih bertanggung jawab. Namun, masih ada saja pemikiran yang memaklumkan tindak “korupsi mini”, memaklumkan sifat tidak jujur, hingga pemikiran individualis yang melupakan kepentingan bersama. 

Contohnya, mahasiswa melihat beragam kejadian aneh di lingkup sekolah dasar-menengah yang mengindikasikan “kemunduran” Indonesia, seperti, lunturnya pengetahuan dasar, sebagian tidak mampu membaca, sebagian tidak mampu membaca jam analog, hingga murid dan orang tua yang enggan menerima teguran dari guru. Melihat hal ini, timbul rasa ingin menyerah pada mahasiswa karena sudah tidak tahu harus berbuat apa.

Dalam contoh lain, tidak semua mahasiswa memiliki kemampuan non-teknis yang baik. Kemampuan non-teknis yang dimaksud adalah kemampuan inisiatif, komunikasi, etika, dan kebiasaan. Oleh karena itulah, kemampuan ini perlu dilatih/dikembangkan sebagai fondasi dalam mewujudkan Indonesia Maju melalui aspek/sektor yang sesuai dengan minatnya. 

\subsection{Faktor Pendidikan}

Dalam lingkup sekolah, permasalahan tentang pendidikan telah tersebar secara cepat. Pada mulanya, Indonesia menerapkan kurikulum baru yang dinilai lebih membebaskan murid dengan memilih bidang pembelajaran sesuai minatnya. Namun, kurikulum tersebut terlalu bebas hingga membuat pelajar sekarang tidak mendapatkan ajaran tentang pengetahuan umum keindonesiaan.

Selain hal tersebut, terdapat kesenjangan lain yang memengaruhi seseorang dalam melanjutkan pendidikan ke perguruan tinggi. Misalnya, kendala terhadap biaya, akses perguruan tinggi yang terbatas dari zonanya (terlalu jauh), hingga keterbatasan program studi yang menimbulkan seseorang menjadi ragu untuk melanjutkan pendidikannya.

\subsection{Faktor Sosial Budaya}

Bagian ini sedikit terikat dengan faktor etika yang telah dibahas. Dalam momen tertentu, mahasiswa dapat merasa apatis terhadap suatu permasalahan yang sedang terjadi di Indonesia. Hal ini dapat terjadi karena kurangnya kejelasan tugas mahasiswa dalam proses memajukan Indonesia di bidangnya. Sifat apatis tersebut dapat menurunkan solidaritas mahasiswa dalam proses perjuangannya, sehingga partisipasi mahasiswa dalam mewujudkan Indonesia Maju menjadi menurun.

Dalam kasus lain, mahasiswa dapat menjumpai beragam perbedaan sosial dan budaya saat mengikuti tugas mengabdi masyarakat. Adapun perbedaan sosial dan budaya tersebut meliputi perbedaan bahasa, perbedaan logat bahasa, perbedaan kebiasaan masyarakat dengan mahasiswa yang dapat menjadi rintangan dalam mewujudkan kemajuan sosial.


% Bagian 3
\section{Solusi untuk Mengatasi Kendala dalam Peranan Mahasiswa}

\subsection{Membangun Karakter Individu Mahasiswa}

Untuk mengatasi hambatan dari faktor internal mahasiswa, maka mahasiswa perlu mengembangkan karakter dan kepribadian dirinya terlebih dulu. Kepribadian yang perlu diterapkan pada setiap mahasiswa dapat diawali dari sikap optimis, kemudian meningkatkan rasa kepekaan terhadap sekitar hingga menjadi sikap mahasiswa yang progresif.

Dalam segi ketangguhan, mahasiswa perlu memupuk rasa semangat dan jiwa pantang menyerah dalam menghadapi masalah yang sedang terjadi di Indonesia. Hal ini memiliki arti bahwa meskipun Indonesia dilanda oleh masalah seperti masalah pendidikan, mahasiswa tetap berpegang teguh dalam memperjuangkan kemajuan Indonesia demi mewujudkan Indonesia Maju.

Dalam segi kemampuan, mahasiswa perlu melatih dan mengembangkan kemampuan teknis dan non-teknis secara seimbang. Dalam beberapa buku, disebutkan bahwa kemampuan non-teknis berperan penting terhadap kesuksesan mahasiswa karena melibatkan kebiasaan, komunikasi, dan interaksi mahasiswa terhadap orang lain. Oleh karena itu, mahasiswa dapat meluangkan kesempatannya untuk mengembangkan kemampuan non-teknis sebagai fondasi kesuksesan dan partisipasi dalam mewujudkan Indonesia Maju.

\subsection{Membenahi Kesenjangan dalam Pendidikan}

Dalam bagian ini, perlu adanya peran eksternal dalam membenahi masalah dalam pendidikan. Dalam lingkup sekolah, perlu adanya perbaikan dan pembenahan kurikulum dalam jenjang taman kanak-kanak hingga sekolah menengah atas. Hal ini ditujukan agar pelajar memperoleh pengajaran tentang pengetahuan umum tentang keindonesiaan, sehingga pelajar dapat mengenal dan mengingat hal-hal penting dalam Indonesia dan mencapai Indonesia Maju dengan tujuan yang lebih jelas.

Dalam lingkup perkuliahan, kesenjangan pendidikan perlu diatasi bersama-sama dengan membuka peluang akses pendidikan kepada semua masyarakat yang bertekad melanjutkan pendidikan tanpa melalui syarat yang berbelit-belit, sehingga masyarakat dapat menerima pendidikan lanjutan yang mendukung proses dalam memajukan Indonesia.

\subsection{Menguatkan Sosial dan Budaya}

Mahasiswa perlu menghindari sikap apatis agar lebih peduli dan peka terhadap situasi dan masalah di sekitarnya. Dalam menghadapi perbedaan sosial dan budaya, mahasiswa dapat memberanikan diri untuk beradaptasi dengan cara berbaur secara aktif kepada masyarakat selama menjalankan tugas pengabdian. Solusi ini dapat diwujudkan melalui penerapan kemampuan non-teknis yang merujuk pada kemampuan komunikasi dan empati. Hal ini perlu didukung dengan memupuk rasa solidaritas terhadap mahasiswa dan masyarakat untuk memperjuangkan kemajuan sosial budaya bersama-sama.