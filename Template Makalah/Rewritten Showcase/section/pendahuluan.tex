\chapter{Pendahuluan}

%=========================================================%
%              TULIS ISI PENDAHULUAN DI SINI              %
% Jangan Lupa untuk Menghapus Contoh Tulisan di Bawah Ini %
%=========================================================%

\section{Latar Belakang}

Indonesia yang terdiri atas persatuan kebudayaan dan kekayaan alamnya tetap berusaha untuk berkembang menjadi negara yang lebih baik. Menurut informasi yang beredar, ada yang menyampaikan bahwa Indonesia sudah termasuk negara maju, tetapi ada juga yang menyampaikan bahwa Indonesia belum layak untuk dinobatkan sebagai negara maju karena beberapa kekurangan yang terjadi dalam bangsa. Penyimpangan tersebut membuat seluruh rakyat terus berjuang dalam memajukan bangsa Indonesia. Salah satu kelompok yang dapat membantu memajukan Indonesia adalah mahasiswa.

Mahasiswa dapat diartikan sebagai orang yang belajar dan bersekolah di perguruan tinggi. Dalam masa belajarnya, mahasiswa dapat mengikuti kegiatan atau organisasi yang menerapkan praktik pembinaan langsung sebagai bagian dari perwujudan Indonesia Maju. Contohnya kegiatan seperti Kuliah Kerja Nyata (KKN) yang menugaskan mahasiswa untuk berkunjung dan berbakti kepada masyarakat di desa, atau kelompok organisasi seperti Badan Eksekutif Mahasiswa (BEM) yang terhubung dengan departemen dan kementerian untuk membantu menyampaikan aspirasinya. Dari kegiatan/minat tersebutlah mahasiswa dapat memajukan aspek-aspek kecil secara bertahap, hingga kemajuan tersebut mampu membantu Indonesia dalam meraih status negara yang maju. Perwujudan Indonesia Maju juga sejalan dengan keinginan Indonesia dalam menggapai Indonesia Emas 2045. Meski demikian, ada beberapa masalah internal dari mahasiswa itu sendiri ataupun masalah eksternal yang menghambat perwujudan Indonesia Maju.

\section{Rumusan Masalah}

Berdasarkan latar belakang yang telah diuraikan, rumusan masalah dalam makalah ini adalah:

\begin{enumerate}[nosep]
    \item Apa yang membuat mahasiswa dapat berperan sebagai pendukung dalam mewujudkan Indonesia Maju?
    \item Apa saja kendala yang dihadapi dalam upaya mewujudkan Indonesia Maju?
    \item Bagaimana solusi yang dapat dilakukan untuk mengatasi kendala dalam perwujudan Indonesia Maju?
\end{enumerate}

\section{Tujuan}

Tujuan dari penyusunan makalah ini adalah:

\begin{enumerate}[nosep]
    \item Menjelaskan peran mahasiswa dalam mewujudkan Indonesia Maju.
    \item Mendeskripsikan kendala yang dihadapi dalam mewujudkan Indonesia Maju.
    \item Menjelaskan solusi untuk mengatasi kendala dalam perwujudan Indonesia Maju.
\end{enumerate}