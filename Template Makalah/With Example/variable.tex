%======================%
% NORMAL TEXT VARIABLE %
%======================%

\newcommand{\judul}{Judul Karya Tulis Makalah}

% Informasi Mata Kuliah, Sesi, TUgas, dan Tutor
\newcommand{\skemaTutorial}{Tutorial \textit{Online}}
\newcommand{\sesiKe}{3}
\newcommand{\tugasKe}{1}
\newcommand{\tanggalLengkap}{\today}
\newcommand{\tahun}{\the\year}

\newcommand{\namaMataKuliah}{Pengendalian Berbasis Desktop}
\newcommand{\kodeMataKuliah}{STSI9999}
\newcommand{\KodeMataKuliah}{STSI-9999}

\newcommand{\kodeKelas}{256}

\newcommand{\namaTutorPengampu}{Revo Wibowo, S.Pd., M.Pd., M.Sc.}

% Informasi Mahasiswa
\newcommand{\namaMahasiswa}{Yoeru Sandaru}
\newcommand{\nimMahasiswa}{081298765432}
\newcommand{\programStudi}{Sains Data}
\newcommand{\kodeProgramStudi}{/257}
\newcommand{\fakultas}{Sains dan Teknologi}
\newcommand{\kodeFakultas}{/127}
\newcommand{\utDaerah}{UT Medan}
\newcommand{\kodeUTDaerah}{/28}
\newcommand{\perguruanTinggi}{Universitas Terbuka}
\newcommand{\daearhMahasiswa}{Medan}

%==========================%
% ANOTHER COMMAND VARIABLE %
%==========================%

\newcommand{\linesskip}[1]{\vspace{#1\baselineskip}}
\newcommand{\titleindent}{1.24cm}

% Keterangan dan Sumber
\newcommand{\longcaption}[1]{\caption{\begin{tabular}[t]{@{}l@{}}#1\end{tabular}}}
\newcommand{\tablesource}[1]{\vspace{.3\baselineskip}\caption*{Sumber: #1}\vspace{-\baselineskip}}
\newcommand{\tablesourceleft}[2]{
    
    \raggedright\medskip\hspace{#1}\small Sumber: #2}
\newcommand{\figuresource}[1]{\vspace{-.9em}\caption*{Sumber: #1}\vspace{-.3\baselineskip}}
\newcommand{\lstsource}[1]{\begin{center}\vspace{-1.3\baselineskip}\singlespacing\small Sumber: #1 \vspace{.5\baselineskip}\end{center}}

% Notasi Matematika Tegak
\newcommand{\deriv}{\mathrm{d}}
\newcommand{\adj}[1]{\mathrm{adj}#1}
\newcommand{\euler}{\mathrm{e}}
\newcommand{\imaginary}{\mathrm{i}}

\newcommand{\upa}{\mathrm{a}}
\newcommand{\upb}{\mathrm{b}}
\newcommand{\upc}{\mathrm{c}}
\newcommand{\upd}{\mathrm{d}}
\newcommand{\upe}{\mathrm{e}}
\newcommand{\upf}{\mathrm{f}}
\newcommand{\upg}{\mathrm{g}}
\newcommand{\uph}{\mathrm{h}}
\newcommand{\upi}{\mathrm{i}}
\newcommand{\upj}{\mathrm{j}}
\newcommand{\upk}{\mathrm{k}}
\newcommand{\upl}{\mathrm{l}}
\newcommand{\upm}{\mathrm{m}}
\newcommand{\upn}{\mathrm{n}}
\newcommand{\upo}{\mathrm{o}}
\newcommand{\upp}{\mathrm{p}}
\newcommand{\upq}{\mathrm{q}}
\newcommand{\upr}{\mathrm{r}}
\newcommand{\ups}{\mathrm{s}}
\newcommand{\upt}{\mathrm{t}}
\newcommand{\upu}{\mathrm{u}}
\newcommand{\upv}{\mathrm{v}}
\newcommand{\upw}{\mathrm{w}}
\newcommand{\upx}{\mathrm{x}}
\newcommand{\upy}{\mathrm{y}}
\newcommand{\upz}{\mathrm{z}}

\newcommand{\upA}{\mathrm{A}}
\newcommand{\upB}{\mathrm{B}}
\newcommand{\upC}{\mathrm{C}}
\newcommand{\upD}{\mathrm{D}}
\newcommand{\upE}{\mathrm{E}}
\newcommand{\upF}{\mathrm{F}}
\newcommand{\upG}{\mathrm{G}}
\newcommand{\upH}{\mathrm{H}}
\newcommand{\upI}{\mathrm{I}}
\newcommand{\upJ}{\mathrm{J}}
\newcommand{\upK}{\mathrm{K}}
\newcommand{\upL}{\mathrm{L}}
\newcommand{\upM}{\mathrm{M}}
\newcommand{\upN}{\mathrm{N}}
\newcommand{\upO}{\mathrm{O}}
\newcommand{\upP}{\mathrm{P}}
\newcommand{\upQ}{\mathrm{Q}}
\newcommand{\upR}{\mathrm{R}}
\newcommand{\upS}{\mathrm{S}}
\newcommand{\upT}{\mathrm{T}}
\newcommand{\upU}{\mathrm{U}}
\newcommand{\upV}{\mathrm{V}}
\newcommand{\upW}{\mathrm{W}}
\newcommand{\upX}{\mathrm{X}}
\newcommand{\upY}{\mathrm{Y}}
\newcommand{\upZ}{\mathrm{Z}}

% Code Snippet Berwarna
\newcommand{\inlinesnippet}[1]{\colorbox{gray!10}{\lstinline‖#1‖}}
\newcommand{\verbsnippet}[1]{\colorbox{gray!10}{\verb|#1|}}

% Email AutoLink
\newcommand{\emailref}[1]{\href{mailto:#1}{#1}}

%=====================%
% PENGGANTIAN ISTILAH %
%=====================%

\addto\captionsindonesian{\renewcommand{\bibname}{Referensi}} % Bibliography
\addto\captionsindonesian{\renewcommand{\refname}{Referensi}} % Reference

\renewcommand{\chapterautorefname}{Bab}
\renewcommand{\sectionautorefname}{Bagian}
\renewcommand{\subsectionautorefname}{Sub-bagian}
\renewcommand{\subsubsectionautorefname}{Sub-sub-bagian}
\renewcommand{\paragraphautorefname}{Paragraf}
\renewcommand{\subparagraphautorefname}{Sub-paragraf}

\renewcommand{\figureautorefname}{Gambar}
\renewcommand{\tableautorefname}{Tabel}
\renewcommand{\equationautorefname}{Persamaan}
\renewcommand{\lstlistingname}{Kode}

\DeclareTblrTemplate{contfoot-text}{normal}{Lanjutan di halaman berikutnya}
\SetTblrTemplate{contfoot-text}{normal}
\DeclareTblrTemplate{conthead-text}{normal}{(Lanjutan)}
\SetTblrTemplate{conthead-text}{normal}

\DeclareTblrTemplate{remark-tag}{normal}{
    \UseTblrFont {remark-tag} \InsertTblrRemarkTag
}
\SetTblrTemplate{remark-tag}{normal}

%=======================%
% MODIFIKASI DAFTAR ISI %
%=======================%

\renewcommand{\cftchappresnum}{Bab }
\renewcommand{\cftchapfont}{\bfseries}
\renewcommand{\cftchapaftersnum}{:}
\setlength{\cftchapnumwidth}{3.8em}

%=============%
% COLOR NAMES %
%=============%

\definecolor{monokai-bg}{RGB}{250,250,250}     % Light gray background
\definecolor{monokai-comment}{RGB}{117,113,94} % Grayish comments
\definecolor{monokai-purple}{RGB}{137,89,168}  % Purple for keywords
\definecolor{monokai-green}{RGB}{58,110,59}    % Green for strings
\definecolor{monokai-orange}{RGB}{253,151,31}  % Orange for specials
\definecolor{monokai-blue}{RGB}{81,154,186}    % Blue for types/functions
