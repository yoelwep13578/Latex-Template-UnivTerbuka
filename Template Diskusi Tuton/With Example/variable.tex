%======================%
% NORMAL TEXT VARIABLE %
%======================%

\newcommand{\judul}{Lembar Jawaban Tutorial \textit{Online}}

% Informasi Mata Kuliah, Sesi, TUgas, dan Tutor
\newcommand{\sesiKe}{3}
\newcommand{\tugasKe}{1}
\newcommand{\tanggalLengkap}{\today}
\newcommand{\tahun}{\the\year}

\newcommand{\namaMataKuliah}{Mata Kuliah Berbasis Komputer}
\newcommand{\kodeMataKuliah}{STSI9999}
\newcommand{\KodeMataKuliah}{STSI-9999}

\newcommand{\kodeKelas}{256}

\newcommand{\namaTutorPengampu}{Revi Soekatno, S.Pd., M.Pd., M.Sc.}

% Informasi Mahasiswa
\newcommand{\namaMahasiswa}{Yoeru Sandaru}
\newcommand{\nimMahasiswa}{081298765432}
\newcommand{\programStudi}{Sains Data}
\newcommand{\kodeProgramStudi}{/257}
\newcommand{\fakultas}{Sains dan Teknologi}
\newcommand{\kodeFakultas}{/127}
\newcommand{\utDaerah}{UT Medan}
\newcommand{\kodeUTDaerah}{/28}
\newcommand{\perguruanTinggi}{Universitas Terbuka}

%==========================%
% ANOTHER COMMAND VARIABLE %
%==========================%

\newcommand{\linesskip}[1]{\vspace{#1\baselineskip}}
\newcommand{\titleindent}{1.24cm}

% Keterangan dan Sumber
\newcommand{\longcaption}[1]{\caption{\begin{tabular}[t]{@{}l@{}}#1\end{tabular}}}
\newcommand{\tablesource}[1]{\caption*{Sumber: #1}}
\newcommand{\tablesourceleft}[2]{\raggedright\medskip\hspace{#1} Sumber: #2}
\newcommand{\figuresource}[1]{Sumber: #1}
\newcommand{\lstsource}[1]{\begin{center}\vspace{-1.4\baselineskip}\singlespacing Sumber: #1 \vspace{.5\baselineskip}\end{center}}

% Notasi Matematika Tegak
\newcommand{\deriv}{\mathrm{d}}
\newcommand{\adj}[1]{\mathrm{adj}#1}
\newcommand{\euler}{\mathrm{e}}
\newcommand{\imaginary}{\mathrm{i}}

%=====================%
% PENGGANTIAN ISTILAH %
%=====================%

\addto\captionsindonesian{\renewcommand{\bibname}{Referensi}} % Bibliography
\addto\captionsindonesian{\renewcommand{\refname}{Referensi}} % Reference

\renewcommand{\figureautorefname}{Gambar}
\renewcommand{\tableautorefname}{Tabel}
\renewcommand{\equationautorefname}{Pernyataan}

\renewcommand{\lstlistingname}{Kode}

\DeclareTblrTemplate{contfoot-text}{normal}{Lanjutan di halaman berikutnya}
\SetTblrTemplate{contfoot-text}{normal}
\DeclareTblrTemplate{conthead-text}{normal}{(Lanjutan)}
\SetTblrTemplate{conthead-text}{normal}

\DeclareTblrTemplate{remark-tag}{normal}{
    \UseTblrFont {remark-tag} \InsertTblrRemarkTag
}
\SetTblrTemplate{remark-tag}{normal}

%===================================%
% ISTILAH DAlAM APA6 DAFTAR PUSTAKA %
%===================================%

\renewcommand{\onemaskedcitationmsg}[1]{%
    \emph{(#1\ sitasi dihilangkan untuk tinjauan anonim)}}%
\renewcommand{\maskedcitationsmsg}[1]{%
    \emph{(#1\ sitasi dihilangkan untuk tinjauan anonim)}}%
\renewcommand{\authorindexname}{Indeks Penulis}% Nama Indeks Penulis
%%
%% A note before the references if a meta-analysis is reported.
\renewcommand{\APACmetaprenote}{%
    Referensi yang ditandai dengan tanda bintang menunjukkan studi
    yang dimasukkan dalam meta-analisis.}%
%%
%% Commands for specific types of @misc entries.
\renewcommand{\bibmessage}{Pesan}%
\renewcommand{\bibcomputerprogram}{Program komputer}%
\renewcommand{\bibcomputerprogrammanual}{Manual program komputer}%
\renewcommand{\bibcomputerprogramandmanual}{Program dan manual komputer}%
\renewcommand{\bibcomputersoftware}{Perangkat lunak}%
\renewcommand{\bibcomputersoftwaremanual}{Manual perangkat lunak}%
\renewcommand{\bibcomputersoftwareandmanual}{Perangkat lunak dan manual}%
\renewcommand{\bibprogramminglanguage}{Bahasa pemrograman}%
%%
%% Other labels
\renewcommand{\bibnodate}{t.t.\hbox{}}%   % ``tanpa tanggal''
\renewcommand{\BIP}{dalam proses terbit}%            % ``in press''
\renewcommand{\BOthers}[1]{et al.\hbox{}}% ``dan lain-lain''
\renewcommand{\BOthersPeriod}[1]{et al.\hbox{}}% ``dan lain-lain'' dengan titik
\renewcommand{\BIn}{Dalam}%                  % for ``In '' editor...
\renewcommand{\Bby}{oleh}%                  % for ``by '' editor... (in reprints)
\renewcommand{\BED}{Penyunt.}%          % penyunting
\renewcommand{\BEDS}{Penyunt.}%        % para penyunting
\renewcommand{\BTRANS}{Penerj.}%    % penerjemah
\renewcommand{\BTRANSS}{Penerj.}%   % para penerjemah
\renewcommand{\BTRANSL}{terj.}%   % terjemahan, for the year field
\renewcommand{\BCHAIR}{Ketua}%            % ketua simposium
\renewcommand{\BCHAIRS}{Para Ketua}%          % para ketua
\renewcommand{\BVOL}{Jilid}%        % jilid
\renewcommand{\BVOLS}{Jilid}%        % jilid
\renewcommand{\BNUM}{No.\hbox{}}%         % nomor
\renewcommand{\BNUMS}{No.\hbox{}}%       % nomor
\renewcommand{\BEd}{ed.\hbox{}}%          % edisi
\renewcommand{\BCHAP}{bab}%        % bab
\renewcommand{\BCHAPS}{bab}%       % bab
\renewcommand{\BPG}{h.\hbox{}}%           % halaman
\renewcommand{\BPGS}{hlm.\hbox{}}%         % halaman
%% Default technical report type name.
\renewcommand{\BTR}{Lap. Tek.}
%% Default PhD thesis type name.
\renewcommand{\BPhD}{Disertasi doktor}
%% Default unpublished PhD thesis type name.
\renewcommand{\BUPhD}{Disertasi doktor tidak diterbitkan}
%% Default master's thesis type name.
\renewcommand{\BMTh}{Tesis magister}
%% Default unpublished master's thesis type name.
\renewcommand{\BUMTh}{Tesis magister tidak diterbitkan}
%%
\renewcommand{\BAuthor}{Penulis}% ``Penulis'' jika penerbit = penulis
\renewcommand{\BOWP}{Karya asli diterbitkan}%
\renewcommand{\BREPR}{Dicetak ulang dari}%
\renewcommand{\BAvailFrom}{Tersedia dari\ }%       Situs web; perhatikan spasi.
%% The argument is the date on which it was last checked.
\renewcommand{\BRetrieved}[1]{Diakses {#1}, dari\ }% Situs web; perhatikan spasi.
\renewcommand{\BRetrievedFrom}{Diakses dari\ }% Situs web; perhatikan spasi.
\renewcommand{\BMsgPostedTo}{Pesan dikirim ke\ }%     Pesan; perhatikan spasi.


%=============%
% COLOR NAMES %
%=============%

\definecolor{monokai-bg}{RGB}{250,250,250}     % Light gray background
\definecolor{monokai-comment}{RGB}{117,113,94} % Grayish comments
\definecolor{monokai-purple}{RGB}{137,89,168}  % Purple for keywords
\definecolor{monokai-green}{RGB}{58,110,59}    % Green for strings
\definecolor{monokai-orange}{RGB}{253,151,31}  % Orange for specials
\definecolor{monokai-blue}{RGB}{81,154,186}    % Blue for types/functions