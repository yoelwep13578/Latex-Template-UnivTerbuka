%==========%
% PREAMBLE %
%==========%

\documentclass[a4paper, 12pt]{article}

% Margin & Geometry
\usepackage[
    left=3cm,
    right=3cm,
    top=3cm,
    bottom=3cm
]{geometry}

% Linking & Reffering
\usepackage[hidelinks]{hyperref}

% Daftar Pustaka & APA 6
\usepackage{apacite}
\bibliographystyle{apacite}

% Bahasa APA 6
%\input{preset/APA-bahasa-indonesia.tex}
\renewcommand{\onemaskedcitationmsg}[1]{%
    \emph{(#1\ sitasi dihilangkan untuk tinjauan anonim)}}%
\renewcommand{\maskedcitationsmsg}[1]{%
    \emph{(#1\ sitasi dihilangkan untuk tinjauan anonim)}}%
\renewcommand{\authorindexname}{Indeks Penulis}% Nama Indeks Penulis
%%
%% A note before the references if a meta-analysis is reported.
\renewcommand{\APACmetaprenote}{%
    Referensi yang ditandai dengan tanda bintang menunjukkan studi
    yang dimasukkan dalam meta-analisis.}%
%%
%% Commands for specific types of @misc entries.
\renewcommand{\bibmessage}{Pesan}%
\renewcommand{\bibcomputerprogram}{Program komputer}%
\renewcommand{\bibcomputerprogrammanual}{Manual program komputer}%
\renewcommand{\bibcomputerprogramandmanual}{Program dan manual komputer}%
\renewcommand{\bibcomputersoftware}{Perangkat lunak}%
\renewcommand{\bibcomputersoftwaremanual}{Manual perangkat lunak}%
\renewcommand{\bibcomputersoftwareandmanual}{Perangkat lunak dan manual}%
\renewcommand{\bibprogramminglanguage}{Bahasa pemrograman}%
%%
%% Other labels
\renewcommand{\bibnodate}{t.t.\hbox{}}%   % ``tanpa tanggal''
\renewcommand{\BIP}{dalam proses terbit}%            % ``in press''
\renewcommand{\BOthers}[1]{et al.\hbox{}}% ``dan lain-lain''
\renewcommand{\BOthersPeriod}[1]{et al.\hbox{}}% ``dan lain-lain'' dengan titik
\renewcommand{\BIn}{Dalam}%                  % for ``In '' editor...
\renewcommand{\Bby}{oleh}%                  % for ``by '' editor... (in reprints)
\renewcommand{\BED}{Penyunt.}%          % penyunting
\renewcommand{\BEDS}{Penyunt.}%        % para penyunting
\renewcommand{\BTRANS}{Penerj.}%    % penerjemah
\renewcommand{\BTRANSS}{Penerj.}%   % para penerjemah
\renewcommand{\BTRANSL}{terj.}%   % terjemahan, for the year field
\renewcommand{\BCHAIR}{Ketua}%            % ketua simposium
\renewcommand{\BCHAIRS}{Para Ketua}%          % para ketua
\renewcommand{\BVOL}{Jilid}%        % jilid
\renewcommand{\BVOLS}{Jilid}%        % jilid
\renewcommand{\BNUM}{No.\hbox{}}%         % nomor
\renewcommand{\BNUMS}{No.\hbox{}}%       % nomor
\renewcommand{\BEd}{ed.\hbox{}}%          % edisi
\renewcommand{\BCHAP}{bab}%        % bab
\renewcommand{\BCHAPS}{bab}%       % bab
\renewcommand{\BPG}{h.\hbox{}}%           % halaman
\renewcommand{\BPGS}{hlm.\hbox{}}%         % halaman
%% Default technical report type name.
\renewcommand{\BTR}{Lap. Tek.}
%% Default PhD thesis type name.
\renewcommand{\BPhD}{Disertasi doktor}
%% Default unpublished PhD thesis type name.
\renewcommand{\BUPhD}{Disertasi doktor tidak diterbitkan}
%% Default master's thesis type name.
\renewcommand{\BMTh}{Tesis magister}
%% Default unpublished master's thesis type name.
\renewcommand{\BUMTh}{Tesis magister tidak diterbitkan}
%%
\renewcommand{\BAuthor}{Penulis}% ``Penulis'' jika penerbit = penulis
\renewcommand{\BOWP}{Karya asli diterbitkan}%
\renewcommand{\BREPR}{Dicetak ulang dari}%
\renewcommand{\BAvailFrom}{Tersedia dari\ }%       Situs web; perhatikan spasi.
%% The argument is the date on which it was last checked.
\renewcommand{\BRetrieved}[1]{Diakses {#1}, dari\ }% Situs web; perhatikan spasi.
\renewcommand{\BRetrievedFrom}{Diakses dari\ }% Situs web; perhatikan spasi.
\renewcommand{\BMsgPostedTo}{Pesan dikirim ke\ }%     Pesan; perhatikan spasi.

\renewcommand{\BBAB}{dan}% between authors in in-text citation

% Bahasa & Istilah
\usepackage[indonesian]{babel}

% Grafis
\usepackage{graphicx}
\usepackage{xcolor}

% Matematika / Math Mode
\usepackage{amsmath,amsthm,amssymb}
%\usepackage{MnSymbol}
\usepackage{xfrac}
\usepackage{unicode-math}
\allowdisplaybreaks

% Float
\usepackage{float}

% Tabel
\usepackage{tabularray}

\SetTblrStyle{caption}{font=\vspace{-\baselineskip}\singlespacing\small}
\SetTblrStyle{capcont}{font=\vspace{-\baselineskip}\singlespacing\small}
\SetTblrStyle{note}{font=\small}
\SetTblrStyle{remark}{font=\small}

%========================%
% FONT & FONT SELECTIONS %
%========================%

\usepackage[T1]{fontenc}
\usepackage{fontspec}

% Serif/Main Font --------------------------- %
\setmainfont{Times New Roman}[Ligatures=Rare] % TNR + Ligature
%\setmainfont{TeX Gyre Termes}                % Alternatif Mirip TNR
%\setmainfont{XITS}                           % Alternatif Mirip TNR

% Sans-Serif ---------------------------------%
\setsansfont{Noto Sans}[Scale=MatchLowercase] % Noto Sans
%\setsansfont{Calibri}[Scale=MatchUppercase]  % Calibri

% Monospace ------------------------------------ % 
%\setmonofont{Courier New}[Scale=MatchLowercase] % Courier New (Windows < 10)
%\setmonofont{Cascadia Code Light}[              % Cascadia Code (Windows >= 10)
%    BoldFont={Cascadia Code Bold},
%    ItalicFont={Cascadia Code Light Italic},
%    BoldItalicFont={Cascadia Code Bold Italic},
%    Scale=MatchLowercase
%]
\setmonofont{Fira Code Light}[                  % Fira Code (must installed)
    BoldFont={Fira Code Medium},
    Contextuals=Alternate,
    Scale=MatchLowercase
]

% Font Matematika -----
\setmathfont{XITS Math}

%============%
% CODE BLOCK %
%============%

\usepackage{listings}
\usepackage{lstfiracode}

\lstdefinelanguage{JavaScript}{ % Setting utk JavaScript karena tdk support
    keywords={typeof, new, true, false, catch, function, return, null, catch, switch, var, if, in, while, do, else, case, break},
    ndkeywords={class, export, boolean, throw, implements, import, this},
    sensitive=false,
    comment=[l]{//},
    morecomment=[s]{/*}{*/},
    morestring=[b]',
    morestring=[b]"
}

\lstset{
    style=FiraCodeStyle,
    basicstyle=\ttfamily,
    numberstyle=\footnotesize\color{gray},
    numberbychapter=false,
    captionpos=t,
    breaklines=true,
    frame={top|bottom},
    showstringspaces=false,
    commentstyle=\itshape\color{monokai-comment},
    keywordstyle=\bfseries\color{monokai-purple},
    keywordstyle={[2]{\itshape\bfseries\color{monokai-purple}}},
    keywordstyle={[3]{\itshape\bfseries\color{monokai-blue}}},
    ndkeywordstyle=\itshape\bfseries\color{monokai-orange},
    stringstyle=\color{monokai-green},
    identifierstyle=\color{black},
}

%============================%
% FORMATTING TEKS & PARAGRAF %
%============================%

% Parskip Paragraf
\usepackage[indent=1.24cm]{parskip}
%\setlength{\parskip}{.5\baselineskip}

% Line Spacing
\usepackage{setspace}

% Keterangan
\usepackage[font=small]{caption}

%============================%
% FORMATTING TEKS & PARAGRAF %
%============================%

\usepackage{titlesec}

% PILIH SATU
% Alphanumeric Numbering
\setcounter{secnumdepth}{5}

\renewcommand{\thesection}{\Roman{section}.}
\renewcommand{\thesubsection}{\Alph{subsection}.}
\renewcommand{\thesubsubsection}{\arabic{subsubsection}.}
\renewcommand{\theparagraph}{\alph{paragraph}.}
\renewcommand{\thesubparagraph}{\arabic{subparagraph})}

%\titleformat{\chapter}[block]
%{\normalfont\bfseries\centering\LARGE}
%{}
%{0em}
%{}

%\titleformat{\chapter}[display]
%{\normalfont\bfseries\centering\Large}
%{\MakeUppercase{Bab \thechapter}}
%{-.5em}
%{\MakeUppercase}

\titleformat{\section}[hang]
{\normalfont\bfseries\raggedright}
{\hspace{\titleindent}\llap{\parbox[b]{\titleindent}{\normalfont\bfseries\thesection\hfill}}}
{0em}
{\MakeUppercase}

\titleformat{\subsection}[hang]
{\normalfont\bfseries\raggedright}
{\hspace{\titleindent}\llap{\parbox[b]{\titleindent}{\normalfont\bfseries\thesubsection\hfill}}}
{0em}
{}

\titleformat{\subsubsection}[hang]
{\normalfont\bfseries\itshape\raggedright}
{\hspace{\titleindent}\llap{\parbox[b]{\titleindent}{\normalfont\bfseries\thesubsubsection\hfill}}}
{0em}
{}

\titleformat{\paragraph}[runin]
{\normalfont\bfseries\normalsize}
{\hspace{\titleindent}\llap{\parbox[b]{\titleindent}{\normalfont\bfseries\theparagraph\hfill}}}
{0em}
{}[.]

\titleformat{\subparagraph}[runin]
{\normalfont\bfseries\itshape\normalsize}
{\hspace{\titleindent}\llap{\parbox[b]{\titleindent}{\normalfont\bfseries\thesubparagraph\hfill}}}
{0em}
{}[.]
% Multilevel Numbering
%\setcounter{secnumdepth}{5}

\renewcommand{\thesection}{\arabic{section}.}
\renewcommand{\thesubsection}{\arabic{section}.\arabic{subsection}}

\titleformat{\section}[block]
{\normalfont\bfseries\raggedright\Large}
{\normalfont\bfseries\thesection}
{1em}
{\MakeUppercase}

\titleformat{\subsection}[block]
{\normalfont\bfseries\raggedright\Large}
{\normalfont\bfseries\thesubsection}
{1em}
{}

\titleformat{\subsubsection}[block]
{\normalfont\bfseries\itshape\raggedright\large}
{\normalfont\bfseries\thesubsubsection}
{1em}
{}

\titleformat{\paragraph}[runin]
{\normalfont\bfseries\normalsize}
{\normalfont\bfseries\theparagraph}
{1em}
{}[.]

\titleformat{\subparagraph}[runin]
{\normalfont\bfseries\itshape\normalsize}
{\normalfont\bfseries\thesubparagraph}
{1em}
{}[.]

% Spacing Heading
\titlespacing*{\section}{0pt}{*4}{*3}
\titlespacing*{\subsection}{0pt}{*2}{*1.5}
\titlespacing*{\subsubsection}{0pt}{*2}{*1.5}
\titlespacing*{\paragraph}{0pt}{*1.5}{.5em}
\titlespacing*{\subparagraph}{0pt}{*1.5}{.5em}

% List
\usepackage{enumitem}

\setlist[enumerate]{leftmargin=\parindent}
\setlist[itemize]{leftmargin=\parindent}

\newlist{essaylist}{enumerate}{2}
\setlist[essaylist,1]{
    leftmargin=\parindent, 
    labelwidth=\parindent, 
    labelsep=0pt, 
    align=left, 
    label=\arabic*.
    }
\setlist[essaylist,2]{
    leftmargin=0pt, 
    labelwidth=\parindent, 
    labelsep=0pt, 
    align=left, 
    label=\arabic{essaylisti}. \alph*.
    }

%============%
% PAGE STYLE %
%============%

\usepackage{fancyhdr}
\usepackage{lastpage}

\fancypagestyle{head-info}{
    \fancyhf{}
    \renewcommand{\headrulewidth}{0pt}
    \setlength{\topmargin}{-1.5cm}
    \setlength{\headsep}{1.5cm}
    \fancyhead[R]{
        \begin{tblr}{colspec={r l}, vline{2}={1}{solid, gray}, cells={valign=m}, colsep=14pt, columns={fg=gray}}
            {\textit{Tutorial Online} | \textit{Sesi \sesiKe} \\ \textit{\KodeMataKuliah} | \textit{\kodeKelas} | \textit{\namaMataKuliah}} & {\thepage\ / \pageref{LastPage}}
        \end{tblr} \hspace*{-2.6cm}
    }
}
\fancypagestyle{plain}[head-info]{}

% Load Variabel
%======================%
% NORMAL TEXT VARIABLE %
%======================%

\newcommand{\judul}{Metode Klasifikasi Jaringan Saraf Tiruan \textit{Backpropagation} Pada Mahasiswa Statistika Universitas Terbuka}

% Informasi Waktu
\newcommand{\tanggalLengkap}{\today}
\newcommand{\tahun}{\the\year}

% Informasi Mahasiswa
\newcommand{\namaMahasiswa}{Yoeru Sandaru}
\newcommand{\niMahasiswa}{081298765432}
\newcommand{\programStudi}{Matematika}
\newcommand{\fakultas}{Fakultas Sains dan Teknologi}
\newcommand{\utDaerah}{UT Medan}
\newcommand{\perguruanTinggi}{Universitas Terbuka}
\newcommand{\daerahMahasiswa}{Medan}
\newcommand{\negaraMahasiswa}{Indonesia}

\newcommand{\emailMahasiswa}{\niMahasiswa @ecampus.ut.ac.id} % Email E-Campus sesuai NIM

% Informasi Tutor/Dosen Pembimbing
\newcommand{\namaDosen}{Revo Wibowo}
\newcommand{\niDosen}{081298765432}
\newcommand{\programStudiDosen}{Matematika}
\newcommand{\fakultasDosen}{Fakultas Sains dan Teknologi}
\newcommand{\utDaerahDosen}{UT Medan}
\newcommand{\perguruanTinggiDosen}{Universitas Terbuka}
\newcommand{\daerahDosen}{Medan}
\newcommand{\negaraDosen}{Indonesia}

\newcommand{\emailDosen}{bowo@ecampus.ut.ac.id}

%==========================%
% ANOTHER COMMAND VARIABLE %
%==========================%

\newcommand{\linesskip}[1]{\vspace{#1\baselineskip}}
\newcommand{\titleindent}{1cm}

% Keterangan dan Sumber
\newcommand{\longcaption}[1]{\caption{\begin{tabular}[t]{@{}l@{}}#1\end{tabular}}}
\newcommand{\tablesource}[1]{\vspace{.3\baselineskip}\caption*{Sumber: #1}\vspace{-\baselineskip}}
\newcommand{\tablesourceleft}[2]{
    
    \raggedright\medskip\hspace{#1}\small Sumber: #2}
\newcommand{\figuresource}[1]{\vspace{-.9em}\caption*{Sumber: #1}\vspace{-.3\baselineskip}}
\newcommand{\lstsource}[1]{\begin{center}\vspace{-1.3\baselineskip}\singlespacing\small Sumber: #1 \vspace{.5\baselineskip}\end{center}}

% Notasi Matematika Tegak
\newcommand{\deriv}{\mathrm{d}}
\newcommand{\adj}[1]{\mathrm{adj}#1}
\newcommand{\euler}{\mathrm{e}}
\newcommand{\imaginary}{\mathrm{i}}

\newcommand{\upa}{\mathrm{a}}
\newcommand{\upb}{\mathrm{b}}
\newcommand{\upc}{\mathrm{c}}
\newcommand{\upd}{\mathrm{d}}
\newcommand{\upe}{\mathrm{e}}
\newcommand{\upf}{\mathrm{f}}
\newcommand{\upg}{\mathrm{g}}
\newcommand{\uph}{\mathrm{h}}
\newcommand{\upi}{\mathrm{i}}
\newcommand{\upj}{\mathrm{j}}
\newcommand{\upk}{\mathrm{k}}
\newcommand{\upl}{\mathrm{l}}
\newcommand{\upm}{\mathrm{m}}
\newcommand{\upn}{\mathrm{n}}
\newcommand{\upo}{\mathrm{o}}
\newcommand{\upp}{\mathrm{p}}
\newcommand{\upq}{\mathrm{q}}
\newcommand{\upr}{\mathrm{r}}
\newcommand{\ups}{\mathrm{s}}
\newcommand{\upt}{\mathrm{t}}
\newcommand{\upu}{\mathrm{u}}
\newcommand{\upv}{\mathrm{v}}
\newcommand{\upw}{\mathrm{w}}
\newcommand{\upx}{\mathrm{x}}
\newcommand{\upy}{\mathrm{y}}
\newcommand{\upz}{\mathrm{z}}

\newcommand{\upA}{\mathrm{A}}
\newcommand{\upB}{\mathrm{B}}
\newcommand{\upC}{\mathrm{C}}
\newcommand{\upD}{\mathrm{D}}
\newcommand{\upE}{\mathrm{E}}
\newcommand{\upF}{\mathrm{F}}
\newcommand{\upG}{\mathrm{G}}
\newcommand{\upH}{\mathrm{H}}
\newcommand{\upI}{\mathrm{I}}
\newcommand{\upJ}{\mathrm{J}}
\newcommand{\upK}{\mathrm{K}}
\newcommand{\upL}{\mathrm{L}}
\newcommand{\upM}{\mathrm{M}}
\newcommand{\upN}{\mathrm{N}}
\newcommand{\upO}{\mathrm{O}}
\newcommand{\upP}{\mathrm{P}}
\newcommand{\upQ}{\mathrm{Q}}
\newcommand{\upR}{\mathrm{R}}
\newcommand{\upS}{\mathrm{S}}
\newcommand{\upT}{\mathrm{T}}
\newcommand{\upU}{\mathrm{U}}
\newcommand{\upV}{\mathrm{V}}
\newcommand{\upW}{\mathrm{W}}
\newcommand{\upX}{\mathrm{X}}
\newcommand{\upY}{\mathrm{Y}}
\newcommand{\upZ}{\mathrm{Z}}

% Code Snippet Berwarna
\newcommand{\inlinesnippet}[1]{\colorbox{gray!10}{\lstinline‖#1‖}}
\newcommand{\verbsnippet}[1]{\colorbox{gray!10}{\verb‖#1‖}}

% Email AutoLink
\newcommand{\emailref}[1]{\href{mailto:#1}{#1}}

%=====================%
% PENGGANTIAN ISTILAH %
%=====================%

\addto\captionsindonesian{\renewcommand{\abstractname}{Abstrak}} % Abstract
\addto\captionsindonesian{\renewcommand{\bibname}{Daftar Pustaka}} % Bibliography
\addto\captionsindonesian{\renewcommand{\refname}{Daftar Pustaka}} % Reference

\renewcommand{\chapterautorefname}{Bab}
\renewcommand{\sectionautorefname}{Bagian}
\renewcommand{\subsectionautorefname}{Sub-bagian}
\renewcommand{\subsubsectionautorefname}{Sub-sub-bagian}
\renewcommand{\paragraphautorefname}{Paragraf}
\renewcommand{\subparagraphautorefname}{Sub-paragraf}

\renewcommand{\figureautorefname}{Gambar}
\renewcommand{\tableautorefname}{Tabel}
\renewcommand{\equationautorefname}{Persamaan}
\renewcommand{\lstlistingname}{Kode}

\DeclareTblrTemplate{contfoot-text}{normal}{Lanjutan di halaman berikutnya}
\SetTblrTemplate{contfoot-text}{normal}
\DeclareTblrTemplate{conthead-text}{normal}{(Lanjutan)}
\SetTblrTemplate{conthead-text}{normal}

\DeclareTblrTemplate{remark-tag}{normal}{
    \UseTblrFont {remark-tag} \InsertTblrRemarkTag
}
\SetTblrTemplate{remark-tag}{normal}

%====================%
% MODIFIKASI ABSTRAK %
%====================%

\makeatletter
\renewenvironment{abstract}{%
    \if@twocolumn
    \section*{\abstractname}%
    \else %% <- \small dihapus
    \begin{center}%
        {\bfseries \normalsize\abstractname\vspace{\z@}}%  %% <- \normalsize ditambah
    \end{center}%
    \quotation
    \fi}
{\if@twocolumn\else\endquotation\fi}
\makeatother


%=============%
% COLOR NAMES %
%=============%

\definecolor{monokai-bg}{RGB}{250,250,250}     % Light gray background
\definecolor{monokai-comment}{RGB}{117,113,94} % Grayish comments
\definecolor{monokai-purple}{RGB}{137,89,168}  % Purple for keywords
\definecolor{monokai-green}{RGB}{58,110,59}    % Green for strings
\definecolor{monokai-orange}{RGB}{253,151,31}  % Orange for specials
\definecolor{monokai-blue}{RGB}{81,154,186}    % Blue for types/functions


%============+&
% ISI DOKUMEN &
%=============&

\begin{document}
    
    % Line Spacing 1,5
    \onehalfspacing
    
    % Load Bagian
    \begin{titlepage}
    
    \centering
    
    % Judul
    {\Large\textbf{\MakeUppercase{\judul}}}
    
    \large
    \textbf{\MakeUppercase{Mata Kuliah \namaMataKuliah}} \\
    \textbf{\kodeMataKuliah}
    
    \normalsize 
    
    % Logo Perguruan Tinggi
    \vfill
    \includegraphics[width=.4\linewidth]{image/Logo_Universitas_Terbuka.png}
    \vfill
    
    {\large \textbf{TUTOR PENGAMPU}} \\
    
    \namaTutorPengampu
    
    \linesskip{1}
    
    {\large \textbf{DISUSUN OLEH}}
    
    \begin{tblr}{
            colspec={l c l}, 
            colsep=0pt, 
            rowsep=0pt, 
            column{2}={colsep=4pt}
            }
        Nama &:& \namaMahasiswa \\
        NIM &:& \nimMahasiswa \\
        Kode Kelas &:& \kodeKelas
    \end{tblr}
    
    \linesskip{2}
    
    \large
    \textbf{\MakeUppercase{Program Studi \programStudi}} \\
    \textbf{\MakeUppercase{Fakultas \fakultas}} \\
    \textbf{\MakeUppercase{UPBJJ \utDaerah}} \\
    \textbf{\MakeUppercase{\perguruanTinggi}} \\
    \textbf{\tahun}
    
    \normalsize
    
\end{titlepage}
    \pagestyle{head-info}
    \section{Soal}

%======================================================%
%                  TULIS SOAL DI SINI                  %
% Jangan Lupa untuk Menghapus Contoh Soal di Bawah Ini %
%======================================================%

Dikumpulkan dalam Bentuk Word dan tuliskan identitas sdr/i berikut dalam Program R. \\
Nama : \\
UT Daerah :

\noindent Bila terindikasi jawaban persis sama dengan mahasiswa lainnya maka akan diberikan nilai 20

\noindent Mahasiswa diberikan beberapa kasus dan mendiskusikan mengenai

\begin{enumerate}[nosep]
    \item Pembangkitan nilai suatu distribusi dengan nilai parameter tertentu
    \item Membuat plot fungsi distribusi suatu distribusi
\end{enumerate}

\subsection*{Pertanyaan 1}

Tuliskan perintah untuk membangkitkan data binomial dengan jumlah observasi $n$ adalah 30, banyaknya percobaan (\textit{size}) sebesar 10 dan peluang sukses sebesar 0.3, kemudian simpan hasilnya dalam matriks \inlinesnippet{data_binomial}, dengan jumlah kolom 5.

\subsection*{Pertanyaan 2}

Tuliskan perintah R untuk membuat plot fungsi kepadatan distribusi chisquare dengan derajat bebas 10, dimana $x$ berisi 100 data dengan rentang nilai 0 sampai dengan 5.

\subsection*{Pertanyaan 3}

Setelah Anda mempelajari materi sesi ini, buatlah 2 contoh Program R serta lampirkan hasil output R tsb.
    \chapter{Jawaban}

%=========================================================%
%                  TULIS JAWABAN DI SINI                  %
% Jangan Lupa untuk Menghapus Contoh Jawaban di Bawah Ini %
%=========================================================%

% Bagian 1

\section{Analisis Studi Kelayakan Aspek Pemasaran}

\subsection{Tingkat Persaingan di Area \textit{Outlet} Baru}

\subsubsection{Perkiraan Sumber/Keberadaan dan Jenis Pesaing}

\paragraph{Pesaing Langsung} 

Pesaing yang telah ada di sana (area \textit{outlet} baru) sejak lama yang memiliki bisnis makanan dengan jenis yang sama seperti Warung Lezat, serta “harga terjangkau” yang mampu mereka samakan dengan harga pasaran dari makanan kuliner nusantara Warung Lezat. Misalnya diambil contoh secara sembarang yaitu warung makan bernuansa daerah (Padang, Sunda, Tegal, dll.), serta beberapa warung makan lainnya.

\paragraph{Pesaing Tidak Langsung}

Keberadaannya sama seperti pesaing langsung yang telah hadir sejak lama di area tersebut, hanya saja dengan bisnis jenis kuliner yang berbeda. Misalnya diambil contoh secara sembarang yaitu beberapa bisnis kuliner selain nusantara berupa restoran modern, kafe, serta tempat makan lainnya yang mungkin berfokus melayani konsumen di waktu siang/malam saja.

\paragraph{Pesaing Baru}

Pesaing yang memiliki kemungkinan untuk “tiba” setelah \textit{outlet} Warung Lezat berdiri dan beroperasi di sana.

\subsubsection{Perkiraan Kekuatan \& Kelemahan Pesaing}

Perkiraan tentang kekuatan dan kelemahan yang dimiliki pesaing akan lebih sederhana kalau dijabarkan sebagai pertanyaan-pertanyaan berikut:

\paragraph{Apa Keunggulan yang Dimiliki Pesaing?}

Setiap pesaing yang ada di sana pastinya punya kekuatan tersendiri yang menjadi kiri khas tempat usahanya. Sebagai contoh, pesaing berupa warung makan kuliner bernuansa daerah mungkin kuat dalam segi harga, pelayanan, dan nama yang sudah dikenal masyarakat (\textit{brand awareness}), kemudian pesaing lain berupa bisnis makanan modern mungkin kuat dalam segi variasi menu, lokasi, atau layanan pengantaran makanannya.

\paragraph{Apa Kelemahan/Celah yang Bisa Dimanfaatkan?}

Kelemahan/hal-hal yang kurang pada pesaing biasanya bisa dijadikan celah untuk menutup kelemahan tersebut — bukan untuk membaguskan pesaing menjadi sempurna, melainkan sebagai “senjata” agar bisa memperoleh keuntungan.

\subsubsection{Target Pasar di Kota Besar}

\paragraph{Dari Tempat Strategis Konsumen}

Makanan kuliner nusantara Warung Lezat mungkin bisa ditujukan ke pasar/tempat yang terjangkau dan strategis bagi konsumen. Misalnya saja kalau tempatnya dekat dengan kantor mungkin target pasarnya cocok untuk pekerja kantoran di sana. Kalau tempatnya dekat dengan sekolah/kampus mungkin target pasarnya cocok untuk pelajar. Demikian seterusnya dengan menyesuaikan target konsumen dari tempatnya.

\paragraph{Dari Kebiasaan/Selera Konsumen}

Secara sederhana biasanya kebiasaan/selera konsumen terdiri dari yang suka cepat saji atau yang suka berlama-lama di tempat makan. Misalnya saja, apakah target pasarnya ditujukan untuk orang-orang yang hanya sekadar makan lalu pergi (cepat saji) atau ditujukan untuk orang yang benar-benar menikmati proses menyantap makanannya dan tempat makan yang ada.

\subsection{Strategi Promosi}

\subsubsection{Pemasaran \textit{Online}/Digital}

\paragraph{Melalui Media Sosial}

Misalnya dengan mem-posting beragam foto dan video makanan, suasana warung, serta testimoni pelanggan dari cabang sebelumnya di media sosial Instagram, TikTok, Facebook, dll. Kalau seandainya masih kurang mempan, mungkin bisa coba untuk membayar jasa iklan \textit{online} yang disediakan dari media sosialnya.

\paragraph{Bergabung dalam Jasa Pengiriman Makanan \textit{Online}}

Misalnya dengan mendaftarkan Warung Lezat (dengan lokasi \textit{outlet} yang baru) sebagai “anggota” dari jasa \textit{online} seperti GoFood, GrabFood, atau ShopeeFood supaya bisa menjangkau masyarakat yang ingin membelinya.

\paragraph{Mendaftarkan \textit{Outlet} di Peta Digital}

Misalnya dengan mendaftarkan lokasi \textit{outlet} di Maps untuk meyakinkan orang tentang lokasi dan kredibilitas \textit{outlet} Warung Lezat.

\subsubsection{Pemasaran Offline/Konvensional}

Salah satu cara pemasaran offline/konvensional yang lumayan bagus untuk dicoba (saat baru saja membuka \textit{outlet} di tempat baru) yaitu dengan mengadakan promo berupa diskon untuk hari pertama buka. Meski berpotensi dimanfaatkan orang untuk “cari murah”, tapi setidaknya orang-orang dapat membeli dan mencoba makanan kuliner nusantara yang disediakan oleh Warung Lezat.


% Bagian 2

\section{Analisis Studi Kelayakan Aspek Keuangan}

\subsection{Estimasi (Biaya) Investasi Awal}

Misalnya akan dijumlahkan semua biaya investasi yang telah disebutkan (dengan asumsi tanpa ada biaya-biaya lainnya)

\begin{center}
    \framebox(11cm, 3cm){
        \parbox{10cm}{
            Biaya Sewa Tempat \& Renovasi \hfill Rp50 000 000 \\
            Peralatan Dapur \& Perlengkapan \hfill Rp30 000 000 \\
            Modal Kerja Awal \hfill Rp20 000 000 \\
            Total \hfill Rp100 000 000
        }
    }
\end{center}

\subsection{Titik Impas (\textit{Break-Even Point})}

Adapun rumus dasar yang biasa dipakai untuk mengetahui \textit{BEP}/\textit{Break-Even Point} berbentuk seperti berikut:

\[
    \text{BEP} = \frac{\text{Biaya Tetap}}{\left(\text{Harga Jual per Unit} - \text{Biaya Variabel per Unit}\right)}
\]

Akan tetapi, rumus dasar tersebut perlu dimodifikasi agar lebih sesuai dengan analisis di Warung Lezat. Misalnya titik impasnya yang dicari adalah lamanya waktu yang diperlukan agar investasi yang dikeluarkan dapat tertutup dengan pendapatan bertahap (bulanan) — seperti hasil perhitungan dan gambarannya yang dikira-kira seperti berikut:

\begin{align*}
    \text{BEP} &= \frac{\text{Total Investasi Awal}}{\text{Pendapatan Bulanan}} \\
    &= \frac{100 000 000}{10 000 000} \\
    &= \boxed{10} \implies \boxed{10 \text{ Bulan}}
\end{align*}

Dari hasil tersebut, didapat gambaran sederhananya bahwa Warung Lezat perlu waktu 10 bulan untuk bisa menutup semua biaya investasi awal yang telah dikeluarkan (“balik modal”).

Dengan demikian, investasi sebesar Rp100 juta tersebut akan dikembalikan dengan penghasilan sebesar Rp10 juta per bulan (dalam waktu) selama 10 bulan. 


% Bagian 3

\section{Tantangan yang Mungkin Dihadapi dalam Pengelolaan SDM}

\subsection{Sulitnya Mencari Tenaga Kerja yang Pas}

Di daerah kota, biasanya terjadi kesulitan untuk mencari/merekrut karyawan dengan keahlian memasak menu Nusantara autentik yang konsisten, pelayanan pelanggan yang baik, dan yang dapat beradaptasi dengan lingkungan kerja di kota besar. Misalnya ingin mencari juru masak, kalau membuka lowongan di daerah kota biasanya cukup sulit untuk menemukan orang yang “sesuai kriteria” untuk menjadi juru masak kuliner nusantara.

\subsection{Persaingan “Lowongan” Tenaga Kerja}

Di daerah kota besar, biasanya masyarakat lebih berminat dengan pekerjaan yang dapat dikerjakan tanpa tekanan berlebih, gaji yang besar/cukup sesuai UMR, serta profesi yang memiliki tugas sederhana. Dari situasi tersebut, Warung Lezat mau tak mau harus berusaha menawarkan/membujuk orang untuk bekerja dengan memberinya gaji, tunjangan, serta lingkungan kerja yang tidak kalah bagus dengan tawaran pekerjaan lainnya.

\subsection{Orientasi dan Pelatihan}

Seandainya perekrutan tenaga kerja berhasil membuat orang di sekitar kota tersebut tertarik untuk bekerja di Warung Lezat, karyawan baru tersebut tentunya harus diberi “sambutan perkenalan” mengenai lingkungan dan tugas kerjanya, serta harus melatihnya agar dapat bekerja sesuai dengan tanggung jawabnya. Orientasi dan pelatihan tersebut biasanya butuh waktu yang cukup lama untuk benar-benar sukses mengubah karyawan baru menjadi karyawan yang “mahir.”

\subsection{Menjaga dan “Menahan” Karyawan yang Ada}

Semua karyawan (yang di tempat lama ataupun yang di \textit{outlet} baru ini) ada kalanya untuk mengalami rasa jenuh, penurunan produktivitas, hingga timbul rasa ingin resign dari Warung Lezat — entah karena merasa karirnya macet di situ-situ saja, perolehan upah yang tidak cukup untuk masa mendatang, dan lain sebagainya. Oleh karena itulah disebut sebagai tantangan dalam menjaga kesejahteraan karyawan agar selalu bersama.


% Bagian 4

\section{Analisis Studi Kelayakan Operasional}

\subsection{Tantangan dalam Pengadaan Bahan Baku}

\subsubsection{Akses Menuju Pemasok}

Berhubung \textit{outlet} tersebut masih baru berada di kota besar, biasanya belum begitu kenal dengan tempat/lokasi mana saja yang menyediakan pasokan bahan baku untuk kelancaran operasionalnya — mengingat bahwa mencari pemasok bahan baku khas Nusantara yang berkualitas tinggi dan cocok tidak mudah untuk ditemukan di kota besar.

\subsubsection{Harga Bahan Baku}

Harga bahan baku yang tersedia di kota besar biasanya lebih mahal dari tempat sebelumnya. Perbedaan harga antara tempat lama dengan tempat di kota baru ini dapat memengaruhi biaya produksi dan gambaran rencana yang telah dibuat sebelumnya.

\subsubsection{Perkiraan Kebutuhan Bahan Baku}

Di tempat lamanya, kebutuhan bahan baku sudah diukur secara pas agar tidak terlalu banyak dan juga tidak terlalu sedikit untuk menyajikan makanan kuliner nusantara. Di tempat/\textit{outlet} yang berada di kota besar, karena masih baru, biasanya belum bisa diukur secara aman — dan sewaktu-waktu Warung Lezat bisa memiliki bahan yang terpaksa basi karena terlalu banyak atau kekurangan sajian/porsi makanan untuk konsumen karena bahan yang telah habis.

\subsection{Hal yang Dapat Mendukung Kelayakan Operasional dalam Pengadaan Bahan Baku}

\subsubsection{Adanya Relasi dengan Pemasok Bahan Baku di Tempat Sebelumnya}

Selama berbisnis di tempat lamanya, Warung Lezat tentunya sudah memiliki hubungan dan relasi (baik secara langganan atau kerja sama) kepada pemasok yang ada di tempat lamanya. Untuk \textit{outlet} yang berlokasi di kota besar ini, tidak ada salahnya untuk memperoleh pasokan bahan baku dari pemasok bahan yang ada di tempat lamanya, asalkan dapat mengatur agar distribusinya efisien untuk antar-kota.

\subsubsection{Peluang untuk Langganan/Kerja Sama dengan Pemasok Bahan di Kota Besar}

Seandainya Warung Lezat berhasil menemukan satu/beberapa pihak pemasok bahan baku yang berada di sekitar \textit{outlet} baru dalam kota besar tersebut, maka pihak tersebut mungkin saja bersedia untuk bekerja sama dan menjadi pemasok setia bagi Warung Lezat. Pihak pemasok bisa dibuat yakin terhadap kerja sama tersebut dengan menunjukkan identitas dan \textit{brand} dari UMKM Warung Lezat itu sendiri bahwa \textit{outlet} barunya adalah cabang dari UMKM yang telah berdiri cukup lama.

\subsubsection{Peluang untuk Negosiasi (Harga, Porsi, Kualitas, dll.)}

Ketika kerja sama antar-pemasok bahan dari tempat lama hingga di kota besar ini telah berjalan cukup lama, mungkin bisa mencoba untuk melakukan negosiasi kepada pihak pemasok tersebut — bisa berupa harga yang dibuat lebih murah, porsi yang lebih banyak, kualitas yang lebih, atau meminta “perlakuan khusus” bagi Warung Lezat. Untuk meyakinkan negosiasi tersebut, bisa dengan menunjukkan identitas, \textit{brand} UMKM, dan capaian keuntungan Warung Lezat, lalu tidak lupa juga untuk memberi penawaran yang bisa saling menguntungkan antara Warung Lezat dengan pihak pemasok bahan.
    
    % Daftar Pustaka
    \nocite{*}
    \bibliography{reference}
    
    % Compile Mark
    %\vfill
    %{\footnotesize\noindent Di-\textit{compile} dengan \LaTeX\ pada \tanggalLengkap}
    
\end{document}