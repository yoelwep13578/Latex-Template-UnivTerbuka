\section{Soal}

%======================================================%
%                  TULIS SOAL DI SINI                  %
% Jangan Lupa untuk Menghapus Contoh Soal di Bawah Ini %
%======================================================%

Dikumpulkan dalam Bentuk Word dan tuliskan identitas sdr/i berikut dalam Program R. \\
Nama : \\
UT Daerah :

\noindent Bila terindikasi jawaban persis sama dengan mahasiswa lainnya maka akan diberikan nilai 20

\noindent Mahasiswa diberikan beberapa kasus dan mendiskusikan mengenai

\begin{enumerate}[nosep]
    \item Pembangkitan nilai suatu distribusi dengan nilai parameter tertentu
    \item Membuat plot fungsi distribusi suatu distribusi
\end{enumerate}

\subsection*{Pertanyaan 1}

Tuliskan perintah untuk membangkitkan data binomial dengan jumlah observasi $n$ adalah 30, banyaknya percobaan (\textit{size}) sebesar 10 dan peluang sukses sebesar 0.3, kemudian simpan hasilnya dalam matriks \inlinesnippet{data_binomial}, dengan jumlah kolom 5.

\subsection*{Pertanyaan 2}

Tuliskan perintah R untuk membuat plot fungsi kepadatan distribusi chisquare dengan derajat bebas 10, dimana $x$ berisi 100 data dengan rentang nilai 0 sampai dengan 5.

\subsection*{Pertanyaan 3}

Setelah Anda mempelajari materi sesi ini, buatlah 2 contoh Program R serta lampirkan hasil output R tsb.